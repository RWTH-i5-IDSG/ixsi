\chapter{Technical Realization}
\label{sec:TechnischeRealisierung}


\section{Message Encoding}
The messages are transferred as XML-files between both systems. A precise type-definition is set by the XML-scheme, corresponding to the interface.

In case the overhead, resulting of the introduction of XML turns problematic, there is the possibility to use the Efficient XML Interchange (EXI) protocol. The use of EXI would reduce the size of messages significantly, without loosing the advantages of XML. 

\section{Communication channel}
As the interface allows asynchronous subscription besides the usual request/response scheme, usage of the WebSocket protocol instead of plain HTTP is recommended.
WebSockets allow persistent connections between both systems and a bidirectional message exchange. Because the interface also provides an asynchronous subscription model, besides the request and response scheme and a low response rate is desirable, the WebSocket-protocol is recommended for the communication channel. WebSockets allow to establish a connection between two systems and exchange messages bidirectional. The VRS represents the server and the TIS the client (in HTTP context). In principal, any amount of communication channels can be opened. Updates of subscribed objects are delivered via the same connection as the one they are subscribed at. In case of an interrupt of the connection, the subscription ends.

\section{Authentication}
IXSI is a B2B interface and therefore does not include an internal authentication mechanism. Instead the communication partners are advised to use existing authentication mechanisms such as SSL certificates, a virtual private network (VPN) or HTTP authentication.

\subsection*{End-customer}
Because via IXSI, even user-controlled requests are transferred from system to system, it is necessary that a user can authenticate himself towards the VRS.
This usually happens via the tripel provider reference/user reference/password. To avoid typing the password in plaintext, alternatively to the password, a token can be used. This token can be generated by the VRS with the help of the function  TokenGeneration (vgl. \cref{subsec:Nachrichten:DienstC:Tokengenerierung}).
Thus, the user initially has to type his password once, afterwards he is able to authenticate himself via the generated token. For example, this can be saved on the user's device. 

In case of various consecutive requests which are invoked by the same user, a authentication should not be executed in every one. Therefore, with a user's first request, a session will be opened (explicitly via OpenSession or implicitly). For following requests of the same user, the ID of the opened session instead of an authentication can be transferred. These session have a time limited validity. In case this validity expires, the user needs to be identified again and a new session needs to be opened. In addition, a session can be closed explicitly via CloseSession.

\section{Connection Security}
To ensure confidentiality of the transferred data, encryption is required. Therefore, the SSL/TLS-protocol is suitable. This should be used in case that the connection is not already secured through other measures (e.g., by using VPN).
