\chapter{Technische Realisierung}
\label{sec:TechnischeRealisierung}


\section{Nachrichtenkodierung}
Die Nachrichten zwischen den beiden System werden als XML übertragen. Eine präzise Typdefinition wird durch das zur Schnittstelle gehörige XML Schema vorgegeben.

Falls sich der Overhead, der durch die Einführung von XML, entsteht als problematisch herausstellen sollte, besteht die Möglichkeit das Efficient XML Interchange (EXI) Protokoll einzusetzen. Der Einsatz von EXI würde die Größe der Nachrichten erheblich verringern, ohne die Vorteile der Verwendung von XML zu verlieren.

\section{Kommunikationskanal}
Da die Schnittstelle neben dem Anfrage- und Antwort-Schema auch ein asynchrones Abonnementmodell vorsieht und eine geringe Antwortzeit wünschenswert ist, wird für den Kommunikationskanal das WebSocket-Protokoll empfohlen. WebSockets erlauben es, eine bestehende Verbindung der beiden Systeme herzustellen und über diese bidirektional Nachrichten auszutauschen. Das FVS stellt den Server und das RIS den Client (im HTTP Kontext) dar. Es können im Prinzip beliebig viele Kommunikationskanäle geöffnet werden. Aktualisierungen von abonnierten Objekten werden über die gleiche Verbindung geliefert, über die sie abonniert wurden. Bei Unterbrechung der Verbindung endet das Abonnement.

\section{Authentifizierung}
IXSI ist als B2B-Schnittstelle konzipiert und enthält deswegen keinen eigenen Authentifizierungsmechanismus. Falls erforderlich können die Kommunikationspartner vorhandene Mechanismen wie eine SSL-Zertifikatsauthentifizierung (empfohlen), VPN oder HTTP Authentifizierung verwenden.

\subsection*{Endkunde}
Da per IXSI auch nutzergesteuerte Requests von System zu System geschickt werden, ist
es nötig, dass sich ein Nutzer gegenüber dem FVS authentifizieren kann.
Dies geschieht normalerweise über das Tripel Anbieter Referenz/Nutzer Referenz/Passwort. Um das  Passwort nicht im Klartext speichern zu müssen, kann alternativ zum Passwort ein Token
verwendet werden, welches über die Funktion TokenGeneration (vgl. \cref{subsec:Nachrichten:DienstC:Tokengenerierung}) vom FVS
generiert werden kann. So muss der Nutzer zwar initial sein Passwort einmalig eingeben, anschließend kann er sich aber über das daraus generierte, gespeicherte Token
authentifizieren. Dieses kann z.\,B. auf dem Endgerät des Nutzers gespeichert werden.

Bei aufeinanderfolgenden Anfragen, die vom selben Nutzer ausgelöst werden, soll nicht in jedem Request eine Authentifizierung durchgeführt werden. Daher wird mit dem ersten Request eines Nutzers eine Session eröffnet (explizit durch OpenSession oder implizit). Für Folgerequests desselben Nutzers kann dann anstatt einer Authentifizierung die ID der eröffneten Session mitgeschickt werden. Diese Sessions besitzen nur eine beschränkte zeitliche Gültigkeit. Nach Ablauf dieser Gültigkeit muss der Nutzer neu identifiziert und eine neue Session eröffnet werden. Eine Session kann auch explizit durch CloseSession
geschlossen werden.

\section{Verbindungssicherheit}
Um die Sicherheit der übermittelten Daten zu gewährleisten ist eine Verschlüsselung der Verbindung notwendig. Hierfür eignet sich SSL/TLS-Protokoll. Dieses sollte verwendet werden, wenn die Verbindung nicht bereits durch andere entsprechende Maßnahmen (z.\,B. durch die Verwendung von Virtuellen Privaten Netzwerken (VPN)) gesichert ist.