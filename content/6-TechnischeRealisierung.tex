\chapter{Technische Realisierung}
\label{sec:TechnischeRealisierung}
Die Schnittstelle verwendet die Standards XML und Websockets. Beide Standards werden bereits in verschiedenen Anwendungsbereichen erfolgreich eingestetzt.

\section{Nachrichtenkodierung}
Die Nachrichten zwischen den beiden System werden zur Übertragung als XML Dokumente dargestellt. Da es viele Werkzeuge zur Erzeugung und Verarbeitung von XML Dokumenten, vereinfacht dies die Realisierung der Schnittstelle. Des Weiteren erlaubt XML eine präzise Datentypdefinition, wodurch Uneindeutigkeiten vermieden werden können und eine grundlegende automatische Validierung von Nachrichten ermöglicht wird. Dadurch lassen sich insbesondere Fehler in der Implementierung einfacher feststellen. Falls sich der Overhead, der durch die Einführung von XML, entsteht als problematisch herausstellen sollte, besteht die Möglichkeit das Efficient XML Interchange (EXI) Protokoll einzusetzen. Der Einsatz von EXI würde die Größe der Nachrichten erheblich verringern, ohne die Vorteile der Verwendung von XML zu verlieren.

\section{Kommunikationskanal}
Da die Kommunikation nicht nach einem reinen Anfrage-Antwort-Schema abläuft, sondern beide System aktiv Nachrichten verschicken müssen (z.B. für Abo-Aktualisierungen) und eine geringe Antwortzeit wünschenswert ist, wird das WebSocket-Protokoll verwendet. Das WebSocket-Protokoll erlaubt es eine bestehende Verbindung der beiden Systeme herzustellen und über diese bidirektional Nachrichten auszutauschen.

\section{Verbindungssicherheit}
Die Schnittstelle kann auch über öffentliche Netze zur Verfügung gestellt werden. Um die Sicherheit der übermittelten Daten zu gewährleisten ist eine Verschlüsselung der Verbindung notwendig. Des Weiteren ist es notwendig sicherzustellen, dass die Schnittstelle nur von autorisierten Systemen genutzt werden kann. Beide Anforderungen werden vom SSL/TLS-Protokoll erfüllt. Dieses sollte verwendet werden, wenn die Verbindung nicht bereits durch andere entsprechende Maßnahmen (z.B. durch die Verwendung von Virtuellen Privaten Netzwerken (VPN)) gesichert ist.