\chapter{Nachrichten}
\label{cha:Nachrichten}
Die zwischen den Interaktionspartnern ausgetauschten Nachrichten basieren auf vier Nachrichtengrundtypen: Request, Response und Push. Alle Nachrichten vom RIS an das FVS stellen Anfragen dar und sind daher vom Typ Request. Der Typ Response wird für direkte Antworten des FVS auf Anfragen des RIS verwendet. Um eine Zuordnung der Response-Nachrichten zu den zugehörigen Request-Nachrichten zu erlauben, wird jede Request-Nachrichte mit einer eindeutigen Transaction-ID markiert, die in der zugehörigen Response-Nachricht wieder mitgegeben werden muss. Eine Response-Nachricht kann anstelle ihres normalen Inhalts auch einen Fehler enthalten. Der letzte Nachrichtentyp Push findet in den Fällen Verwendung, wenn das RIS fortlaufende Aktualisierungen (Abonnement) FVS angefordert hat. 

\section{Basisnachrichten}
\label{sec:Nachrichten:Basisnachrichten}

\subsection*{Basisklasse}
\input{xml/generated/IxsiMessageType}

\subsection*{Basisantwort (baseresponse)}
\input{xml/generated/AbstractBaseResponseType}

\medskip

\textit{Eine Basisklasse für einen Request ist nicht vorhanden, da nicht erforderlich.}

\section{Dienst 1}
\label{sec:Nachrichten:Dienst1}

\subsection*{Abfrage Buchungsziele}
\label{subsec:Nachrichten:Dienst1:BookingTargets}
\input{xml/generated/BookingTargetsInfoRequestType}
\input{xml/generated/BookingTargetsInfoResponseType}

\subsection*{Abfrage Änderungen Buchungszielen}
\label{subsec:Nachrichten:Dienst1:ChangeProviders}
\input{xml/generated/ChangedProvidersRequestType}
\input{xml/generated/ChangedProvidersResponseType}

\section{Dienst 2}
\label{subsec:Nachrichten:Dienst2}

\subsection*{Abfrage Verfügbarkeiten}
\label{subsec:Nachrichten:Dienst2:Availability}
\input{xml/generated/AvailabilityRequestType}
\input{xml/generated/AvailabilityResponseType}

\subsection*{Abfrage Stationskapazitäten (Dienst 2a)}
\label{subsec:Nachrichten:Dienst2:PlaceAvailability}
\input{xml/generated/PlaceAvailabilityRequestType}
\input{xml/generated/PlaceAvailabilityResponseType}

\section{Dienst 3}
\label{subsec:Nachrichten:Dienst3}

\subsection*{Verfügbarkeitsabonnement}
\label{subsec:Nachrichten:Dienst3:AvailabilitySubscription}
\input{xml/generated/AvailabilitySubscriptionRequestType}
\input{xml/generated/AvailabilitySubscriptionResponseType}
\input{xml/generated/AvailabilityPushMessageType}

\subsection*{Vollständige Verfügbarkeitsabonnementinformation}
\label{subsec:Nachrichten:Dienst3:CompleteAvailability}
\input{xml/generated/CompleteAvailabilityRequestType}
\input{xml/generated/CompleteAvailabilityResponseType}


\subsection*{Stationskapazitätsabonnement (Dienst 3a)}
\label{subsec:Nachrichten:Dienst3:PlaceAvailabilitySubscription}
\input{xml/generated/PlaceAvailabilitySubscriptionRequestType}
\input{xml/generated/PlaceAvailabilitySubscriptionResponseType}
\input{xml/generated/PlaceAvailabilityPushMessageType}

\subsection*{Vollständige Stationskapazitätsabonnementinformation (Dienst 3a)}
\label{subsec:Nachrichten:Dienst3:CompletePlaceAvailability}
\input{xml/generated/CompletePlaceAvailabilityRequestType}
\input{xml/generated/CompletePlaceAvailabilityResponseType}


\section{Dienst 4}
\label{subsec:Nachrichten:Dienst4}

\subsection*{Buchung}
\label{subsec:Nachrichten:Dienst4:Booking}
\input{xml/generated/BookingRequestType}
\input{xml/generated/BookingResponseType}

\subsection*{Buchungsänderung}
\label{subsec:Nachrichten:Dienst4:ChangeBooking}
\input{xml/generated/ChangeBookingRequestType}
\input{xml/generated/ChangeBookingResponseType}


\section{Dienst 5}
\label{subsec:Nachrichten:Dienst5}

\subsection*{Buchungsabonnement}
\label{subsec:Nachrichten:Dienst5:BookingAlertSubscription}
\input{xml/generated/BookingAlertSubscriptionRequestType}
\input{xml/generated/BookingAlertSubscriptionResponseType}
\input{xml/generated/BookingAlertPushMessageType}

\subsection*{Vollständige Buchungsabonnementinformation (Dienst 3a)}
\label{subsec:Nachrichten:Dienst5:CompleteBookingAlert}
\input{xml/generated/CompleteBookingAlertRequestType}
\input{xml/generated/CompleteBookingAlertResponseType}

\section{Dienst 6}
\label{subsec:Nachrichten:Dienst6}

\subsection*{Preisinformation}
\label{subsec:Nachrichten:Dienst6:PriceInformation}
\input{xml/generated/PriceInformationRequestType}
\input{xml/generated/PriceInformationResponseType}

\section{Dienst A}
\label{subsec:Nachrichten:DienstA}

\subsection*{Session öffnen}
\label{subsec:Nachrichten:DienstA:OpenSession}
\input{xml/generated/OpenSessionRequestType}
\input{xml/generated/OpenSessionResponseType}
\subsection*{Session schließen}
\label{subsec:Nachrichten:DienstA:CloseSession}
\input{xml/generated/CloseSessionRequestType}
\input{xml/generated/CloseSessionResponseType}

\section{Dienst B}
\label{subsec:Nachrichten:DienstB}

\subsection*{Heartbeat}
\label{subsec:Nachrichten:DienstB:HeartBeat}
\input{xml/generated/HeartBeatRequestType}
\input{xml/generated/HeartBeatResponseType}


\section{Dienst C}
\label{subsec:Nachrichten:DienstC}

\subsection*{Tokengenerierung}
\label{subsec:Nachrichten:DienstC:HeartBeat}
\input{xml/generated/TokenGenerationRequestType}
\input{xml/generated/TokenGenerationResponseType}