\chapter{Nachrichten}
\label{sec:Nachrichten}
Die zwischen den Interaktionspartnern ausgetauschten Nachrichten basieren auf vier Nachrichtengrundtypen: Handshake, Request, Response und Update. Vom Typ Handshake sind nur die Nachrichten, die während der Handshake-Interaktion auf Dienstebene 0 ausgetauscht werden. Alle Nachrichten vom RIS an das FVS stellen Anfragen dar und sind daher vom Typ Request. Der Typ Response wird für direkte Antworten des FVS auf Anfragen des RIS verwendet. Um eine Zuordnung der Response-Nachrichten zu den zugehörigen Request-Nachrichten zu erlauben, wird jede Request-Nachrichte mit einer eindeutigen Transaction-ID markiert, die in der zugehörigen Response-Nachricht wieder mitgegeben werden muss. Eine Response-Nachricht kann anstelle ihres normalen Inhalts auch einen Fehler enthalten. Der letzte Nachrichtentyp „Update“ findet in den Fällen Verwendung, wenn das RIS fortlaufende Aktualisierungen vom FVS angefordert hat. Dazu wird dem RIS in der Antwort auf die Anfrage der fortlaufenden Aktualisierungen eine Abo-ID mitgeteilt, die in jeder zugehörigen Update-Nachricht mitgegeben wird.

\input{xml/generated/messages.tex}

\section{Fehler}
Das FVS kann alternativ zu den normalen Response-, Update- und Handshake-Nachrichten auch im Fall eines Fehlers Fehlernachrichten an das RIS schicken. Eine Fehlernachricht besteht aus einer Fehlernummer, die den Typ des Fehlers definiert, einem kurzen und einem langen Fehlertext.

\input{xml/generated/error.tex}

\subsection{Fehlercodes}
Fehlercodes werden im Datentyp ErrorCodeType gespeichert.
\begin{flushleft}
\rowcolors{1}{}{gray!25}
\begin{tabularx}{\linewidth}{l>{\raggedright\arraybackslash}X} 
\toprule
Wert & Bedeutung\\
auth\_provider\_unknown & Authentifizierung: Unbekannte Provider-ID\\
auth\_invalid\_password & Authentifizierung: User-Passwort-Kombination ungültig\\
auth\_invalid\_token & Authentifizierung: User-Token-Kombination ungültig\\
auth\_session\_invalid & Authentifizierung: Session ist ungültig/abgelaufen\\
auth\_anon\_not\_allowed & Authentifizierung: Anonymer User nicht erlaubt\\
auth\_not\_authorized & Autorisation: Nutzer ist zu dieser Anfrage nicht berechtigt\\
sys\_backend\_failed & System: Hintergrundsystem antwortet nicht\\
sys\_unknown\_failure & System: unbekannter Fehler\\
sys\_not\_implemented & System: Request nicht implementiert\\
sys\_request\_not\_plausible & System: Request ist nicht plausibel. Dieser Wert sollte stets verwendet werden, wenn inhaltlich Fehler im Request enthalten sind.\\
booking\_target\_unknown & Buchungsziel unbekannt\\
price\_info\_not\_available & Preisinformationen nicht verfügbar\\
booking\_too\_short & Buchungsdauer zu kurz\\
booking\_too\_long & Buchungsdauer zu lang\\
booking\_target\_not\_available & Buchungsziel im gegebenen Zeitraum nicht buchbar\\
booking\_change\_not\_possible & Buchungsänderung kann nicht durchgeführt werden\\
booking\_id\_unknown & Unbekannte Buchungs-ID. Dieser Wert sollte auch verwendet werden, wenn die Buchungs-ID einem anderen User zugeordnet ist.\\
language\_not\_supported & Angefragte Sprache nicht vollständig unterstützt, andere Sprache geliefert.\\
\bottomrule
\end{tabularx}
\end{flushleft}

% \TODO{XML Beispiel}
% 
% \section{Ebene 0 -- Handshake}
% Der Handshake zwischen dem RIS und FVS besteht aus einer Folge von fünf Nachrichten. Nach dem Verbindungsaufbau durch das RIS sendet das FVS zunächst die unterstützen Protokollversionen in einer Nachricht vom Typ ProtocolVersionListHandshake mit der Bezeichnung ,,protocol-version-list-handshake''. Darauf antwortet das RIS mit der Auswahl einer Protokollversion in einer Nachricht vom Typ ProtocolVersionSelectHandshake mit der Bezeichnung ,,protocol-version-select-handshake''. Das FVS sendet nun den maximale unterstützte Dienstebene in einer Nachricht vom Typ ProtocolLevelMaxHandshake mit der Bezeichnung ,,protocol-level-max-handshake''. Das RIS antwortet darauf mit der ausgewählten Dienstebene in einer Nachricht vom Typ ProtocolLevelSelectHandshake mit der Bezeichnung ,,protocol-level-select-handshake''. Das FVS schließt den Handshake mit einer Nachricht vom Typ FinishHandshake und der Bezeichnung ,,finish-handshake'' ab. Alle Nachrichtentypen des Handshakes sind abgeleitet vom abstrakten Nachrichtentyp Handshake.
% 
% \input{xml/generated/handshake-messages.tex}
% 
% \subsection{Liste der unterstützten Protokollversionen}
% Eine Nachricht vom Typ ProtocolVersionListHandshake dient zur Übermittlung der Liste der vom FVS unterstützten Protokollversionen an das RIS. Sie enthält eine Liste aller vom FVS unterstützen Protokollversionen. Anstelle einer Nachricht dieses Typs kann ggf. ein Autorisierungsfehler (110, 111, 112) oder ein interner Serverfehler (130) gesendet werden.
% 
% \input{xml/generated/protocol-version-list-handshake.tex}
% 
% \subsection{Auswahl einer Protokollversion}
% Eine Nachricht vom Typ ProtocolVersionSelectHandshake dient zur Übermittlung der ausgewählten Protokollversion vom RIS zum FVS. Sie enthält nur die gewählte Protokollversion.
% 
% \input{xml/generated/protocol-version-select-handshake.tex}
% 
% \subsection{Maximale unterstützte Dienstebene}
% In einer Nachricht vom Typ ProtocolLevelMaxHandshake wird die vom FVS maximal unterstützte Dienstebene übermittelt. Die Nachricht enthält die Dienstebene. Anstelle der Nachricht kann ggf. auch einen Parameterfehler (140), Nicht-Verstanden-Fehler (120) oder internen Serverfehler (130) gesendet werden.
% 
% \input{xml/generated/protocol-level-max-handshake.tex}
% 
% \subsection{Auswahl einer Dienstebene}
% Nachrichten vom Typ ProtocolLevelSelectHandshake werden verwendet um die gewählte Dienstebene vom RIS an das FVS zu übermitteln. Dazu enthält die Nachricht die gewählte Dienstebene.
% 
% \input{xml/generated/protocol-level-select-handshake.tex}
% 
% \subsection{Bestätigung}
% Eine Nachricht vom Typ FinishHandshake dient zur erfolgreichen Beendigung des Handshake. Anstelle der Nachricht kann ggf. auch einen Parameterfehler (140), Nicht-Verstanden-Fehler (120) oder internen Serverfehler (130) gesendet werden. 
% 
% \input{xml/generated/finish-handshake.tex}
% 
% \TODO{XML Beispiel}
% 
% \section{Ebene 0 -- Abfrage der Betreiber}
% Das RIS kann eine Liste der Betreiber vom FVS abfragen. Dazu sendet zunächst das RIS eine Nachricht vom Typ OperatorListRequest mit der Bezeichnung „operator-list-request“. Das FVS antwortet auf diese Nachricht mit einer Nachricht vom Typ OperatorListResponse mit der Bezeichnung „operator-list-response“.
% 
% \input{xml/generated/operator-list-messages.tex}
% 
% \subsection{Anfrage Betreiberliste}
% Der Nachrichtentyp OperatorListRequest wird zur Anfrage von Betreiberlisten verwendet. Nachrichten dieses Typs können die Versions-ID einer aktuell bereits vorhandenen Betreiberliste enthalten.
% 
% \input{xml/generated/operator-list-request.tex}
% 
% \subsection{Antwort Betreiberliste}
% Der Nachrichtentyp OperatorListResponse dient zur Übermittlung einer Betreiberliste vom FVS zu RIS. Er enthält eine Betreiberliste und eine zugehörige Versions-ID. Alternativ kann eine Nachricht dieses Typs auch ein „no-change“ Element enthalten, falls sich die aktuelle Betreiberliste nicht von der Liste mit der in der Anfrage gegebenen Versions-ID  unterscheidet. Falls ein Fehler aufgetreten ist, kann anstelle der Nachricht auch einen Nicht-Verstanden-Fehler (120), internen Serverfehler (130) oder Parameterfehler (240, 241) gesendet werden.
% 
% \input{xml/generated/operator-list-response.tex}
% 
% \TODO{XML Beispiel}
% 
% \section{Ebene 0 -- Abfrage der Stationen}
% Das RIS kann eine Liste der Stationen vom FVS abfragen. Dazu sendet zunächst das RIS eine Nachricht vom Typ StationListRequest mit der Bezeichnung „station-list-request“. Das FVS antwortet auf diese Nachricht mit einer Nachricht vom Typ StationListResponse mit der Bezeichnung „station-list-response“.
% 
% \input{xml/generated/station-list-messages.tex}
% 
% \subsection{Anfrage Stationsliste}
% Der Nachrichtentyp StationListRequest wird zur Anfrage von Stationslisten verwendet. Nachrichten dieses Typs müssen eine Betreiber-ID enthalten. Diese legt den Betreiber fest, dessen Stationen abgefragt werden sollen. Die Nachrichten können die Versions-ID einer aktuell bereits vorhandenen Stationsliste enthalten. Des weiteren können die Nachrichten einen Postleitzahlenbereich enthalten, der für eine Einschränkung der Stationsliste genutzt werden soll.
% 
% \input{xml/generated/station-list-request.tex}
% 
% \subsection{Antwort Stationsliste}
% Der Nachrichtentyp StationListResponse dient zur Übermittlung einer Stationsliste vom FVS zu RIS. Er enthält eine Stationsliste und eine zugehörige Versions-ID. Alternativ kann eine Nachricht dieses Typs auch ein „no-change“ Element enthalten, falls sich die aktuelle Stationsliste nicht von der Liste mit der in der Anfrage gegebenen Versions-ID  unterscheidet. Die Liste muss genau die Stationen des in der Anfrage gegebenen Betreibers und ggf. die im gegebenen PLZ-Bereich liegen enthalten. Falls ein Fehler aufgetreten ist, kann anstelle der Nachricht auch einen Nicht-Verstanden-Fehler (120), internen Serverfehler (130) oder Parameterfehler (240, 241) gesendet werden.
% 
% \input{xml/generated/station-list-response.tex}
% 
% \TODO{XML Beispiel}
% 
% \section{Ebene 1 -- Abfrage der Zielgebiete}
% Das RIS kann eine Liste der Zielgebiete vom FVS abfragen. Dazu sendet zunächst das RIS eine Nachricht vom Typ TargetAreaListResponse mit der Bezeichnung „target-area-list-request“. Das FVS antwortet auf diese Nachricht mit einer Nachricht vom Typ TargetAreaListResponse mit der Bezeichnung „target-area-list-response“.
% 
% \input{xml/generated/target-area-messages.tex}
% 
% \subsection{Anfrage Zielgebietliste}
% Der Nachrichtentyp TargetAreaListRequest wird zur Anfrage von Zielgebietslisten verwendet. Nachrichten dieses Typs müssen eine Betreiber-ID enthalten. Diese legt den Betreiber fest, dessen Zielgebiete abgefragt werden sollen. Die Nachrichten können die Versions-ID einer aktuell bereits vorhandenen Zielgebietsliste enthalten.
% 
% \input{xml/generated/target-area-request.tex}
% 
% \subsection{Antwort Zielgebietliste}
% Der Nachrichtentyp TargetAreaListResponse dient zur Übermittlung einer Zielgebietsliste vom FVS zu RIS. Er enthält eine Zielgebietsliste und eine zugehörige Versions-ID. Alternativ kann eine Nachricht dieses Typs auch ein „no-change“ Element enthalten, falls sich die aktuelle Zielgebietsliste nicht von der Liste mit der in der Anfrage gegebenen Versions-ID  unterscheidet. Die Liste muss genau die Zielgebiete des in der Anfrage gegebenen Betreibers enthalten. Falls ein Fehler aufgetreten ist, kann anstelle der Nachricht auch einen Nicht-Verstanden-Fehler (120), internen Serverfehler (130) oder Parameterfehler (240, 241) gesendet werden.
% 
% \input{xml/generated/target-area-response.tex}
% 
% \TODO{XML Beispiel}
% 
% \section{Ebene 1 -- Abfrage der Buchungsziele}
% Das RIS kann eine Liste der Buchungsziele vom FVS abfragen. Diese wird nach der Übertragung des aktuellen Standes durch Update-Nachrichten aktualisiert. Dazu sendet zunächst das RIS eine Nachricht vom Typ ReservationTargetRequest mit der Bezeichnung „reservation-target-request“. Das FVS antwortet auf diese Nachricht mit einer Nachricht vom Typ ReservationTargetResponse mit der Bezeichnung „reservation-target-response“. Aktualisierungen sendet das FVS als Nachrichten vom Typ ReservationTargetUpdate mit der Bezeichnung „reservation-target-update“. Ein bestehendes Aktualisierungsabonnement kann durch das RIS mit einer Nachricht vom Typ ReservationTargetUpdateCancelRequest mit der Bezeichnung „reservation-target-update-cancel-request“. Diese wird durch eine Nachricht vom Typ ReservationTargetUpdateCancelResponse mit der Bezeichnung „reservation-target-update-cancel-response“ vom FVS bestätigt.
% 
% \input{xml/generated/reservation-target-messages.tex}
% 
% \subsection{Anfrage Buchungsziele}
% Buchungsziele werden durch eine Nachricht vom Typ ReservationTargetRequest angefordert. Diese muss eine Betreiber-ID enthalten. Diese schränkt die angefragten Buchungsziele auf Buchungsziele dieses Betreibers ein. Optional kann die Nachricht entweder einen Postleitzahlenbereich, ein GPS-Polygon oder eine Stations-ID enthalten. Diese dienen zur Filterung der zu übertragenden Buchungsziele.
% 
% \input{xml/generated/reservation-target-request.tex}
% 
% \subsection{Antwort Buchungszielliste}
% Nachrichten vom Typ ReservationTargetResponse enthalten eine Buchungsliste und eine Abo-ID, die in folgenden zugehörigen Update-Nachrichten zur Referenzierung verwendet wird. Falls ein Fehler aufgetreten ist, kann anstelle der Nachricht auch einen Nicht-Verstanden-Fehler (120), internen Serverfehler (130) oder Parameterfehler (240, 241) gesendet werden.
% 
% \input{xml/generated/reservation-target-response.tex}
% 
% \subsection{Aktualisierung eines Buchungsziels}
% Eine Aktualisierung der Buchungszielliste erfolgt durch Nachrichten vom Typ ReservationTargetUpdate. Wenn sich ein Buchungsziel geändert hat, enthält die Nachricht die veränderte Version des Buchungsziels. Die Buchungsziel-ID darf sich dabei nicht ändern. Wenn ein neues Buchungsziel hinzugekommen ist, enthält die Nachricht dieses Buchungsziel. Die Buchungsziel-ID des neuen Buchungsziels muss eindeutig über allen beim RIS noch vorliegenden Buchungszielen sein. Wenn ein Buchungsziel entfernt wurde, enthält die Nachricht nur eine Referenz auf das Buchungsziel in Form der Buchungsziel-ID. Der Inhalt des entfernten Buchungsziels wird nicht erneut übertragen.
% 
% \input{xml/generated/reservation-target-update.tex}
% 
% \subsection{Kündigung eines Buchungziels-Abos}
% Zur Kündigung eines Buchungszielabonnements wird vom RIS eine Nachricht vom Typ ReservationTargetUpdateCancelRequest verwendet. Diese enthält eine Referenz auf das Abo in Form einer Abo-ID.
% 
% \input{xml/generated/reservation-target-update-cancel-request.tex}
% 
% \subsection{Bestätigung der Kündigung}
% Eine Nachricht zur Kündigung eines Buchungszielabonnements wird vom FVS mit einer Nachricht vom Typ ReservationTargetUpdateCancelResponse bestätigt. Anstelle der  Nachricht kann auch einen Nicht-Verstanden-Fehler (120), internen Serverfehler (130) oder Parameterfehler (240) gesendet werde.
% 
% \input{xml/generated/reservation-target-update-cancel-response.tex}
% 
% \TODO{XML Beispiel}
% 
% \section{Ebene 2 -- Abfrage der Tarifliste}
% Das RIS kann eine Liste der Tarife vom FVS abfragen. Dazu sendet zunächst das RIS eine Nachricht vom Typ TariffListRequest mit der Bezeichnung „tariff-list-request“. Das FVS antwortet auf diese Nachricht mit einer Nachricht vom Typ TariffListResponse mit der Bezeichnung „tariff-list-response“.
% 
% \input{xml/generated/tariff-list-messages.tex}
% 
% \subsection{Anfrage Tarifliste}
% Der Nachrichtentyp TariffListRequest wird zur Anfrage von Tariflisten verwendet. Nachrichten dieses Typs müssen eine Betreiber-ID enthalten. Diese legt den Betreiber fest, dessen Tarife abgefragt werden sollen. Die Nachrichten können die Versions-ID einer aktuell bereits vorhandenen Tarifliste enthalten.
% 
% \input{xml/generated/tariff-list-request.tex}
% 
% \subsection{Antwort Tarifliste}
% Der Nachrichtentyp TariffListResponse dient zur Übermittlung einer Tarifliste vom FVS zu RIS. Er enthält eine Tarifliste und eine zugehörige Versions-ID. Alternativ kann eine Nachricht dieses Typs auch ein „no-change“ Element enthalten, falls sich die aktuelle Tarifliste nicht von der Liste mit der in der Anfrage gegebenen Versions-ID  unterscheidet. Die Liste muss genau die Tarife des in der Anfrage gegebenen Betreibers enthalten. Falls ein Fehler aufgetreten ist, kann anstelle der Nachricht auch einen Nicht-Verstanden-Fehler (120), internen Serverfehler (130) oder Parameterfehler (240, 241) gesendet werden.
% 
% \input{xml/generated/tariff-list-response.tex}
% 
% \TODO{XML Beispiel}
% 
% \section{Ebene 3 -- Authentifizierung eines Benutzers}
% Das RIS kann die Authentifizierung eines Benutzers beim FVS anfordern. Dazu sendet zunächst das RIS eine Nachricht vom Typ AuthenticationRequest mit der Bezeichnung „authentication-request“. Das FVS antwortet auf diese Nachricht mit einer Nachricht vom Typ AuthenticationResponse mit der Bezeichnung „authentication-response“.
% 
% \input{xml/generated/authentication-messages.tex}
% 
% \subsection{Anfrage Benutzerauthentifizierung}
% Eine Nachricht vom Typ AuthenticationRequest wird zur Anforderung der Authentifizierung eines Benutzers verwendet. Die Nachricht enthält eine Betreiber-ID, ein Identifikationsmerkmal des Benutzers (z.B. Name oder Email-Adresse) und ein Authorisierungsmerkmal des Benutzers (z.B. Passwort).
% 
% \input{xml/generated/authentication-request.tex}
% 
% \subsection{Antwort Benutzerinformationen}
% Eine Nachricht vom Typ AuthenticationResponse dient zur Übermittlung von Benutzerinformationen und Authentifizierungs-Token vom FVS zum RIS. Sie enthält einen Benutzerdatensatz. Falls ein Fehler aufgetreten ist, kann anstelle der Nachricht auch einen Nicht-Verstanden-Fehler (120), internen Serverfehler (130), Parameterfehler (240) oder Benutzerauthentifizierungsfehler (250, 251, 252) gesendet werden.
% 
% \input{xml/generated/authentication-response.tex}
% 
% \TODO{XML Beispiel}
% 
% \section{Ebene 3 -- Abfrage einer Preisauskunft}
% Das RIS kann eine Preisauskunft vom FVS abfragen. Dazu sendet zunächst das RIS eine Nachricht vom Typ PriceinfoRequest mit der Bezeichnung „price-info-request“. Das FVS antwortet auf diese Nachricht mit einer Nachricht vom Typ PriceInfoResponse mit der Bezeichnung „price-info-response“.
% 
% \input{xml/generated/price-info-messages.tex}
% 
% \subsection{Anfrage Preis}
% Eine Nachricht vom Typ PriceInfoRequest wird zur Anforderung einer Preisinformation verwendet. Die Nachricht enthält:
% \begin{itemize}
% \item eine Betreiber-ID,
% \item eine User-ID als Referenz auf den Benutzer, für den die Preisauskunft gelten soll,
% \item eine Buchungsziel-ID als Referenz auf das Buchungsziel, auf das sich die Preisauskunft beziehen soll,
% \item eine Tarif-ID als Referenz auf den Tarif, auf den sich die Preisauskunft beziehen soll,
% \item einen Zeitbereich, für den das Fahrzeug ausgeliehen werden soll,
% \item eine Distanz, die mit dem Fahrzeug zurückgelegt werden soll
% \item und ein optionales Ziel, an dem das Fahrzeug zurückgegeben werden soll.
% \end{itemize}
% 
% \input{xml/generated/price-info-request.tex}
% 
% \subsection{Antwort Preisauskunft}
% Eine Nachricht vom Typ PriceInfoResponse dient zur Übermittlung von Preisinformationen vom FVS zum RIS. Sie enthält einen Preisinformationsdatensatz bestehend aus Kostenbeträgen für die Anmietung des Fahrzeugs, die zurückgelegte Distanz, die Mietzeit und Sonstiges. Falls ein Fehler aufgetreten ist, kann anstelle der Nachricht auch einen Nicht-Verstanden-Fehler (120), internen Serverfehler (130) oder Parameterfehler (240) gesendet werden.
% 
% \input{xml/generated/price-info-response.tex}
% 
% \TODO{XML Beispiel}
% 
% \section{Ebene 3 -- Anfrage zur genauen Verfügbarkeit}
% Das RIS kann eine Verfügbarkeitsauskunft vom FVS abfragen. Dazu sendet zunächst das RIS eine Nachricht vom Typ AvailabilityRequest mit der Bezeichnung „availability-request“. Das FVS antwortet auf diese Nachricht mit einer Nachricht vom Typ AvailabilityResponse mit der Bezeichnung „availability-response“.
% 
% \input{xml/generated/availability-messages.tex}
% 
% \subsection{Anfrage Verfügbarkeit}
% Eine Nachricht vom Typ AvailabilityRequest wird zur Anforderung einer Verfügbarkeitsauskunft verwendet. Die Nachricht enthält:
% \begin{itemize}
% \item eine Betreiber-ID,
% \item eine User-ID als Referenz auf den Benutzer, für den die Verfügbarkeitsauskunft gelten soll,
% \item eine Buchungsziel-ID als Referenz auf das Buchungsziel, auf das sich die Preisauskunft beziehen soll,
% \item einen Zeitbereich, für den das Fahrzeug ausgeliehen werden soll
% \item und ein optionales Ziel, an dem das Fahrzeug zurückgegeben werden soll.
% \end{itemize}
% 
% \input{xml/generated/availability-request.tex}
% 
% \subsection{Antwort Verfügbarkeit}
% Eine Nachricht vom Typ AvailabilityResponse dient zur Übermittlung von Verfügbarkeitsinformationen vom FVS zum RIS. Sie enthält einen Verfügbarkeitsinformationsdatensatz bestehend aus einem Zeitbereich, für den ein Fahrzeug mit den bei der Anfrage angegebenen Parametern verfügbar ist, und einer Liste von Tarif-IDs, welche die möglichen Tarife für Buchungen darstellen. Der Zeitbereich muss mindestens den Zeitbereich, der bei der Anfrage angegeben wurde umfassen. Falls ein Fehler aufgetreten ist, kann anstelle der Nachricht auch einen Nicht-Verstanden-Fehler (120), internen Serverfehler (130) oder Parameterfehler (240) gesendet werden.
% 
% \input{xml/generated/availability-response.tex}
% 
% \TODO{XML Beispiel}
% 
% \section{Ebene 3 -- Auftrag zur Buchung}
% Das RIS kann eine Buchung beim FVS durchführen. Dazu sendet zunächst das RIS eine Nachricht vom Typ ReservationRequest mit der Bezeichnung „reservation-request“. Das FVS antwortet auf diese Nachricht mit einer Nachricht vom Typ ReservationResponse mit der Bezeichnung „reservation-response“.
% 
% \input{xml/generated/reservation-messages.tex}
% 
% \subsection{Anfrage Buchung}
% Eine Nachricht vom Typ ReservationRequest wird zur Anfrage für die Durchführung einer Buchung verwendet. Die Nachricht enthält:
% \begin{itemize}
% \item eine Betreiber-ID,
% \item eine User-ID als Referenz auf den Benutzer, für den die Buchung durchgeführt werden soll,
% \item ein Authentifizierungs-Token, zur Authentifizierung des Benutzers,
% \item eine Tarif-ID als Referenz auf den Tarif, der für die Buchung verwendet werden soll,
% \item eine Buchungsziel-ID als Referenz auf das Buchungsziel, auf das sich die Buchung beziehen soll,
% \item einen Zeitbereich, für den das Fahrzeug ausgeliehen werden soll
% \item und ein optionales Ziel, an dem das Fahrzeug zurückgegeben werden soll.
% \end{itemize}
% 
% \input{xml/generated/reservation-request.tex}
% 
% \subsection{Antwort Buchung}
% Eine Nachricht vom Typ ReservationResponse dient zur Bestätigung der Durchführung einer Buchung und zur einmaligen Übermittlung der Buchungsinformationen vom FVS zum RIS. Sie enthält einen Buchungsinformationsdatensatz. Die Informationen des Datensatzes müssen den Informationen die in der zugehörigen Anfrage angegeben wurden entsprechen. Falls ein Fehler aufgetreten ist, kann anstelle der Nachricht auch einen Nicht-Verstanden-Fehler (120), internen Serverfehler (130), Parameterfehler (240), Benutzer-Token-Fehler (260) oder Buchungsfehler (271) gesendet werden.
% 
% \input{xml/generated/reservation-response.tex}
% 
% \TODO{XML Beispiel}
% 
% \section{Ebene 3 -- Anfrage für Aktualisierungen zu einer Buchung}
% Das RIS kann Aktualisierungen zu einer Buchung vom FVS abonnieren. Dazu sendet zunächst das RIS eine Nachricht vom Typ ReservationSubscribeRequest mit der Bezeichnung ,,reservation-subscribe-request''. Das FVS antwortet auf diese Nachricht mit einer Nachricht vom Typ ReservationSubscribeResponse mit der Bezeichnung ,,reservation-subscribe-response''. Aktualisierungen sendet das FVS als Nachrichten vom Typ ReservationUpdate mit der Bezeichnung ,,reservation-update''. Ein bestehendes Aktualisierungsabonnement kann durch das RIS mit einer Nachricht vom Typ ReservationUpdateCancelRequest mit der Bezeichnung „reservation-update-cancel-request“. Diese wird durch eine Nachricht vom Typ ReservationUpdateCancelResponse mit der Bezeichnung „reservation-update-cancel-response“ vom FVS bestätigt.
% 
% \input{xml/generated/reservation-update-messages.tex}
% 
% \subsection{Anfrage zur Abonnierung einer Buchung}
% Ein Abonnement für Informationen zu einer Buchung werden durch eine Nachricht vom Typ ReservationSubscribeRequest angefordert. Diese muss eine Betreiber-ID und Buchungs-ID enhalten um die zu abonnierende Buchung eindeutig zu identifizieren.
% 
% \input{xml/generated/reservation-subscribe-request.tex}
% 
% \subsection{Antwort zur Abonnierung einer Buchung}
% Nachrichten vom Typ ReservationSubscribeResponse enthalten einen Buchungsinformationsdatensatz und eine Abo-ID, die in folgenden zugehörigen Update-Nachrichten zur Referenzierung verwendet wird. Falls ein Fehler aufgetreten ist, kann anstelle der Nachricht auch einen Nicht-Verstanden-Fehler (120), internen Serverfehler (130) oder Parameterfehler (240, 241) gesendet werden.
% 
% \input{xml/generated/reservation-subscribe-response.tex}
% 
% \subsection{Aktualisierung der Buchungsinformationen}
% Eine Aktualisierung der Buchungsinformationen erfolgt durch Nachrichten vom Typ ReservationUpdate. Wenn sich die Buchung geändert hat, enthält die Nachricht die veränderte Version der Buchung. Die Buchungs-ID darf sich dabei nicht ändern. Wenn die Buchung entfernt wurde, enthält die Nachricht nur eine Referenz auf die Buchung in Form der Buchungs-ID. Der Inhalt des entfernten Buchung wird nicht erneut übertragen.
% 
% \input{xml/generated/reservation-update.tex}
% 
% \subsection{Kündigung eine Buchungs-Abos}
% Zur Kündigung eines Buchungsabonnements wird vom RIS eine Nachricht vom Typ ReservationUpdateCancelRequest verwendet. Diese enthält eine Referenz auf das Abo in Form einer Abo-ID.
% 
% \input{xml/generated/reservation-update-cancel-request.tex}
% 
% \subsection{Bestätigung der Kündigung}
% Eine Nachricht zur Kündigung eines Buchungsabonnements wird vom FVS mit einer Nachricht vom Typ ReservationUpdateCancelResponse bestätigt. Anstelle einer Bestätigungsnachricht kann auch einen Nicht-Verstanden-Fehler (120), internen Serverfehler (130) oder Parameterfehler (240) gesendet werden.
% 
% \input{xml/generated/reservation-update-cancel-response.tex}
% 
% \TODO{XML Beispiel}
% 
% \section{Ebene 3 -- Auftrag zur Stornierung einer Buchung}
% Das RIS kann die Stornierung einer Buchung beim FVS durchführen. Dazu sendet zunächst das RIS eine Nachricht vom Typ ReservationCancelRequest mit der Bezeichnung „reservation-cancel-request“. Das FVS antwortet auf diese Nachricht mit einer Nachricht vom Typ ReservationCancelResponse mit der Bezeichnung „reservation-cancel-response“.
% 
% \input{xml/generated/reservation-cancel-messages.tex}
% 
% \subsection{Anfrage Stornierung}
% Eine Nachricht vom Typ ReservationCancelRequest wird zur Anfrage für die Durchführung einer Stornierung einer Buchung verwendet. Die Nachricht enthält:
% \begin{itemize}
% \item eine Betreiber-ID und Buchungs-ID zur eindeutigen Identifizierung der zu stornierenden Buchung,
% \item eine User-ID als Referenz auf den Benutzer, für den die Stornierung durchgeführt werden soll
% \item und ein Authentifizierungs-Token, zur Authentifizierung des Benutzers.
% \end{itemize}
% 
% \input{xml/generated/reservation-cancel-request.tex}
% 
% \subsection{Antwort Stornierung}
% Eine Nachricht vom Typ ReservationResponse dient zur Bestätigung der Durchführung der Stornierung einer Buchung durch das FVS. Falls ein Fehler aufgetreten ist, kann anstelle der Nachricht auch einen Nicht-Verstanden-Fehler (120), internen Serverfehler (130), Parameterfehler (240), Benutzer-Token-Fehler (260) oder Stornierungsfehler (272) gesendet werden.
% 
% \input{xml/generated/reservation-cancel-response.tex}
% 
% \TODO{XML Beispiel}
% 
% \section{Ebene 4 -- Auftrag zu Umbuchung einer Buchung}
% Das RIS kann eine Buchungsänderung beim FVS durchführen. Dazu sendet zunächst das RIS eine Nachricht vom Typ ReservationChangeRequest mit der Bezeichnung „reservation-change-request“. Das FVS antwortet auf diese Nachricht mit einer Nachricht vom Typ ReservationChangeResponse mit der Bezeichnung „reservation-change-response“.
% 
% \input{xml/generated/reservation-change-messages.tex}
% 
% \subsection{Anfrage Umbuchung}
% Eine Nachricht vom Typ ReservationChangeRequest wird zur Anfrage für die Durchführung einer Buchungänderung verwendet. Die Nachricht enthält:
% \begin{itemize}
% \item eine Betreiber-ID und Buchungs-ID zur eindeutigen Identifizierung der zu stornierenden Buchung,
% \item eine User-ID als Referenz auf den Benutzer, für den die Buchungsänderung durchgeführt werden soll,
% \item ein Authentifizierungs-Token, zur Authentifizierung des Benutzers,
% \item eine optionale Tarif-ID als Referenz auf den Tarif, falls dieser geändert werden soll,
% \item eine optionale Buchungsziel-ID als Referenz auf das Buchungsziel, falls dieses geändert werden soll,
% \item einen optionalen Zeitbereich, falls dieser geändert werden soll
% \item und ein optionales Ziel an dem das Fahrzeug zurückgegeben werden soll, falls dieses geändert werden soll.
% \end{itemize}
% 
% \input{xml/generated/reservation-change-request.tex}
% 
% \subsection{Antwort Umbuchung}
% Eine Nachricht vom Typ ReservationChangeResponse dient zur Bestätigung der Durchführung einer Buchungsänderung und zur einmaligen Übermittlung der neuen Buchungsinformationen vom FVS zum RIS. Sie enthält einen Buchungsinformationsdatensatz. Die Informationen des Datensatzes müssen den Informationen die in der zugehörigen Anfrage angegeben wurden entsprechen. Insbesondere muss die Buchungs-ID des Buchungsinformationsdatensatzes der in der Anfrage angegebenen Buchungs-ID entsprechen. Falls ein Fehler aufgetreten ist, kann anstelle der Nachricht auch einen Nicht-Verstanden-Fehler (120), internen Serverfehler (130), Parameterfehler (240), Benutzer-Token-Fehler (260), Buchungsfehler (271) oder Stornierungsfehler (272) gesendet werden.
% 
% \input{xml/generated/reservation-change-response.tex}
% 
% \TODO{XML Beispiel}