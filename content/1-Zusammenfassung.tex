\chapter{Zusammenfassung}
\label{cha:Zusammenfassung}

Ziel dieser Schnittstellenspezifikation ist die Kopplung von Verleihsystemen für Fahrzeuge des Individualverkehrs mit Reiseinformationssystemen. Der Grund für die Kopplung ist die Bedienung der trendgetriebenen Anforderung an intermodale Reiseketten,  hinsichtlich der Integration von Verleihsystemen.\\

Die Schnittstellenspezifikation besteht aus:\begin{itemize}
\item einem Rollenmodell der beteiligten Akteure,
\item einer Empfehlung für eine Dienstebenenhierarchie zur Bereitstellung unterschiedlicher Qualitäten der Informationskopplung,
\item der Interaktionsprotokolle zur Beschreibung der Nachrichtenabfolgen zwischen den beteiligten Akteuren zur Bereitstellung der Informationskopplung entsprechend der spezifizierten Dienstebenen,
\item der Beschreibung der zugrundeliegenden Datentypen für die Nachrichten der Interaktionsprotokolle,
\item der Spezifikation von geeigneten Technologien zur Darstellung der Daten, der Kommunikation zwischen den zu koppelnden Informationssystemen, der Implementierung der Interaktionsprotokolle und der abschließenden Verarbeitung der Informationen und
\item einer Reihe von Codetabellen, die erlaubte Werte für Aufzählungen enthält.
\end{itemize}
\bigskip
Diese Version von IXSI enthält die folgenden MoblityBroker-spezifischen Erweiterungen:
\begin{itemize}
\item Fahrzeugfreischaltung
\item Verbrauchsdatenaustausch
\item Abonnement von externen Buchungen
\item Anlegen und Sperren von FVS-Nutzern
\item Synchroniseren von Fahrzeugeinstellungen
\end{itemize}
Diese Version von IXSI enthält die folgenden Smartcar-spezifischen Erweiterungen:
\begin{itemize}
\item Synchroniseren von Fahrzeugeinstellungen
\item Fernkonfiguration Navigationssystem und Fahrtverlauf überwachen
\item Dialog ÖPNV Alternative (TODO)
\end{itemize}
