\chapter{Datenmodell}
\label{sec:Datenmodell}
Dieser Abschnitt beschreibt das der Schnittstelle zugrunde liegende Datenmodell. \medskip

\noindent Symbollegende:\\
$\medcircle$ Auswahl (\verb|xs:choice|)\\
$\medsquare$ Optional (\verb|minOccurs=0|)\\
$\boxbox$ Mehrwertig (\verb|maxOccurs=0|)\\


\section{Basisdatentypen}
\label{subsec:Datenmodell:Basis}

\subsection*{Einfache Basistypen}
Einfache Basistypen sind Aliasnamen für vorhandene Datentypen um eine semantische Unterscheidung zu erlauben.
\input{xml/generated/simpleTypes}
\input{xml/generated/simpleTypes-schema}
Erlaubte Werte für die Aufzählungen \emph{ClassType}, \emph{EngineType}, \emph{AttributeClassType} und \emph{ErrorCodeType} sind in \autoref{cha:CodeTabellen} dargestellt.

\subsection*{Text}
\input{xml/generated/TextType}
\input{xml/generated/TextType-schema}

\subsection*{Ortskoordinaten}
\input{xml/generated/CoordType}
\input{xml/generated/CoordType-schema}

\subsection*{Kreis}
\input{xml/generated/GeoCircleType}
\input{xml/generated/GeoCircleType-schema}

\subsection*{Rechteck}
\input{xml/generated/GeoRectangleType}
\input{xml/generated/GeoRectangleType-schema}

\subsection*{Gebiet}
\input{xml/generated/GeoAreaType}
\input{xml/generated/GeoAreaType-schema}

\subsection*{Gebiet (Einschluss / Ausschluss)}
\input{xml/generated/IncExcGeoAreaType}
\input{xml/generated/IncExcGeoAreaType-schema}

\subsection*{Zeitperiode}
\input{xml/generated/TimePeriodType}
\input{xml/generated/TimePeriodType-schema}

\subsection*{Zeitperiode (Vorschlag)}
\input{xml/generated/TimePeriodProposalType}
\input{xml/generated/TimePeriodProposalType-schema}

\subsection*{Herkunft/Ziel}
\input{xml/generated/OriginDestType}
\input{xml/generated/OriginDestType-schema}

\subsection*{Verknüpfung zum ÖPNV}
\input{xml/generated/StopLinkType}
\input{xml/generated/StopLinkType-schema}

\section{Basisgruppen}

\subsection*{Standort oder Gebiet}
\input{xml/generated/PlaceOrAreaGroup}
\input{xml/generated/PlaceOrAreaGroup-schema}

\subsection*{Dauer}
\input{xml/generated/DurationGroup}
\input{xml/generated/DurationGroup-schema}

\section{Dienst 1}
\label{subsec:Datenmodell:Dienst1}

\subsection*{Buchungsziel ID}
\input{xml/generated/BookingTargetIDType}
\input{xml/generated/BookingTargetIDType-schema}

\subsection*{Verleihstations ID}
\input{xml/generated/ProviderPlaceIDType}
\input{xml/generated/ProviderPlaceIDType-schema}

\subsection*{Fahrzeugattribut}
\input{xml/generated/AttributeType}
\input{xml/generated/AttributeType-schema}

\subsection*{Verleihstation}
\input{xml/generated/PlaceType}
\input{xml/generated/PlaceType-schema}

\subsection*{Gebiet mit Dichteangabe}
\input{xml/generated/DensityAreaType}
\input{xml/generated/DensityAreaType-schema}

\subsection*{Freefloating Gebiet}
\input{xml/generated/FloatingAreaType}
\input{xml/generated/FloatingAreaType-schema}

\subsection*{Gruppe von Verleihstationen}
\input{xml/generated/PlaceGroupType}
\input{xml/generated/PlaceGroupType-schema}

\subsection*{Provider}
\input{xml/generated/ProviderType}
\input{xml/generated/ProviderType-schema}

\subsection*{Buchungsziel}
\input{xml/generated/BookingTargetType}
\input{xml/generated/BookingTargetType-schema}


\section{Dienst 2}
\label{subsec:Datenmodell:Dienst2}

\subsection*{Buchungszieleigenschaften}
\input{xml/generated/BookingTargetPropertiesType}
\input{xml/generated/BookingTargetPropertiesType-schema}


\section{Dienst 4}
\label{subsec:Datenmodell:Dienst4}

\subsection*{Fahrzeugverfügbarkeit}
\input{xml/generated/BookingTargetChangeAvailabilityType}
\input{xml/generated/BookingTargetChangeAvailabilityType-schema}

\subsection*{Standortkapazität}
\input{xml/generated/PlaceAvailabilityType}
\input{xml/generated/PlaceAvailabilityType-schema}

\section{Dienst 5}
\label{subsec:Datenmodell:Dienst5}

\subsection*{Buchungsverfügbarkeit}
\input{xml/generated/BookingChangeType}
\input{xml/generated/BookingChangeType-schema}

\section{Dienst 6}
\label{subsec:Datenmodell:Dienst6}

\subsection*{Preisinformation}
\input{xml/generated/TariffType}
\input{xml/generated/TariffType-schema}

\section{Authentifizierung}
\label{sec:Datenmodell:Auth}

\subsection*{Benutzeridentifikation}
\input{xml/generated/UserInfoType}
\input{xml/generated/UserInfoType-schema}

\subsection*{Authentifizierung}
\input{xml/generated/AuthType}
\input{xml/generated/AuthType-schema}


\section{Fehlerbehandlung}
\subsection*{Fehler}
\input{xml/generated/ErrorType}
\input{xml/generated/ErrorType-schema}
