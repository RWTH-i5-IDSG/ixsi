\chapter{Datenmodell}
\label{sec:Datenmodell}
Dieser Abschnitt beschreibt das der Schnittstelle zugrunde liegende Datenmodell. \medskip

\noindent Symbollegende:\\
$\medcircle$ Auswahl (\verb|xs:choice|)\\
$\medsquare$ Optional (\verb|minOccurs=0|)\\
$\boxbox$ Mehrwertig (\verb|maxOccurs=0|)\\


\section{Basisdatentypen}
\label{subsec:Datenmodell:Basis}

\subsection*{Simple Basistypen}
\input{xml/generated/simpleTypes}
\input{xml/generated/simpleTypes-schema}

\subsection*{Text}
\input{xml/generated/TextType}
\input{xml/generated/TextType-schema}

\subsection*{Ortskoordinaten}
\input{xml/generated/CoordType}
\input{xml/generated/CoordType-schema}

\subsection*{Kreis}
\input{xml/generated/GeoCircleType}
\input{xml/generated/GeoCircleType-schema}

\subsection*{Rechteck}
\input{xml/generated/GeoRectangleType}
\input{xml/generated/GeoRectangleType-schema}

\subsection*{Gebiet}
\input{xml/generated/GeoAreaType}
\input{xml/generated/GeoAreaType-schema}


\subsection*{Gebiet (Einschluss / Ausschluss)}
\input{xml/generated/IncExcGeoAreaType}
\input{xml/generated/IncExcGeoAreaType-schema}

\subsection*{Zeitrahmen}
\input{xml/generated/TimePeriodType}
\input{xml/generated/TimePeriodType-schema}

\subsection*{Zeitrahmen (Vorschlag)}
\input{xml/generated/TimePeriodProposalType}
\input{xml/generated/TimePeriodProposalType-schema}

\subsection*{Herkunft/Ziel}
\input{xml/generated/OriginDestType}
\input{xml/generated/OriginDestType-schema}


\section{Basisgruppen}

\subsection*{Station oder Gebiet}
\input{xml/generated/PlaceOrAreaGroup}
\input{xml/generated/PlaceOrAreaGroup-schema}

\subsection*{Dauer}
\input{xml/generated/DurationGroup}
\input{xml/generated/DurationGroup-schema}

\section{Dienst 1}
\label{subsec:Datenmodell:Dienst1}

\subsection*{Buchungsziel ID}
\input{xml/generated/BookingTargetIDType}
\input{xml/generated/BookingTargetIDType-schema}

\subsection*{Verleihstations ID}
\input{xml/generated/ProviderPlaceIDType}
\input{xml/generated/ProviderPlaceIDType-schema}

\subsection*{Verknüpfung zum ÖPNV}
\input{xml/generated/StopLinkType}
\input{xml/generated/StopLinkType-schema}

\subsection*{Fahrzeugattribut}
\input{xml/generated/AttributeType}
\input{xml/generated/AttributeType-schema}

\subsection*{Verleihstation}
\input{xml/generated/PlaceType}
\input{xml/generated/PlaceType-schema}

\subsection*{Gebiet mit Dichteangabe}
\input{xml/generated/DensityAreaType}
\input{xml/generated/DensityAreaType-schema}

\subsection*{Freefloating Gebiet}
\input{xml/generated/FloatingAreaType}
\input{xml/generated/FloatingAreaType-schema}

\subsection*{Gruppe von Verleihstationen}
\input{xml/generated/PlaceGroupType}
\input{xml/generated/PlaceGroupType-schema}

\subsection*{Station oder Freefloatinggebiet}
\input{xml/generated/PlaceOrAreaGroup}
\input{xml/generated/PlaceOrAreaGroup-schema}

\subsection*{Buchungsziel}
\input{xml/generated/BookingTargetType}
\input{xml/generated/BookingTargetType-schema}

\subsection*{Provider}
\input{xml/generated/ProviderType}
\input{xml/generated/PlaceOrAreaGroup-schema}

\subsection*{Fahrzeugeigenschaft}
\input{xml/generated/AttributeType}
\input{xml/generated/PlaceOrAreaGroup-schema}

\subsection*{Freefloating Gebiet}
\input{xml/generated/FloatingAreaType}
\input{xml/generated/PlaceOrAreaGroup-schema}

\section{Dienst 2}
\label{subsec:Datenmodell:Dienst2}

\subsection*{Buchungszieleigenschaften}
\input{xml/generated/BookingTargetPropertiesType}
\input{xml/generated/BookingTargetPropertiesType-schema}


\section{Dienst 4}
\label{subsec:Datenmodell:Dienst4}

\subsection*{Fahrzeugverfügbarkeit}
\input{xml/generated/BookingTargetChangeAvailabilityType}
\input{xml/generated/BookingTargetChangeAvailabilityType-schema}

\subsection*{Stationskapazität}
\input{xml/generated/PlaceAvailabilityType}
\input{xml/generated/PlaceAvailabilityType-schema}

\section{Dienst 5}
\label{subsec:Datenmodell:Dienst5}

\subsection*{Buchungsverfügbarkeit}
\input{xml/generated/BookingChangeType}
\input{xml/generated/BookingChangeType-schema}

\section{Dienst 6}
\label{subsec:Datenmodell:Dienst6}

\subsection*{Preisinformation}
\input{xml/generated/TariffType}
\input{xml/generated/TariffType-schema}

\section{Authentifizierung}
\subsection*{Authentifizierung}
\input{xml/generated/AuthType}
\input{xml/generated/AuthType-schema}

\subsection*{Benutzeridentifikation}
\input{xml/generated/UserInfoType}
\input{xml/generated/UserInfoType-schema}

\section{Fehlerbehandlung}
\subsection*{Fehler}
\input{xml/generated/ErrorType}
\input{xml/generated/ErrorType-schema}

\subsection*{Fehlercodes}
Das FVS kann alternativ zu den normalen Response-, Update- und Handshake-Nachrichten auch im Fall eines Fehlers Fehlernachrichten an das RIS schicken. Fehlercodes werden im Datentyp ErrorCodeType gespeichert.
\begin{flushleft}
\rowcolors{1}{}{gray!10}
\begin{tabularx}{\linewidth}{l>{\raggedright\arraybackslash}X} 
\toprule
Wert & Bedeutung\\
\midrule
\verb|auth_provider_unknown| & Authentifizierung: Unbekannte Provider-ID\\
\verb|auth_invalid_password| & Authentifizierung: User-Passwort-Kombination ungültig\\
\verb|auth_invalid_token| & Authentifizierung: User-Token-Kombination ungültig\\
\verb|auth_session_invalid| & Authentifizierung: Session ist ungültig/abgelaufen\\
\verb|auth_anon_not_allowed| & Authentifizierung: Anonymer User nicht erlaubt\\
\verb|auth_not_authorized| & Autorisation: Nutzer ist zu dieser Anfrage nicht berechtigt\\
\verb|sys_backend_failed| & System: Hintergrundsystem antwortet nicht\\
\verb|sys_unknown_failure| & System: unbekannter Fehler\\
\verb|sys_not_implemented| & System: Request nicht implementiert\\
\verb|sys_request_not_plausible| & System: Request ist nicht plausibel. Dieser Wert sollte stets verwendet werden, wenn inhaltlich Fehler im Request enthalten sind.\\
\verb|booking_target_unknown| & Buchungsziel unbekannt\\
\verb|price_info_not_available| & Preisinformationen nicht verfügbar\\
\verb|booking_too_short| & Buchungsdauer zu kurz\\
\verb|booking_too_long| & Buchungsdauer zu lang\\
\verb|booking_target_not_available| & Buchungsziel im gegebenen Zeitraum nicht buchbar\\
\verb|booking_change_not_possible| & Buchungsänderung kann nicht durchgeführt werden\\
\verb|booking_id_unknown| & Unbekannte Buchungs-ID. Dieser Wert sollte auch verwendet werden, wenn die Buchungs-ID einem anderen User zugeordnet ist.\\
\verb|language_not_supported| & Angefragte Sprache nicht vollständig unterstützt, andere Sprache geliefert.\\
\bottomrule
\end{tabularx}
\end{flushleft}