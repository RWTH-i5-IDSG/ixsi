\chapter{Code Tables}
\label{cha:CodeTabellen}
The syntactical definition for IXSI does not contain concrete values (enumerations) for e.g., vehicle types or error codes. Instead, these values are determined within the following code tables. Only these determined values are allowed to be used by IXSI. 

\section{Vehicle Types}
\label{sec:CodeTabellen:ClassType}
For the selection \emph{ClassType}, the following values can be used:

\begin{flushleft}
\rowcolors{1}{}{gray!10}
\begin{tabularx}{\linewidth}{l>{\raggedright\arraybackslash}X}
\toprule
Value & Meaning\\
\midrule
\verb|bike| & Bike\\
\verb|motorcycle| & Motorcycle\\
\verb|micro| & Micro car (z.\,B. Smart4two)\\
\verb|mini| & Mini car (z.\,B. Opel Corsa)\\
\verb|small| & Small car (z.\,B. VW Golf)\\
\verb|medium| & Medium class car (z.\,B. Audi A4)\\
\verb|large| & Upper class car (z.\,B. BMW 7er)\\
\verb|van| & Van (z.\,B. VW T5 Multivan)\\
\verb|transporter| & Transporter (z.\,B. Ford Transit)\\
\bottomrule
\end{tabularx}
\end{flushleft}

\section{Engine Types}
\label{sec:CodeTabellen:EngineType}
For the selection \emph{EngineType}, the following values can be used:

\begin{flushleft}
\rowcolors{1}{}{gray!10}
\begin{tabularx}{\linewidth}{l>{\raggedright\arraybackslash}X}
\toprule
Value & Meaning\\
\midrule
\verb|none| & No engine (Muscle Power)\\
\verb|diesel| & Diesel engine\\
\verb|gasoline| & Otto engine\\
\verb|electric| & Electro engine\\
\verb|liquidgas| & Liquid gas (LPG)\\
\verb|naturalgas| & Natural gas (CNG)\\
\verb|hydrogen| & Hydrogen power\\
\verb|hybrid| & Hybrid power with electro- and internal combustion engine\\
\bottomrule
\end{tabularx}
\end{flushleft}

Note: A pedelec can be described as a combination of vehicle type \verb|bike| and engine type \verb|electric|.

\section{Vehicle Attributes}
\label{sec:CodeTabellen:AttributeClass}
Attributes and characteristics of a booking target can be classified to interpret them automatically. Therefore, attribute classes are used. Attribute classes are saved in IXSI in the data type \emph{AttributeClassType}. The following values are allowed:

\begin{flushleft}
\rowcolors{1}{}{gray!10}
\begin{tabularx}{\linewidth}{l>{\raggedright\arraybackslash}X}
\toprule
Value & Meaning\\
\midrule
\verb|trailer_hitch| & Trailer hitch\\
\verb|automatic,| & Automatic gear\\
\verb|convertible| & Cabriolet\\
\verb|air_condition| & Air condition\\
\verb|navigation| & Navigation system\\
\verb|cruise_control| & Cruise control\\
\verb|winter_tyres| & Winter and all season tires\\
\verb|child_seat_0| & Child seat\\
\verb|child_seat_1| & Child seat (9-18kg)\\
\verb|child_seat_4| & Child seat (15-36kg)\\
\verb|utility| & Kombi\\
\verb|doors_4| & 4/5 Doors\\
\verb|seats_9| & At least 9 seats\\
\verb|seats_7| & At least 7 seats\\
\verb|seats_5| & At least 5 seats\\
\verb|seats_4| & At least 4 seats\\
\bottomrule
\end{tabularx}
\end{flushleft}

\section{Consumption Data}
\label{sec:CodeTabellen:ConsumptionClassType}
Classes of consumption data for accounting of services, e.g., distance, duration or additional services like seats for children. Is saved in \emph{ConsumptionClassType}. The following values are allowed: 

\begin{flushleft}
\rowcolors{1}{}{gray!10}
\begin{tabularx}{\linewidth}{l>{\raggedright\arraybackslash}X}
\toprule
Value & Meaning\\
\midrule
\verb|distance| & Distance\\
\verb|duration| & Duration\\
\bottomrule
\end{tabularx}
\end{flushleft}

\section{User States}
\label{sec:CodeTabellen:UserStateType}

States, a user can have in the system. Is saved in \emph{UserStateType}. The following values are allowed:

\begin{flushleft}
\rowcolors{1}{}{gray!10}
\begin{tabularx}{\linewidth}{l>{\raggedright\arraybackslash}X}
\toprule
Values & Meaning\\
\midrule
\verb|operative| & User is activated and allowed to book vehicles.\\
\verb|nonoperative| & User is locked and not allowed to book vehicles (possible to be unlocked).\\
\verb|deleted| & User is (finally) deleted.\\
\bottomrule
\end{tabularx}
\end{flushleft}

\section{Error Codes}
\label{sec:CodeTabellen:ErrorCode}
Instead of usual response-, update- or handshake-messages, the VRS is able to transfer error- messages to the TIS in case an error occurs. Error codes are saved in the data type \emph{ErrorCodeType}.
\begin{flushleft}
\rowcolors{1}{}{gray!10}
\begin{tabularx}{\linewidth}{l>{\raggedright\arraybackslash}X}
\toprule
Values & Meaning\\
\midrule
\verb|auth_provider_unknown| & Authentication: Unknown provider-ID\\
\verb|auth_invalid_password| & Authentication: User-password-combination invalid\\
\verb|auth_invalid_token| & Authentication: User-token-combination invalid\\
\verb|auth_session_invalid| & Authentication: Session is invalid/ expired\\
\verb|auth_anon_not_allowed| & Authentication: Anonymous user not allowed\\
\verb|auth_not_authorized| & Autorization: User is not justified for this request\\
\verb|sys_backend_failed| & System: Background system does not respond\\
\verb|sys_unknown_failure| & System: Unknown error\\
\verb|sys_not_implemented| & System: Request not implemented\\
\verb|sys_request_not_plausible| & System: Request is not plausible. This value should always be used in case of content errors, included in the request.\\
\verb|booking_target_unknown| & Booking target unknown\\
\verb|price_info_not_available| & Price information not available\\
\verb|booking_too_short| & Booking duration too short\\
\verb|booking_too_long| & Booking duration too long\\
\verb|booking_target_not_available| & Booking target not bookable in the given time span\\
\verb|booking_change_not_possible| & Booking change can not be executed\\
\verb|booking_id_unknown| & Unknown booking-ID. This value should also be used, if the booking-ID is assigned to a different user.\\
\verb|booking_state_change_failed| & Changes of booking status failed.\\
\verb|booking_settings_not_understood| & Booking settings unknown.\\
\verb|booking_navigation_set_failed| & Setting of journey target failed.\\
\verb|language_not_supported| & Requested language not unconditionally supported, other language provided.\\
\bottomrule
\end{tabularx}
\end{flushleft}
