\chapter{Code Tabellen}
\label{cha:CodeTabellen}
Die syntaktische Definition für IXSI enthält keine konkreten Werte (Enumerations) für beispielsweise Fahrzeugtypen oder Fehlercodes. Stattdessen werden diese Werte in den nachfolgenden Codetabellen festgelegt. Nur die hier festgelegten Werte dürfen innerhalb von IXSI verwendet werden.

\section{Fahrzeugklassen}
\label{sec:CodeTabellen:ClassType}
Für die Auswahl \emph{ClassType} können folgende Werte verwendet werden:

\begin{flushleft}
\rowcolors{1}{}{gray!10}
\begin{tabularx}{\linewidth}{l>{\raggedright\arraybackslash}X} 
\toprule
Wert & Bedeutung\\
\midrule
\verb|bike| & Fahrrrad\\
\verb|motorcycle| & Motorrad\\
\verb|micro| & Kleinstwagen (z.B. Smart4two)\\
\verb|mini| & Kleinwagen (z.B. Opel Corsa)\\
\verb|small| & Kompaktwagen (z.B. VW Golf)\\
\verb|medium| & Mittelklassewagen (z.B. Audi A4)\\
\verb|large| & Oberklassewagen (z.B. BMW 7er)\\
\verb|van| & Kleinstwagen (z.B. VW T5)\\
\verb|transporter| & Kleinstwagen (z.B. Ford Transit)\\
\bottomrule
\end{tabularx}
\end{flushleft}

\section{Antriebsklassen}
\label{sec:CodeTabellen:EngineType}
Für die Auswahl \emph{EngineType} können folgende Werte verwendet werden:

\begin{flushleft}
\rowcolors{1}{}{gray!10}
\begin{tabularx}{\linewidth}{l>{\raggedright\arraybackslash}X} 
\toprule
Wert & Bedeutung\\
\midrule
\verb|none| & Kein Kraftantrieb (Muskelkraft)\\
\verb|diesel| & Dieselmotor\\
\verb|gasoline| & Ottomotor\\
\verb|electric| & Elektromotor\\
\verb|liquidgas| & Flüssiggas (LPG)\\
\verb|naturalgas| & Erdgas (CNG)\\
\verb|hydrogen| & Wasserstoffantrieb\\
\verb|hybrid| & Hybridantrieb mit Elektro- und Verbrennungsmotor\\
\bottomrule
\end{tabularx}
\end{flushleft}

Hinweis: Ein Pedelec kann als Kombination Fahrzeugklasse \verb|bike| und Antriebsklasse \verb|electric| dargestellt werden.

\section{Fahrzeugeigenschaften}
\label{sec:CodeTabellen:AttributeClass}
\TODO{Definition von Standortattributen, (Schlüsselkasten? etc)}
Attribute und Eigenschaften eines Buchungsziels oder eines Standorts können klassifiziert werden, um sie automatisch interpretieren zu können. Dazu werden Attributsklassen verwendet. Attributsklassen werden in IXSI im Datentyp \emph{AttributeClassType} gespeichert. Folgende Werte sind dabei erlaubt:

\begin{flushleft}
\rowcolors{1}{}{gray!10}
\begin{tabularx}{\linewidth}{l>{\raggedright\arraybackslash}X} 
\toprule
Wert & Bedeutung\\
\midrule
\verb|trailer_hitch| & Anhängerkupplung\\
\verb|automatic| & Automatikgetriebe\\
\verb|convertible| & Cabriolet\\
\verb|air_condition| & Klimaanlage\\
\verb|navigation| & Navigationssystem\\
\verb|cruise_control| & Tempomat\\
\verb|winter_tyres| & Winter- bzw. Ganzjahresreifen\\
\verb|child_seat_0| & Babyschale\\
\verb|child_seat_1| & Kindersitz (9-18kg)\\
\verb|child_seat_4| & Kindersitz (15-36kg)\\
\verb|utility| & Kombi\\
\verb|doors_4| & 4/5-Türer\\
\verb|seats_9| & Mindestens 9 Sitze\\
\verb|seats_7| & Mindestens 7 Sitze\\
\verb|seats_5| & Mindestens 5 Sitze\\
\verb|seats_4| & Mindestens 4 Sitze\\
\bottomrule
\end{tabularx}
\end{flushleft}

\section{Fehlercodes}
\label{sec:CodeTabellen:ErrorCode}
Das FVS kann alternativ zu den normalen Response-, Update- und Handshake-Nachrichten auch im Fall eines Fehlers Fehlernachrichten an das RIS schicken. Fehlercodes werden im Datentyp \emph{ErrorCodeType} gespeichert.
\begin{flushleft}
\rowcolors{1}{}{gray!10}
\begin{tabularx}{\linewidth}{l>{\raggedright\arraybackslash}X} 
\toprule
Wert & Bedeutung\\
\midrule
\verb|auth_provider_unknown| & Authentifizierung: Unbekannte Provider-ID\\
\verb|auth_invalid_password| & Authentifizierung: User-Passwort-Kombination ungültig\\
\verb|auth_invalid_token| & Authentifizierung: User-Token-Kombination ungültig\\
\verb|auth_session_invalid| & Authentifizierung: Session ist ungültig/abgelaufen\\
\verb|auth_anon_not_allowed| & Authentifizierung: Anonymer User nicht erlaubt\\
\verb|auth_not_authorized| & Autorisation: Nutzer ist zu dieser Anfrage nicht berechtigt\\
\verb|sys_backend_failed| & System: Hintergrundsystem antwortet nicht\\
\verb|sys_unknown_failure| & System: unbekannter Fehler\\
\verb|sys_not_implemented| & System: Request nicht implementiert\\
\verb|sys_request_not_plausible| & System: Request ist nicht plausibel. Dieser Wert sollte stets verwendet werden, wenn inhaltlich Fehler im Request enthalten sind.\\
\verb|booking_target_unknown| & Buchungsziel unbekannt\\
\verb|price_info_not_available| & Preisinformationen nicht verfügbar\\
\verb|booking_too_short| & Buchungsdauer zu kurz\\
\verb|booking_too_long| & Buchungsdauer zu lang\\
\verb|booking_target_not_available| & Buchungsziel im gegebenen Zeitraum nicht buchbar\\
\verb|booking_change_not_possible| & Buchungsänderung kann nicht durchgeführt werden\\
\verb|booking_id_unknown| & Unbekannte Buchungs-ID. Dieser Wert sollte auch verwendet werden, wenn die Buchungs-ID einem anderen User zugeordnet ist.\\
\verb|language_not_supported| & Angefragte Sprache nicht vollständig unterstützt, andere Sprache geliefert.\\
\bottomrule
\end{tabularx}
\end{flushleft}