\chapter{Hierarchical Model}
\label{cha:Hierachiemodell}
The hierarchical model describes different qualities of information coupling, based on service groups and serves as a recommendation concerning possible stages of implementation.
To realize a coupling between VRS and TIS, at least service 1 (static data) of both interacting parties has to be supported. In \cref{fig:depend}, dependencies between different services are depicted.

\begin{figure}[h]
\centering
\resizebox{1\columnwidth}{!} {
  \begin{tikzpicture}
  % Dienste
  \tikzstyle{ann} = [draw=none,fill=none,right]
  \node[rectangleTX, align=center](s1) {\textbf{Service 1} \\ Static Data};
  \node[rectangleTX, align=center, below left=10mm and -7mm of s1](s3) {\textbf{Service 3} \\ Availability \\ Subscription};
  \node[rectangleTX, align=center, left=5mm of s3](s2) {\textbf{Service 2} \\ Subscription \\ Handling};
  \node[rectangleTX, align=center, right=5mm of s3](s4) {\textbf{Service 4} \\ Booking \\ \vphantom{b}};
  \node[rectangleTX, align=center, right=5mm of s4](s6) {\textbf{Service 6} \\ Price \\ Information};
  \node[rectangleTX, align=center, below=10mm of s4](s5) {\textbf{Service 5} \\ Booking \\ Subscription};

  %   arrows
  \draw (s2) edge[->,thick=1cm] (s1);
  \draw (s3) edge[->,thick=1cm] (s1);
  \draw (s4) edge[->,thick=1cm] (s1);
  \draw (s5) edge[->,thick=1cm] (s4);
  \draw (s6) edge[->,thick=1cm] (s1);
  \end{tikzpicture}
}
\caption{IXSI Service Groups.}
\label{fig:depend}
\end{figure}


\section{Base Service A -- Session Handling}
\label{sec:Hierachiemodell:BasisdienstA}
Service A enables the authentication of end-customers towards the VRS.

\subsection*{Functions}
\begin{itemize}
\item Open / Close Session
\end{itemize}

\subsection*{Dependencies}
none

\section{Basisdienst B -- Abonnements (subscription handling)}
\label{sec:Hierachiemodell:BasisdienstB}
Dienst B enthält eine Funktion zur Überprüfung des Status einer Abonnementverbindung (heartbeat)

\subsection*{Funktionen}
\begin{itemize}
\item Heartbeat
\end{itemize}

\subsection*{Abhängigkeiten}
keine

\section{Basisdienst C -- tokens}
\label{sec:Hierachiemodell:BasisdienstC}
Dienst C enthält eine Funktion zur Erstellung von Authentifizierungstokens für Endkunden, die an Stelle von Klartextpasswörtern gespeichert / übertragen werden können.

\subsection*{Funktionen}
\begin{itemize}
\item Erstellung von Tokens
\end{itemize}

\subsection*{Abhängigkeiten}
keine


\section{Dienst 1 -- Statische Daten (static data)}
\label{sec:Hierachiemodell:Dienst1}
Dienst 1 dient dem Austausch von Informationen über Fahrzeugverleihanbietern und statischen Daten von Buchungszielen. Hierzu gehören Anbieter, Standort- und Fahrzeugdaten. 

Dienst 1 kann beispielsweise dafür verwendet werden, um nur Hinweise zu vorhandenen Standorten eines FVS-Betreibers in einem RIS anzuzeigen.
\subsection*{Funktionen}
\begin{itemize}
\item Abruf von Buchungsziel- und Betreiberinformationen
\end{itemize}

\subsection*{Abhängigkeiten}
keine

\section{Dienst 2 -- Verfügbarkeitsauskunft (availability query) }
\label{sec:Hierachiemodell:Dienst2}
Dienst 2 dient dem synchronen Abruf von Verfügbarkeitsinformationen.

Die tatsächlichen Verfügbarkeitszeiten von Buchungszielen werden während der Reiseauskunft durch das RIS beim FVS abgerufen.

\subsection*{Funktionen}
\begin{itemize}
\item Abruf von Verfügbarkeiten von Buchungszielen
\item Abruf von Standortkapazitäten (Dienst 2a)
\end{itemize}

\subsection*{Abhängigkeiten}
\begin{itemize}
\item Dienst 1
\end{itemize}

\section{Dienst 3 -- Verfügbarkeitsabonnement (availability subscription) }
\label{sec:Hierachiemodell:Dienst3}
Dienst 2 dient dem asynchronen Austausch von Verfügbarkeitsinformationen.

Um die Reiseauskunft zu beschleunigen, kann das RIS Verfügbarkeitszeiträume von Buchungszielen abonnieren um eine Abfrage während der Reiseauskunft zu vermeiden. Nach dem Abonnement einer Menge von Buchungszielen, informiert das FVS fortlaufend über Änderungen an Verfügbarkeitszeiträumen.

\subsection*{Funktionen}
\begin{itemize}
\item Abonnement von Verfügbarkeiten
\item Verfügbarkeitsinformation (push)
\item Abonnement von Standortkapazitäten (Dienst 3a)
\end{itemize}

\subsection*{Abhängigkeiten}
\begin{itemize}
\item Dienst 1
\item Basisdienst B
\end{itemize}


\section{Dienst 4 -- Buchung (booking)}
\label{sec:Hierachiemodell:Dienst4}
Dienst 4 dient der Buchung, Umbuchung und Stornierung von Fahrzeugen durch das RIS im Auftrag eines Kunden des FVS.

Für die Buchung eines Fahrzeuges ist es notwendig, dass der Kunde sich gegenüber dem FVS authentifiziert. Dazu werden die Authentifizierungsinformationen vom RIS an das FVS weitergeleitet. Das Geheimnis (Passwort, PIN, etc.) des Kunden darf vom RIS aus Sicherheitsgründen nicht gespeichert werden. Das RIS erhält bei erfolgreicher Authentifizierung eines Kunden für diesen ein Authentifizierungs-Token. Dieser kann entweder im RIS oder auf dem Endgerät des Kunden gespeichert werden. Unter Verwendung des Authentifizierungs-Tokens kann das RIS Anfragen zur Buchung und Stornierung von Buchungen durchführen.
Um dem Nutzer die Änderung einer Buchung zu ermöglichen, kann eine Buchung durch eine Umbuchungsanfrage des RIS durch eine neue Buchung ersetzt werden. Dabei muss sichergestellt werden, dass im Fall, dass keine neue Buchung möglich ist, die alte Buchung ihre Gültigkeit behält.

\subsection*{Funktionen}
\begin{itemize}
\item Authentifizierung von Nutzern gegenüber dem FVS.
\item Anfrage zur Buchung
\item Anfrage zur Buchungsänderung / Stornierung
\end{itemize}

\subsection*{Abhängigkeiten}
\begin{itemize}
\item Dienst 1
\end{itemize}

\section{Dienst 5 -- Buchungsabonnement (booking subscription)}
\label{sec:Hierachiemodell:Dienst5}
Dienst 5 dient dem Abonnement von Buchungsänderungen.

Das RIS kann durchgeführte Buchungen beim FVS abonnieren, um den Benutzer bei Änderungen (z.\,B. durch ein defektes Fahrzeug) informieren zu können.
\subsection*{Funktionen}
\begin{itemize}
\item Buchungsabonnement
\item Buchungsalarm (Push)
\end{itemize}

\subsection*{Abhängigkeiten}
\begin{itemize}
\item Dienst 4
\item Basisdienst B
\end{itemize}

\section{Dienst 6 -- Preisauskunft (price information)}
\label{sec:Hierachiemodell:Dienst6}
Dienst 6 dient der Preisauskunft von Verleihdienstleistungen.

Durch die Übermittlung von Start-, Zielort und Startzeitpunkt und Endzeitpunkt der Fahrt kann das RIS beim FVS eine unverbindliche Preisauskunft zur Information des Benutzers einholen. Das FVS antwortet mit einem Preis und ggfs. Einzelposten.

\subsection*{Funktionen}
\begin{itemize}
\item Anfrage eines Preises
\end{itemize}

\subsection*{Abhängigkeiten}
\begin{itemize}
\item Dienst 1
\end{itemize}


