\chapter{Hierachiemodell}
\label{sec:Dienstebenen}
Das Hierachiemodell beschreibt unterschiedliche Qualitäten der Informationskopplung, basierend auf Dienstgruppen und dient als Empfehlung für unterschiedliche Ausbaustufen der Implementierung.
Um eine Kopplung zwischen FVS und RIS zu realisieren, muss mindestens der Dienst 1 Statische Daten (static data) von beiden Interaktionspartnern unterstützt werden. In \cref{fig:depend} sind die Abhängigkeiten zwischen den Diensten dargestellt.

\begin{figure}[h]
  \centering
  \begin{tikzpicture}
  % Dienste
  \tikzstyle{ann} = [draw=none,fill=none,right]
  \node[rectangleTX](s1) {Dienst 1\\Statische Daten};
  \node[rectangleTX, below left of=s1](s3) {Dienst 3\\Verfügbarkeitsabonnement};
  \node[rectangleTX, left= 1cm of s3](s2) {Dienst 2\\Verfügbarkeitsauskunft};
  \node[rectangleTX, below right of=s1](s4) {Dienst 4\\Buchung};
  \node[rectangleTX, right= 1cm of s4](s6) {Dienst 6\\Preisauskunft};
  \node[rectangleTX, below= 1cm of s4](s5) {Dienst 5\\Buchungsabonnement};

  %   arrows
  \draw (s2) edge[->,thick=1cm] (s1);
  \draw (s3) edge[->,thick=1cm] (s1);
  \draw (s4) edge[->,thick=1cm] (s1);
  \draw (s5) edge[->,thick=1cm] (s4);
  \draw (s6) edge[->,thick=1cm] (s1);
  \end{tikzpicture}
  \label{fig:depend}
  \caption{Dienstabhängigkeiten}
\end{figure}

\section{Dienst 1 -- Statische Daten (static data)}
Dienst 1 dient dem Austausch von Informationen über Anbieter und statischen Daten von Buchungszielen. Hierzu gehören Anbieter, Standort- und Fahrzeugdaten.

Dienst 1 kann beispielsweise dafür verwendet werden, um nur Hinweise zu vorhandenen Stationen eines FVS-Betreibers in einem RIS anzuzeigen.
\subsection*{Funktionen}
\begin{itemize}
\item Abruf von Buchungsziel- und Betreiberinformationen
\end{itemize}

\subsection*{Abhängigkeiten}
keine 

\section{Dienst 2 -- Verfügbarkeitsauskunft (availability query) }
Dienst 2 dient dem synchronen Abruf von Verfügbarkeitsinformationen.

Die tatsächliche Verfügbarkeitszeiten von Buchungszielen wird während der Reiseauskunft durch das RIS beim FVS abgerufen. 

\subsection*{Funktionen}
\begin{itemize}
\item Abruf von Verfügbarkeiten von Buchungszielen
\item Abruf von Stationskapazitäten (Dienst 2a)
\end{itemize}

\subsection*{Abhängigkeiten}
\begin{itemize}
\item Dienst 1
\end{itemize}

\section{Dienst 3 -- Verfügbarkeitsabonnement (availability subscription) }
Dienst 2 dient dem asynchronen Austausch von Verfügbarkeitsinformationen.

Um die Reiseauskunft zu beschleunigen, kann das RIS Verfügbarkeitszeiträume von Buchungszielen abonnieren um eine Abfrage während der Reiseauskunft zu vermeiden. Nach dem Abonnement einer Menge von Buchungszielen, informiert das FVS fortlaufend über Änderungen an Verfügbarkeitszeiträumen.

\subsection*{Funktionen}
\begin{itemize}
\item Abonnement von Verfügbarkeiten
\item Verfügbarkeitsinformation (push)
\item Abonnement von Stationskapazitäten (Dienst 3a)
\end{itemize}

\subsection*{Abhängigkeiten}
\begin{itemize}
\item Dienst 1
\item Basisdienst B
\end{itemize}


\section{Dienst 4 -- Buchung (booking)}
Dienst 4 dient der Buchung, Umbuchungen und Stornierung von Fahrzeugen durch das RIS im Auftrag eines Kunden des FVS.

Für die Buchung eines Fahrzeuges ist es notwendig, dass der Kunde sich gegenüber dem FVS authentifiziert. Dazu werden die Authentifizierungsinformationen vom RIS an das FVS weitergeleitet. Das Geheimnis (Passwort, PIN, etc.) des Kunden darf vom RIS aus Sicherheitsgründen nicht gespeichert werden. Das RIS erhält bei erfolgreicher Authentifizierung eines Kunden für diesen ein Authentifizierungs-Token. Dieser kann entweder im RIS oder auf dem Endgerät des Kunden gespeichert werden. Unter Verwendung des Authentifizierungs-Tokens kann das RIS Anfragen zur Buchung und Stornierung von Buchungen durchführen. 
Um den Nutzer zu ermöglichen eine Buchung zu ändern, kann eine Buchung durch eine Umbuchungsanfrage des RIS durch eine neue Buchung ersetzt werden. Dabei muss sichergestellt werden, dass im Fall, dass keine neue Buchung möglich ist, die alte Buchung ihre Gültigkeit behält.

\subsection*{Funktionen}
\begin{itemize}
\item Authentifizierung von Nutzern gegenüber dem FVS.
\item Anfrage zur Buchung
\item Anfrage zur Buchungsänderung / Stornierung
\end{itemize}

\subsection*{Abhängigkeiten}
\begin{itemize}
\item Dienst 1
\end{itemize}

\section{Dienst 5 -- Buchungsabonnement (booking subscription)}
Dienst 5 dient dem Abonnement von Buchungsänderungen.

Das RIS kann durchgeführte Buchungen beim FVS abbonnieren, um den Benutzer bei Änderungen (z.B. durch ein defektes Fahrzeug) informieren zu können.
\subsection*{Funktionen}
\begin{itemize}
\item Buchungsabonnement
\item Buchungsalarm (Push)
\end{itemize}

\subsection*{Abhängigkeiten}
\begin{itemize}
\item Dienst 4
\item Basisdienst B
\end{itemize}

\section{Dienst 6 -- Preisauskunft (price information)}
Dienst 6 dient der Preisauskunft von Verleihdienstleistungen. 

Durch die Übermittlung von Start-, Zielort und Startzeitpunkt und Endzeitpunkt der Fahrt kann das RIS beim FVS eine unverbindliche Preisauskunft zur Information des Benutzers einhohlen. Das FVS antwortet mit einem Preis und ggfs. Einzelposten.

\subsection*{Funktionen}
\begin{itemize}
\item Anfrage eines Preises
\end{itemize}

\subsection*{Abhängigkeiten}
\begin{itemize}
\item Dienst 1
\end{itemize}


\section{Bassisdienst A -- Sitzungen (session handling)}
Dienst A dient der Authentifizierung von Endkunden gegenüber dem FVS.

\subsection*{Funktionen}
\begin{itemize}
\item Sitzung öffnen / schließen
\end{itemize}

\subsection*{Abhängigkeiten}
\begin{itemize}
\item -
\end{itemize}


\section{Bassisdienst B -- Abonnements (subscription handling)}
Dienst B enthält eine Funktion zur Überprüfung des Status einer Abonnementverbindung (heartbeat)

\subsection*{Funktionen}
\begin{itemize}
\item Heartbeat
\end{itemize}

\subsection*{Abhängigkeiten}
\begin{itemize}
\item -
\end{itemize}

\section{Bassisdienst C -- tokens}
Dienst C enthält eine Funktion zur Erstellung von Authentifizierungstokens für Endkunden, die anstelle von Klartextpasswörtern gespeichert / übertragen werden können.

\subsection*{Funktionen}
\begin{itemize}
\item Erstellung von Tokens
\end{itemize}

\subsection*{Abhängigkeiten}
\begin{itemize}
\item -
\end{itemize}

