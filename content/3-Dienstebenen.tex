\chapter{Hierarchiemodell}
\label{cha:Hierachiemodell}
Das Hierarchiemodell beschreibt unterschiedliche Qualitäten der Informationskopplung, basierend auf Dienstgruppen und dient als Empfehlung für unterschiedliche Ausbaustufen der Implementierung.
Um eine Kopplung zwischen FVS und RIS zu realisieren, muss mindestens der Dienst 1: Statische Daten (static data), von beiden Interaktionspartnern unterstützt werden. In \cref{fig:depend} sind die Abhängigkeiten zwischen den Diensten dargestellt.

\begin{figure}[h]
  \centering
  \resizebox {\columnwidth} {!} {
  \begin{tikzpicture}
  % Dienste
  \tikzstyle{ann} = [draw=none,fill=none,right]
  \node[rectangleTX](s1) {Dienst 1\\Statische Daten};
  \node[rectangleTX, below left of=s1](s3) {Dienst 3\\Verfügbarkeitsabonnement};
  \node[rectangleTX, left= 1cm of s3](s2) {Dienst 2\\Verfügbarkeitsauskunft};
  \node[rectangleTX, below right of=s1](s4) {Dienst 4\\Buchung};
  \node[rectangleTX, right= 1cm of s4](s6) {Dienst 6\\Preisauskunft};
  \node[rectangleTX, right= 1cm of s6](s9) {Dienst 9\\Benutzermanagement};


  \node[rectangleTX, below left = 1cm of s4](s8) {Dienst 8\\Buchungsstatus\\(Freischaltung)};
  \node[rectangleTX, left= 1cm of s8](s7) {Dienst 7\\Verbrauchsdaten};
  \node[rectangleTX, left= 1cm of s7](s5) {Dienst 5\\Buchungsabonnement};
  \node[rectangleTX, right= 1cm of s8](s10) {Dienst 10\\Fahrzeugeinstellungen};
  \node[rectangleTX, right= 1cm of s10](s11) {Dienst 11\\Navigation und Fahrtverlauf};
  %   arrows
  \draw (s2) edge[->,thick=1cm] (s1);
  \draw (s3) edge[->,thick=1cm] (s1);
  \draw (s4) edge[->,thick=1cm] (s1);
  \draw (s5) edge[->,thick=1cm] (s4);
  \draw (s6) edge[->,thick=1cm] (s1);
  \draw (s7) edge[->,thick=1cm] (s4);
  \draw (s8) edge[->,thick=1cm] (s4);
  \draw (s9) edge[->,thick=1cm] (s1);
  \draw (s10) edge[->,thick=1cm] (s4);
  \draw (s11) edge[->,thick=1cm] (s4);
  \end{tikzpicture}
  }
  \caption{Dienstabhängigkeiten \label{fig:depend}}
\end{figure}


\section{Basisdienst A -- Sitzungen (session handling)}
\label{sec:Hierachiemodell:BasisdientA}
Dienst A dient der Authentifizierung von Endkunden gegenüber dem FVS.

\subsection*{Funktionen}
\begin{itemize}
\item Sitzung öffnen / schließen
\end{itemize}

\subsection*{Abhängigkeiten}
keine

\section{Basisdienst B -- Abonnements (subscription handling)}
\label{sec:Hierachiemodell:BasisdientB}
Dienst B enthält eine Funktion zur Überprüfung des Status einer Abonnementverbindung (heartbeat)

\subsection*{Funktionen}
\begin{itemize}
\item Heartbeat
\end{itemize}

\subsection*{Abhängigkeiten}
keine

\section{Basisdienst C -- tokens}
\label{sec:Hierachiemodell:BasisdienstC}
Dienst C enthält eine Funktion zur Erstellung von Authentifizierungstokens für Endkunden, die an Stelle von Klartextpasswörtern gespeichert / übertragen werden können.

\subsection*{Funktionen}
\begin{itemize}
\item Erstellung von Tokens
\end{itemize}

\subsection*{Abhängigkeiten}
keine


\section{Dienst 1 -- Statische Daten (static data)}
\label{sec:Hierachiemodell:Dienst1}

Dienst 1 dient dem Austausch von Informationen über Fahrzeugverleihanbietern und statischen Daten von Buchungszielen. Hierzu gehören Anbieter, Standort- und Fahrzeugdaten. Dienst 1 kann beispielsweise dafür verwendet werden, um nur Hinweise zu vorhandenen Standorten eines FVS-Betreibers in einem RIS anzuzeigen.

\subsection*{Funktionen}
\begin{itemize}
\item Abruf von Buchungsziel- und Betreiberinformationen
\end{itemize}

\subsection*{Abhängigkeiten}
keine

\section{Dienst 2 -- Verfügbarkeitsauskunft (availability query) }
\label{sec:Hierachiemodell:Dienst2}
Dienst 2 dient dem synchronen Abruf von Verfügbarkeitsinformationen. Die tatsächlichen Verfügbarkeitszeiten von Buchungszielen werden während der Reiseauskunft durch das RIS beim FVS abgerufen.

\subsection*{Funktionen}
\begin{itemize}
\item Abruf von Verfügbarkeiten von Buchungszielen
\item Abruf von Standortkapazitäten (Dienst 2a)
\end{itemize}

\subsection*{Abhängigkeiten}
\begin{itemize}
\item Dienst 1
\end{itemize}

\section{Dienst 3 -- Verfügbarkeitsabonnement (availability subscription) }
\label{sec:Hierachiemodell:Dienst3}
Dienst 3 dient dem asynchronen Austausch von Verfügbarkeitsinformationen. Um die Reiseauskunft zu beschleunigen, kann das RIS Verfügbarkeitszeiträume von Buchungszielen abonnieren um eine Abfrage während der Reiseauskunft zu vermeiden. Nach dem Abonnement einer Menge von Buchungszielen, informiert das FVS fortlaufend über Änderungen an Verfügbarkeitszeiträumen.

\subsection*{Funktionen}
\begin{itemize}
\item Abonnement von Verfügbarkeiten
\item Verfügbarkeitsinformation (push)
\item Abonnement von Standortkapazitäten (Dienst 3a)
\end{itemize}

\subsection*{Abhängigkeiten}
\begin{itemize}
\item Dienst 1
\item Basisdienst B
\end{itemize}


\section{Dienst 4 -- Buchung (booking)}
\label{sec:Hierachiemodell:Dienst4}
Dienst 4 dient der Buchung, Umbuchung und Stornierung von Fahrzeugen durch das RIS im Auftrag eines Kunden des FVS.
Für die Buchung eines Fahrzeuges ist es notwendig, dass der Kunde sich gegenüber dem FVS authentifiziert. Dazu werden die Authentifizierungsinformationen vom RIS an das FVS weitergeleitet. Das Geheimnis (Passwort, PIN, etc.) des Kunden darf vom RIS aus Sicherheitsgründen nicht gespeichert werden. Das RIS erhält bei erfolgreicher Authentifizierung eines Kunden für diesen ein Authentifizierungs-Token. Dieser kann entweder im RIS oder auf dem Endgerät des Kunden gespeichert werden. Unter Verwendung des Authentifizierungs-Tokens kann das RIS Anfragen zur Buchung und Stornierung von Buchungen durchführen.
Um dem Nutzer die Änderung einer Buchung zu ermöglichen, kann eine Buchung durch eine Umbuchungsanfrage des RIS durch eine neue Buchung ersetzt werden. Dabei muss bei neuen Buchungen sichergestellt werden, dass eine alte Buchungen ihre Gültigkeit behält, wenn keine neue Buchung möglich ist.

\subsection*{Funktionen}
\begin{itemize}
\item Anfrage zur Buchung
\item Anfrage zur Buchungsänderung / Stornierung
\item Abonnement von externen Buchungen (Dienst 4a)
\end{itemize}

\subsection*{Abhängigkeiten}
\begin{itemize}
\item Dienst 1
\end{itemize}

\section{Dienst 5 -- Buchungsabonnement (booking subscription)}
\label{sec:Hierachiemodell:Dienst5}
Dienst 5 dient dem Abonnement von Buchungsänderungen. Das RIS kann durchgeführte Buchungen beim FVS abonnieren, um den Benutzer bei Änderungen (z.\,B. durch ein defektes Fahrzeug) informieren zu können.

\subsection*{Funktionen}
\begin{itemize}
\item Buchungsabonnement
\item Buchungsalarm (Push)
\end{itemize}

\subsection*{Abhängigkeiten}
\begin{itemize}
\item Dienst 4
\item Basisdienst B
\end{itemize}

\section{Dienst 6 -- Preisauskunft (price information)}
\label{sec:Hierachiemodell:Dienst6}

Dienst 6 dient der Preisauskunft von Verleihdienstleistungen. Durch die Übermittlung von Start-, Zielort und Startzeitpunkt und Endzeitpunkt der Fahrt kann das RIS beim FVS eine unverbindliche Preisauskunft zur Information des Benutzers einholen. Das FVS antwortet mit einem Preis und ggfs. Einzelposten.

\subsection*{Funktionen}
\begin{itemize}
\item Anfrage eines Preises
\end{itemize}

\subsection*{Abhängigkeiten}
\begin{itemize}
\item Dienst 1
\end{itemize}


\section{Dienst 7 -- Abonnement Verbrauchsdaten / Abrechnung (consumption data)}
Dienst 7 dient dem Austausch von Verbrauchsdaten wie z.\,B. Nutzungsdauer und Entfernung. Durch die Übermittlung der Verbrauchsdaten kann das RIS Rechnungen für Endkunden erstellen. Das RIS abonniert hierzu die Verbrauchsdaten einer durchgeführten Buchung beim FVS und erhält dann automatisch die zu dieser Buchung gehörigen Verbrauchsdaten.

\subsection*{Funktionen}
\begin{itemize}
\item Abonnement Verbrauchsdaten
\item Austausch Verbrauchsdaten (Push)
\end{itemize}

\subsection*{Abhängigkeiten}
\begin{itemize}
\item Dienst 4
%\ item Basisdienst A
\item Basisdienst B
\end{itemize}

\section{Dienst 8 -- Buchungsstatus ändern (freischalten / pausieren / abschließen) (booking state change)}
Dienst 8 dient dem Freischalten von Buchungen respektive Fahrzeugen (ggfs. Schlüsselkasten) durch das RIS. Dies erlaubt sowohl die Öffnung bzw. Freischaltung als auch die Rückgabe das Fahrzeug innerhalb der Endanwenderanwendung des RIS, z.\,B. innerhalb einer mobilen App. Weiterhin kann eine Buchung pausiert werden, dies schließt dass Fahrzeug physikalisch ab, beendet aber nicht die Buchung.
Der Status bezieht sich immer auf eine vorausgegangene Buchung und eine Authentifizierung ist erforderlich.

\subsection*{Funktionen}
\begin{itemize}
\item Buchung freischalten
\item Buchung pausieren
\item Buchung abschließen
\end{itemize}

\subsection*{Abhängigkeiten}
\begin{itemize}
\item Dienst 4
\item Basisdienst A
\end{itemize}

\section{Dienst 9 -- Anlegen und Sperren von Nutzern (user management)}
Dienst 9 dient dem Anlegen und Sperren von FVS Nutzern durch das RIS. Dies erlaubt, dass sich Nutzer nur beim RIS Betreiber registrieren müssen und diese Registrierung auch für das FVS gilt.

\subsection*{Funktionen}
\begin{itemize}
\item Benutzer anlegen
\item Benutzer sperren
\end{itemize}

\subsection*{Abhängigkeiten}
\begin{itemize}
\item Dienst 1
\item Basisdienst A
\end{itemize}


\section{Dienst 10 -- Fahrzeug- bzw. Buchungseinstellungsmanagement (booking settings)}
Dienst 10 dient dem Management von Fahrzeugeinstellungen. Das RIS kann Fahrzeugeinstellungen (Temperatur der Klimaanlage, eingestellter Radiosender etc) an das FVS übertragen und Einstellung für bestehende Buchungen abonnieren.

\subsection*{Funktionen}
\begin{itemize}
\item Fahrzeugeinstellungen setzten
\item Fahrzeugeinstellungen abonnieren
\end{itemize}

\subsection*{Abhängigkeiten}
\begin{itemize}
\item Dienst 4
\item Basisdienst A
\end{itemize}

\section{Dienst 11 -- Fernkonfiguration Navigationssystem und Fahrtverlauf überwachen}
Dienst 11 dient der Integration des Fahrzeugnavigationssystems in eine intermodale Reise. Das RIS kann nach der Buchung dem Fahrzeug den Zielort übermitteln, so dass der Reisende den Ort nicht selbst eingeben muss. Zusätzlich kann das RIS den Fortschritt der Fahrt überwachen um ggfs. auf Verzögerungen zu reagieren.

\subsection*{Funktionen}
\begin{itemize}
\item Navigationsziel des Fahrzeugs setzten
\item Fahrtverlauf abonnieren
\end{itemize}

\subsection*{Abhängigkeiten}
\begin{itemize}
\item Dienst 4
\item Basisdienst A
\end{itemize}

\section{Dienst 12 -- Dialog Intermodale Alternative}
Dienst 12 dient der Integration des Fahrzeugnavigationssystems in eine intermodale Reise. Falls sich durch Verzögerungen alternative Routen ergeben, kann das RIS dem Fahrzeug diese übermitteln. Der Reisende kann eine dieser Alternativen wählen und entscheiden, ob reservierungs- oder kostenpflichtige Teile der Route direkt vom RIS gebucht werden sollen.

\subsection*{Funktionen}
\begin{itemize}
\item Reisenden über Alternative Routen informieren
\item Wahl der Alternative und Buchungswunsch abonnieren
\end{itemize}

\subsection*{Abhängigkeiten}
\begin{itemize}
\item Dienst 4
\item Dienst 11
\item Basisdienst A
\end{itemize}
