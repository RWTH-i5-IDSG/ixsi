\chapter{Hierarchical Model}
\label{cha:Hierachiemodell}
The hierarchical model describes different qualities of information coupling, based on service groups and serves as a recommendation concerning possible stages of implementation.
To realize a coupling between VRS and TIS, at least service 1 (static data) of both interacting parties has to be supported. In \cref{fig:depend}, dependencies between different services are depicted.

\begin{figure}[h]
\centering
\resizebox{1\columnwidth}{!} {
  \begin{tikzpicture}
  % Dienste
  \tikzstyle{ann} = [draw=none,fill=none,right]
  \node[rectangleTX, align=center](s1) {\textbf{Service 1} \\ Static Data};
  \node[rectangleTX, align=center, below left=10mm and -7mm of s1](s3) {\textbf{Service 3} \\ Availability \\ Subscription};
  \node[rectangleTX, align=center, left=5mm of s3](s2) {\textbf{Service 2} \\ Subscription \\ Handling};
  \node[rectangleTX, align=center, right=5mm of s3](s4) {\textbf{Service 4} \\ Booking \\ \vphantom{b}};
  \node[rectangleTX, align=center, right=5mm of s4](s6) {\textbf{Service 6} \\ Price \\ Information};
  \node[rectangleTX, align=center, below=10mm of s4](s5) {\textbf{Service 5} \\ Booking \\ Subscription};

  %   arrows
  \draw (s2) edge[->,thick=1cm] (s1);
  \draw (s3) edge[->,thick=1cm] (s1);
  \draw (s4) edge[->,thick=1cm] (s1);
  \draw (s5) edge[->,thick=1cm] (s4);
  \draw (s6) edge[->,thick=1cm] (s1);
  \end{tikzpicture}
}
\caption{IXSI Service Groups.}
\label{fig:depend}
\end{figure}


\section{Base Service A -- Session Handling}
\label{sec:Hierachiemodell:BasisdienstA}
Service A enables the authentication of end-customers towards the VRS.

\subsection*{Functions}
\begin{itemize}
\item Open / Close Session
\end{itemize}

\subsection*{Dependencies}
none

\section{Base Service B -- Subscription Handling)}
\label{sec:Hierachiemodell:BasisdienstB}
Service B contains a function to check the status of a subscription connection (heartbeat)

\subsection*{Functions}
\begin{itemize}
\item Heartbeat
\end{itemize}

\subsection*{Dependencies}
none

\section{Base Service C -- tokens}
\label{sec:Hierachiemodell:BasisdienstC}
Service C contains a function to create authentication tokens for users, which can be saved/ transferred instead of plaintext passwords.

\subsection*{Functions}
\begin{itemize}
\item Creation of tokens
\end{itemize}

\subsection*{Dependencies}
none


\section{Service 1 -- Static Data}
\label{sec:Hierachiemodell:Dienst1}
Service 1 serves the information exchange across vehicle rental companies and static data of booking targets. These include provider-, position-, and vehicle-data. 

For example, service 1 can be used, to display messages solely for existing locations of an VRS-provider in a TIS. 
\subsection*{Functions}
\begin{itemize}
\item Call of booking targets and provider information.
\end{itemize}

\subsection*{Dependencies}
none

\section{Service 2 -- Availability Query}
\label{sec:Hierachiemodell:Dienst2}
Service 2 serves for asynchronous calls of availability information.

The actual availability times of booking targets are called during the travel inquiry by the TIS at the VRS.
\subsection*{Functions}
\begin{itemize}
\item Calls of availabilities of booking targets
\item Calls of location capacities (Service 2a)
\end{itemize}

\subsection*{Dependencies}
\begin{itemize}
\item Service 1
\end{itemize}

\section{Service 3 -- Availability Subscription}
\label{sec:Hierachiemodell:Dienst3}
Services 2 serves the asynchronous exchange of availability information.

To accelerate the travel inquiry, the TIS can subscribe availability timescales of booking targets to avoid a query during the travel inquiry. After subscribing for an amount of booking targets, the VRS informs continuously about changes in availability timescales. 

\subsection*{Functions}
\begin{itemize}
\item Availability subscription
\item Availability information (push)
\item Subscription of location capacities (Service 3a)
\end{itemize}

\subsection*{Dependencies}
\begin{itemize}
\item Service 1
\item Base Service B
\end{itemize}


\section{Dienst 4 -- Buchung (booking)}
\label{sec:Hierachiemodell:Dienst4}
Dienst 4 dient der Buchung, Umbuchung und Stornierung von Fahrzeugen durch das RIS im Auftrag eines Kunden des FVS.

Für die Buchung eines Fahrzeuges ist es notwendig, dass der Kunde sich gegenüber dem FVS authentifiziert. Dazu werden die Authentifizierungsinformationen vom RIS an das FVS weitergeleitet. Das Geheimnis (Passwort, PIN, etc.) des Kunden darf vom RIS aus Sicherheitsgründen nicht gespeichert werden. Das RIS erhält bei erfolgreicher Authentifizierung eines Kunden für diesen ein Authentifizierungs-Token. Dieser kann entweder im RIS oder auf dem Endgerät des Kunden gespeichert werden. Unter Verwendung des Authentifizierungs-Tokens kann das RIS Anfragen zur Buchung und Stornierung von Buchungen durchführen.
Um dem Nutzer die Änderung einer Buchung zu ermöglichen, kann eine Buchung durch eine Umbuchungsanfrage des RIS durch eine neue Buchung ersetzt werden. Dabei muss sichergestellt werden, dass im Fall, dass keine neue Buchung möglich ist, die alte Buchung ihre Gültigkeit behält.

\subsection*{Funktionen}
\begin{itemize}
\item Authentifizierung von Nutzern gegenüber dem FVS.
\item Anfrage zur Buchung
\item Anfrage zur Buchungsänderung / Stornierung
\end{itemize}

\subsection*{Abhängigkeiten}
\begin{itemize}
\item Dienst 1
\end{itemize}

\section{Dienst 5 -- Buchungsabonnement (booking subscription)}
\label{sec:Hierachiemodell:Dienst5}
Dienst 5 dient dem Abonnement von Buchungsänderungen.

Das RIS kann durchgeführte Buchungen beim FVS abonnieren, um den Benutzer bei Änderungen (z.\,B. durch ein defektes Fahrzeug) informieren zu können.
\subsection*{Funktionen}
\begin{itemize}
\item Buchungsabonnement
\item Buchungsalarm (Push)
\end{itemize}

\subsection*{Abhängigkeiten}
\begin{itemize}
\item Dienst 4
\item Basisdienst B
\end{itemize}

\section{Dienst 6 -- Preisauskunft (price information)}
\label{sec:Hierachiemodell:Dienst6}
Dienst 6 dient der Preisauskunft von Verleihdienstleistungen.

Durch die Übermittlung von Start-, Zielort und Startzeitpunkt und Endzeitpunkt der Fahrt kann das RIS beim FVS eine unverbindliche Preisauskunft zur Information des Benutzers einholen. Das FVS antwortet mit einem Preis und ggfs. Einzelposten.

\subsection*{Funktionen}
\begin{itemize}
\item Anfrage eines Preises
\end{itemize}

\subsection*{Abhängigkeiten}
\begin{itemize}
\item Dienst 1
\end{itemize}


