\chapter{Interaktionsprotokolle}
\label{sec:Interaktionsprotokolle}

\section{Dienst 1 -- Statische Daten (static data)}


\subsection{Abfrage Buchungsziele}

\begin{center}
\begin{sequencediagram}
\newthread{ris}{:RIS}
\newinst[7]{fvs}{:FVS}

\begin{sdblock}{BookingTargetsInfo}{}

\begin{call}{ris}{BookingTargetsInfoRequest(ProviderID*)}{fvs}{BookingTargetsInfos}

\end{call}

\end{sdblock}

\end{sequencediagram}\\
\hfill\emph{* optional}
\end{center}


\smallskip
Als Basis für die Informationskopplung dient der Austausch von sogenannten Buchungszielen. Buchungsziele sind eine logische Repräsentation von einem oder mehreren Fahrzeugen mit gemeinsamen Eingeschaften, wie z.B. vom gleichen Anbieter bereitgestellt, gleicher Fahrzeugtyp, gleiche Verleihstation. Diese Eigenschaften sind statisch. Um nur Informationen zu Buchungszielen eines bestimmten Anbieters zu erhalten kann nach \texttt{ProviderID} gefiltert werden.

\subsection{Abfrage Änderungen Buchungszielen}

\begin{center}
\begin{sequencediagram}
\newthread{ris}{:RIS}
\newinst[7]{fvs}{:FVS}

\begin{sdblock}{ChangedProviders}{}

\begin{call}{ris}{ChangedProvidersRequest}{fvs}{ChangedProvidersIDs}
\end{call}

% \begin{messcall}{ris}{Asynchrone Nachricht}{fvs}
% \end{messcall}

\end{sdblock}
\end{sequencediagram}
\end{center}

Um nichtintervallweise alle Buchungszielinformationen übertragen zu müssen kann mit dem Aufruf \texttt{ChangedProviders} angefragt werden bei welchem Anbieter sich Änderungen seit einem bestimmten Zeitpunkt, vorgegeben durch den Parameter \texttt{timestamp}, ergeben haben. Rückgegeben wird eine \texttt{ProviderID} die wiederum bei \texttt{BookingTargetsInfo} als Parameter übergeben werden kann.

\section{Dienst 2 -- Verfügbarkeitsauskunft (availability query) }

\subsection{Abfrage Verfügbarkeit}

\begin{center}
\begin{sequencediagram}
\newthread{ris}{:RIS}
\newinst[7]{fvs}{:FVS}

\begin{sdblock}{AvailabilityRequest}{}

\begin{call}{ris}{BookingTargetList/Circle/GeoRectangle, TimePeriod}{fvs}{BookingTargetList}
\end{call}

\end{sdblock}

\end{sequencediagram}
\end{center}

Um die konkreten Verfügbarkeiten abzufragen sendet das RIS eine Anfrage die entweder eine Liste mit Buchungszielen oder ein Geographisches Gebiet in Form einer Umgebungssuche (Circle) oder eines Rechtecks  (GeoRectangle) und eine geünschte Zeitperiode enthält. Ohne Angabe wird die Verfügbarkeit von allen Buchungszielen zurück gegeben. Als Antwort sendet das FVS eine Liste mit Buchungszielen und deren Verfügbarkeiten zurück.


\subsection{Abfrage aktuelle Stationskapazität (Dienst 2a)}

\begin{center}
\begin{sequencediagram}
\newthread{ris}{:RIS}
\newinst[7]{fvs}{:FVS}

\begin{sdblock}{PlaceAvailability}{}

\begin{call}{ris}{PlaceIDList/Circle/GeoRectangle}{fvs}{PlaceAvailabilityList}
\end{call}

\end{sdblock}

\end{sequencediagram}
\end{center}
Das RIS kann z.B. zur Kartendarstellung, die aktuellen Kapazitäten von Verleihstation anfragen. Hierzu wird eine Liste mit Stations IDs oder ein Gebiet übermittelt und eine Liste mit Stationen und deren aktuelle Anzahl verfügbarer Fahrzeuge zurück gegeben.


\section{Dienst 3 -- Verfügbarkeitsabonnement (availability subscription) }

\subsection{Abfrage Verfügbarkeit}

\begin{center}
\begin{sequencediagram}
\newthread{ris}{:RIS}
\newinst[7]{fvs}{:FVS}

\begin{sdblock}{AvailabilityRequest}{}

\begin{call}{ris}{BookingTargetList/Circle/GeoRectangle, TimePeriod}{fvs}{BookingTargetList}
\end{call}

\end{sdblock}

\end{sequencediagram}
\end{center}

Um die konkreten Verfügbarkeiten abzufragen sendet das RIS eine Anfrage die entweder eine Liste mit Buchungszielen oder ein Geographisches Gebiet in Form einer Umgebungssuche (Circle) oder eines Rechtecks  (GeoRectangle) und eine geünschte Zeitperiode enthält. Ohne Angabe wird die Verfügbarkeit von allen Buchungszielen zurück gegeben. Als Antwort sendet das FVS eine Liste mit Buchungszielen und deren Verfügbarkeiten zurück.

