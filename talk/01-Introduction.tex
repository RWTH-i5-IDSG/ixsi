\section{Introduction}
\subsection{Motivation}

\begin{frame}{Motivation}
Over the last years, public transport has become both more prominent and more diverse. Because of the complex structure of nowadays public transport networks, an electronic guidance is effectively required. \smallskip

\begin{itemize}
\item different transport modalities and service providers offer their own application
\item customization of GUI elements leads to appealing looks but usually also to cluttered presentation of information
\end{itemize}


% 
\end{frame}

\subsection{Cascading Information Theory}

Gamification is a term which was coined by Nick Pelling in the early 2000's. 
\begin{frame}{Cascading Information Theory}
\begin{block}{Definition Gamification}
\blockquote[Deterding2011]{The use of game design elemments characteristic for games in non-game contexts.}
\end{block}\pause

\begin{block}{Definition Cascading Information Theory}
Cascading Information Theory suggests to unveil information about the game in as small amounts as possible to ensure the user's focus exactly on the desired objective. 
Thereby, confusion and misdirection of players by providing excess information is prevented and each iteration of new data can be applied directly.\footnote{\url{http://techcrunch.com/2010/08/25/scvngr-game-mechanics/}}
\end{block}

\end{frame}





