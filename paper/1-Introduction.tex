%!TEX root =  ./Paper.tex
%!TEX encoding = UTF-8 Unicode


%----------------------------------------------------------
%---- Section: Motivation ---------------------------------
%----------------------------------------------------------

\uppercase{\section{Introduction}}
\label{sec:Introduction}
\noindent Lorem ipsum dolor sit amet, consetetur sadipscing elitr, sed diam nonumy eirmod tempor invidunt ut labore et dolore magna aliquyam erat, sed diam voluptua. At vero eos et accusam et justo duo dolores et ea rebum. Stet clita kasd gubergren, no sea takimata sanctus est Lorem ipsum dolor sit amet. Lorem ipsum dolor sit amet, consetetur sadipscing elitr, sed diam nonumy eirmod tempor invidunt ut labore et dolore magna aliquyam erat, sed diam voluptua. At vero eos et accusam et justo duo dolores et ea rebum. Stet clita kasd gubergren, no sea takimata sanctus est Lorem ipsum dolor sit amet.


% \subsection{Beispiele}
% \label{subsec:Motiviation}
% \noindent Lorem ipsum dolor sit amet, consetetur sadipscing elitr, sed diam nonumy eirmod tempor invidunt ut labore et dolore magna aliquyam erat, sed diam voluptua.


% \bigskip Strukturbeschreibung
% 
% \noindent The remainder of this paper is structured as follows: 
% In \autoref{sec:related-work}, existing applications and related research are discussed. 
% \autoref{sec:approach} describes our approach to the problem on an abstract level, whereas \autoref{sec:implementation} demonstrates the actual prototype implementation. 
% The evaluation is presented in \autoref{sec:evaluation} and \autoref{sec:conclusion} concludes the paper.

% 
% Abbildung aus zwei Teilen (ganze Seitenbreite)
% 
% \begin{figure*}[tb]
% \ffigbox[]{
%   \begin{subfloatrow}[2]
%   \ffigbox[\FBwidth]
%   {
%     \caption{Erste Bildunterschrift.}
%   }
%   {
%     \includegraphics[width=0.40\textwidth]{datei1.png}
%     \label{fig:label1}
%   }
%   \ffigbox[\FBwidth]
%   {
%     \caption{Zweite Bildunterschrift}
%   }
%   {
%     \includegraphics[width=0.40\textwidth]{datei2.png}
%     \label{fig:label2}
%   }
%   \end{subfloatrow}
% }
% {\caption{Abbildung aus zwei Teilen (ganze Seitenbreite).\label{fig:labelgesamt}}} 
% \end{figure*}

% Einfache Grafik im Text
% 
% \begin{figure}[bth]
% \centering
% {\includegraphics[width=0.6\columnwidth]{dateiname.jpg}}
% \caption{Bildunterschrift.\label{fig:label}}
% \end{figure}



% Einfache Tabelle
% 
% \begin{table}[htb]
% \caption{�berschrift.}\label{tab:label}
% \begin{small}
% \begin{tabular}{l|rrr} % Links ausgerichtet, Vertikale Linie, 3x rechts ausgerichtet
% \toprule % Rahmen oben
% �berschrift 1 & �berschrift 2 & �berschrift 3 & �berschrift 4\\
% \cmidrule(r){1-1} \cmidrule(lr){2-4} % Rahmen innen
% DB  Navigator & 1:13 & 2:44 & 4:22\\
% Prototype & 2:26 & 3:24 & 5:55\\
% \bottomrule % Rahmen unten
% \end{tabular}
% \end{small}
% \end{table}

% Zitat
% 
% Recently, there have been efforts by several researchers to reach a mutually agreed on definition. 
% One of the more elaborate and most cited versions originates with \cite{Deterding2011}. They define gamification as:
% \blockquote{The use of game design elements characteristic for games in non-game contexts.} 
% Elements of games are components or interaction patterns that, in combination, create the game experience. 
% By abstracting these elements from their game implementation and employing them in another context (e.g. business software etc.) the developer makes use of people's play instinct. 
% If a user is familiar with the ported game element there is a high probability that she will associate it with its original purpose and therefore be able to correctly utilize it. 


% Text mit Acronym und Fussnote
% The \ac{OTP} project\footnote{\url{http://opentripplanner.com}} is an open source platform which has developed a rather large community since its launch in 2009. 
% While still in development, it already features intermodal trip planning with special support for bikes, allows the import of several internationally popular data types and offers a RESTful web \ac{API}. 
% \ac{OTP} is mainly Java-based and consists of a server which performs routing and supplementary services, and a basic but effortlessly extensible web application enabling users to search for a route via their web browsers. 
% Furthermore, the RESTful API is able to provide the same experience using either \ac{XML} or \ac{JSON} formatting, enabling access to the service with mobile clients.

