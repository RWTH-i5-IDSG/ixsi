%----------------------------------------------------------
%---- Packages --------------------------------------------
%----------------------------------------------------------



%---- Have to be loaded at first---------------------------
\usepackage{commandsrwth}
\usepackage{etex}
\usepackage{ifthen}
\usepackage{ifpdf}
\usepackage{xspace}
\usepackage{ragged2e}
\usepackage{epsfig}
% \usepackage{subfigure}

\usepackage{calc}
\usepackage{pslatex}

%---- Coding, Language, Fonts, and Symbols ----------------
\usepackage{babel}
\usepackage[utf8]{inputenc}
\usepackage[T1]{fontenc}
\usepackage{textcomp}
\usepackage{eurosym}
\usepackage[expansion=false]{microtype}

%---- Mathematic ------------------------------------------
\usepackage{amsmath}
\usepackage{amssymb}
\usepackage{amsfonts}
\usepackage{amstext}
\usepackage{amsthm}
\usepackage{thmtools}
\usepackage[all]{onlyamsmath}
\usepackage{latexsym}
\usepackage{stmaryrd}
\usepackage{mathtools}
\usepackage{fixmath}
\usepackage{icomma}
\usepackage{cleveref}
%---- Scientific ------------------------------------------
\usepackage{listings}
\usepackage{bytefield}
\usepackage[
separate-uncertainty,
multi-part-units=repeat,
product-units=repeat,
per-mode=symbol-or-fraction,
]
{siunitx}

%---- Text and Layout -------------------------------------
\usepackage[
babel,
threshold=2,
strict
]{csquotes}
\usepackage{upquote}
\usepackage{varwidth}
\usepackage{multicol}


% \usepackage{enumitem}
% within paragraphs
\usepackage[svgnames]{xcolor}


%---- Floats - Tables, Figures ----------------------------

\usepackage{graphicx}
\usepackage{array}
\usepackage{tabularx}
\usepackage{booktabs}
\usepackage{multirow}
\usepackage{floatrow}
\floatsetup[widefigure]{}
\usepackage{longtable}
\usepackage{rotating}
\usepackage{dcolumn}
\usepackage{float}
\floatstyle{plaintop}
\restylefloat{table}
% flafter forces float to be placed after referenced
% \usepackage{flafter} 

%  fancy captions (use default for insticc)
% \usepackage[small,hang]{caption}

%table caption on top
\usepackage[small]{caption}

\usepackage[
caption=false,
]{subfig}

\graphicspath{{./graphics/}}


% permissive float placement
\IfNotClass{beamer}{%
  \setcounter{topnumber}{4}
  \setcounter{bottomnumber}{4}
  \setcounter{totalnumber}{10}
  \renewcommand{\textfraction}{0.15}
  \renewcommand{\topfraction}{0.85}
  \renewcommand{\bottomfraction}{0.70}
  \renewcommand{\floatpagefraction}{0.66}
}


%---- misc ----------------------------------------------

\IfNotClass{beamer}{
  \usepackage[noadjust]{cite}
  \usepackage{paralist}
  \usepackage[hyphens]{url}
  \usepackage{imakeidx}                    % Indexerstellung 
  \usepackage{ftnright} % Fussnoten nur rechts
}

\IfClass{beamer}{%
  \usepackage[accumulated]{beamerseminar}
  \usepackage[absolute,overlay]{textpos}
  \usepackage{pgfpages}                   % Ermöglicht eine Two-Screen Präsentation
  \usepackage{textpos}                    % Absolute Positionierung von Texten/Graphiken
}

% balance columns use for IEEE/ACM
% \usepackage{flushend} 
% \usepackage{balance} 


%---- hyperref --------------------------------------------
\IfNotClass{beamer}{%
  \usepackage[
  %   draft,
  %   final,
  breaklinks=true,
  linktocpage=true,
  hyperfootnotes=true,
%   pagebackref=true,
  plainpages=false,
  naturalnames=true,
  pageanchor=true,
  bookmarks=true,
%   hidelinks,
  colorlinks=false,
  linkcolor=blue,
  urlcolor=blue,
  citecolor=blue,
  bookmarks=true,
  bookmarksnumbered=true,
  pdfpagemode=UseOutlines,
  pdfpagelabels=true,
  pdfencoding=auto,
%  psdextra,
  pdfstartview={FitH},
  unicode
  ]{hyperref}

% not usefull for pdflatex
%\usepackage[hyphenbreaks]{breakurl}

%---- glossaries ------------------------------------------
\usepackage[
  acronym,
  shortcuts,
  style=super,
  description,
  nonumberlist
]{glossaries}
% \renewcommand*{\glspostdescription}{}


} %if Until here

%----------------------------------------------------------
%---- Settings --------------------------------------------
%----------------------------------------------------------


    
%--- hyperref meta data -----------------------------------
\hypersetup{%
%   pdfauthor={\author},
% %   pdfkeywords={\thiskeywords},
%   pdftitle={\title},
}





%---- listings --------------------------------------------
\lstset{%
  language=Java,
  tabsize=2,
  basewidth=0.55em,
  frame=lines,
  framerule=\heavyrulewidth,
  basicstyle=\scriptsize\ttfamily,
  keywordstyle=\bfseries\color{Maroon},
  stringstyle=\color{Maroon},
  commentstyle=\itshape\color{Green},
  showstringspaces=false,
  aboveskip=8pt,
  belowskip=8pt,
  numbers=left,
  numberstyle=\tiny,
  stepnumber=1,
  numbersep=0.5em,
  % xleftmargin=0.5em,
  % xrightmargin=0.5em,
  morecomment=[l]{//},
  escapeinside={(*}{*)},
  captionpos=b,
}

\lstdefinestyle{XML-style} {
  language=XML,
  morestring=[b]",
  %morestring=[s]{>}{<},
  morecomment=[s]{<?}{?>},
  morecomment=[s]{<!--}{-->},
  stringstyle=\color{black},
  tagstyle=\color{Purple},
  keywordstyle=\color{MediumBlue},
  commentstyle=\color{gray}\slshape,
  morekeywords={xmlns:xsi,xmlns:xsd,xmlns:soap,xmlns,version,type,name,minOccurs,maxOccurs,value,base,
    elementFormDefault,targetNamespace,sex}% list your attributes here
  }
  
  \lstdefinestyle{JSON-style} {
    stringstyle=\color{black},
  }
  
  \lstdefinestyle{XMLX-style} {
    backgroundcolor=\color{white},
    frame=leftline,
    language=XML,
    morestring=[b]",
    %morestring=[s]{>}{<},
    morecomment=[s]{<?}{?>},
    morecomment=[s]{<!--}{-->},
    stringstyle=\color{black},
    tagstyle=\color{Purple},
    keywordstyle=\color{MediumBlue},
    commentstyle=\color{gray}\slshape,
    morekeywords={xmlns:xsi,xmlns:xsd,xmlns:soap,xmlns,version,type,name,minOccurs,maxOccurs,value,base,
      elementFormDefault,targetNamespace,sex},% list your attributes here
      numbers=none
    }
    
    \lstdefinestyle{JSONX-style} {
      backgroundcolor=\color{white},
      frame=leftline,
      stringstyle=\color{black},
      numbers=none
    }


%----------------------------------------------------------
%---- Definitions -----------------------------------------
%----------------------------------------------------------

%---- General definitions ---------------------------------

% Margin notes
\providecommand{\marginline}[1]{\marginpar[\RaggedLeft{#1}]{\RaggedRight #1}}
\newcommand*{\TODO}[1]{\marginline{\scriptsize\textcolor{red}{TODO: #1}}}

% Style abbreviations
\newcommand*{\ebf}[1]{\emph{\textbf{#1}}}
\newcommand*{\bs}[1]{\boldsymbol{#1}}
\renewcommand*{\_}{\ensuremath{\mathunderscore}\xspace}

% Enumeration
\newcommand*{\st}{\ensuremath{1^{\mbox{\scriptsize st}}}\xspace}
\newcommand*{\nd}{\ensuremath{2^{\mbox{\scriptsize nd}}}\xspace}
\newcommand*{\rd}{\ensuremath{3^{\mbox{\scriptsize rd}}}\xspace}
\newcommand*{\nth}[1]{\ensuremath{#1^{\mbox{\scriptsize th}}}\xspace}

% Arrows
\renewcommand*{\implies}{\ensuremath{\rightarrow}\xspace}
\renewcommand*{\iff}{\ensuremath{\leftrightarrow}\xspace}
\newcommand*{\IF}{\ensuremath{\Rightarrow}\xspace}
\newcommand*{\sra}{\ensuremath{\shortrightarrow}\xspace}
\newcommand*{\sla}{\ensuremath{\shortleftarrow}\xspace}


\newcolumntype{M}[1]{
  >{\RaggedRight\arraybackslash\hspace{0pt}}m{#1}%
}
\newcolumntype{R}[1]{
  >{\begin{turn}{90}\begin{minipage}{#1}%
  \RaggedRight\arraybackslash\hspace{0pt}}l
  <{\end{minipage}\end{turn}}%
}
\newcommand*{\armultirow}[3]{%
  \multicolumn{#1}{#2}{%
    \begin{picture}(0,0)%
    \put(0,0){\begin{tabular}[t]{@{}#2@{}}#3\end{tabular}}%
    \end{picture}%
  }%
}
\makeatletter
\newcolumntype{K}[1]{%
  >{\DC@{.}{.}{#1}}l<{\DC@end}%
}
\makeatother

%---- hyperref --------------------------------------------
% Write Section instead of section for \ref
\addto\extrasenglish{%
  \renewcommand*{\sectionautorefname}{Section}%
  \renewcommand*{\subsectionautorefname}{\sectionautorefname}%
  \renewcommand*{\subsubsectionautorefname}{\sectionautorefname}%
}

\newcommand*{\definitionautorefname}{Definiton}
\newcommand*{\subfigureautorefname}{\figureautorefname}


%---- listings --------------------------------------------
% add \lineref command for listing enviroment
\newcommand*{\lineref}[2][]{%
\ifthenelse{\equal{#1}{}}{%
\hyperref[{#2}]{line~\ref*{#2}}%
}{%
\hyperref[{#2}]{\begin{NoHyper}line~\ref{#1}--\ref{#2}\end{NoHyper}}%
}\xspace%
}


%----- Beamer settings --------------------------------------
\IfClass{beamer}{
  \usetheme{rwth}
  \setbeamercovered{transparent}                                  % Versteckter Text ist transparent
  \setbeamertemplate{caption}{\insertcaption}             % Only text as caption
  \setbeamerfont{caption}{size={\footnotesize}}           % Caption size
}

% tweak footnote line
\renewcommand{\footnoterule}{%
  \kern -3pt
  \hrule width 1.5in
  \kern 2.6pt
}

% superscript footnote number
% \deffootnote{0em}{1.6em}{\textsuperscript{\thefootnotemark}}

\usepackage{scrpage2}


\setcounter{tocdepth}{2} % nur 2 Gliederungsebenen im Inhaltsverzeichnis

% ------ Fussnoten
\pagestyle{scrheadings} 

\refoot{\thispublishers ~- \textbf{\thistitle}}
\lofoot{Stand \thisdate ~- Version \thisversion}

\renewcommand*{\chapterpagestyle}{scrheadings}
% ------