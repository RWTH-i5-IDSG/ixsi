\documentclass[a4paper,
  DIV=15, % Faktor Nutzefläche vs. Rand (zwischen 6 und 15)
  BCOR=12mm, % Bindekorrekturfaktor für Klebebindung
  twoside,
  german,
  titlepage=false,
  footsepline=true,
  toc=bib,indent,
  fontsize=12pt % Standardschriftgröße, default = 11pt
]{scrbook}
\usepackage[utf8]{inputenc}


%  metadata in Settings.tex



% load settings
%!TEX root =  ./Paper.tex
%!TEX encoding = UTF-8 Unicode

\def\thisversion{0.1.3}
\def\thisdate{\today}
\def\thistime{\currenttime}

\def\thistitle{Schnittstelle zur Kopplung eines\\ Fahrzeugverleihsystems mit einem\\ Reiseinformationssystem}
\def\thistitleshort{Schnittstelle FVS - RIS}
\def\thisauthor{Sevket Gökay, Karl-Heinz Krempels,\\
Christian Samsel \& Christoph Terwelp}
\def\thispublishers{RWTH Aachen}

\def\thisabstract{
}
%    
%----------------------------------------------------------
%---- Packages --------------------------------------------
%----------------------------------------------------------



%---- Have to be loaded at first---------------------------
\usepackage{rwthcommands}
\usepackage{etex}
\usepackage{ifthen}
\usepackage{ifpdf}
\usepackage{xspace}
\usepackage{ragged2e}
\usepackage{epsfig}
% \usepackage{subfigure}
\usepackage{calc}
\usepackage{pslatex}
\usepackage{apalike}
\usepackage{scrextend}

%---- Coding, Language, Fonts, and Symbols ----------------
\usepackage{babel}
\usepackage[utf8]{inputenc}
\usepackage[T1]{fontenc}
\usepackage{textcomp}
\usepackage{eurosym}
\usepackage[expansion=false]{microtype}

%---- Mathematic ------------------------------------------
\usepackage{amsmath}
\usepackage{amssymb}
\usepackage{amsfonts}
\usepackage{amstext}
\usepackage{amsthm}
\usepackage{thmtools}
\usepackage[all]{onlyamsmath}
\usepackage{latexsym}
%\usepackage{stmaryrd}
\usepackage{mathtools}
\usepackage{fixmath}
\usepackage{icomma}
\usepackage{cleveref}
%---- Scientific ------------------------------------------
\usepackage{listings}
\usepackage{bytefield}
\usepackage[
separate-uncertainty,
multi-part-units=repeat,
product-units=repeat,
per-mode=symbol-or-fraction,
]
{siunitx}

%---- Text and Layout -------------------------------------
\usepackage[babel]{csquotes}
\usepackage{upquote}
\usepackage{varwidth}
\usepackage{multicol}


% \usepackage{enumitem}
% within paragraphs
\usepackage[svgnames]{xcolor}
\usepackage[noadjust]{cite}

%---- Floats - Tables, Figures ----------------------------
\usepackage{graphicx}
\usepackage{array}
\usepackage{tabularx}
\usepackage{booktabs}
\usepackage{multirow}
\usepackage{floatrow}
\floatsetup[widefigure]{}
\usepackage{longtable}
\usepackage{rotating}
\usepackage{dcolumn}
\usepackage{float}
%\usepackage{flafter}
\usepackage[small,hang]{caption}
%\usepackage{caption}
\usepackage[
caption=false,
font=footnotesize
]{subfig}

\IfNotClass{beamer}{
  \usepackage{paralist}
  \usepackage[hyphens]{url}
  \usepackage{makeidx}                    % Indexerstellung
  \usepackage{SciTePress}
  \usepackage{ftnright} % Fussnoten nur rechts
}

\IfClass{beamer}{%
  \usepackage[accumulated]{beamerseminar}
  \usepackage[overlay]{textpos}
  \usepackage{pgfpages}                   % Ermöglicht eine Two-Screen Präsentation
  \usepackage{textpos}                    % Absolute Positionierung von Texten/Graphiken
}

\IfNotClass{beamer}{%
  
}
% \usepackage{flushend} % balance columns use for IEEE

\IfNotClass{beamer}{

}

%---- hyperref --------------------------------------------
\IfNotClass{beamer}{%
  \usepackage[
  %   draft,
  %   final,
  breaklinks=true,
  linktocpage=true,
  hyperfootnotes=true,
  pagebackref=false,
  plainpages=false,
  %   naturalnames=true,
  %   pageanchor=true,
  bookmarks=true,
  bookmarksnumbered=true,
  hidelinks,
  colorlinks=false,
  linkcolor=blue,
  urlcolor=blue,
  citecolor=blue,
  bookmarks=true,
  bookmarksnumbered=true,
  pdfpagemode=UseOutlines,
  pdfpagelabels=true,
  pdfencoding=auto,
  psdextra,
  unicode
  ]{hyperref}
}


%---- Lists -------------------------------------------------
\usepackage[
  acronym,
  shortcuts,
  style=super,
  description,
  nowarn,
  nonumberlist
]{glossaries}

%----------------------------------------------------------
%---- Settings --------------------------------------------
%----------------------------------------------------------

%---- title pages -----------------------------------------

\IfClass{beamer}{
  \title[\thistitle]{\thisconference \\
    \bigskip \thistitle}
    \subtitle{\thisauthor}
    \author{\thispresenter}
    \institute{\thisaffiliation}
    \date{\presentationdate}
    \titlegraphic{\includegraphics[scale=0.3]{pics/logo.png}}
}

\IfNotClass{beamer}{ % SciPress
  \title{\thistitle \subtitle{\thissubtitle}}
  
  \ifthenelse{\equal{\anonymize}{true}}{
    \author{
      \authorname{Undisclosed Authors}
      \affiliation{Undisclosed Affiliations}
      \email{}
    }
  }{
    \author{
      \authorname{\thisauthor}
      \affiliation{\thisaffiliation}
      \email{\thisemail}
    }}
    \keywords{\thiskeywords}
    
    \abstract{\thisabstract}
}
%--- hyperref meta data -----------------------------------

\ifthenelse{\equal{\anonymize}{true}}{
  \hypersetup{%
    pdfauthor={Undisclosed},
    pdfkeywords={\thiskeywords},
    pdftitle={\thistitle ~- \thissubtitle},
    pdfsubject={\thisconference}
  }
}{
  \hypersetup{%
    pdfauthor={\thisauthor},
    pdfkeywords={\thiskeywords},
    pdftitle={\thistitle ~- \thissubtitle},
    pdfsubject={\thisconference}
  }
}

%---- Floats ----------------------------------------------
\graphicspath{{./graphics/}}


\IfNotClass{beamer}{%
  \renewcommand{\dbltopfraction}{0.7}
  \renewcommand*{\topfraction}{0.85}
  \renewcommand*{\bottomfraction}{.5}
  
  \setcounter{topnumber}{1}
  \setcounter{dbltopnumber}{1}  
  \setcounter{bottomnumber}{1}
  \setcounter{totalnumber}{2}
}


%---- glossaries ------------------------------------------
\renewcommand*{\glspostdescription}{}

%---- listings --------------------------------------------
\lstset{%
  language=Java,
  tabsize=2,
  basewidth=0.55em,
  frame=lines,
  framerule=\heavyrulewidth,
  basicstyle=\scriptsize\ttfamily,
  keywordstyle=\bfseries\color{Maroon},
  stringstyle=\color{Maroon},
  commentstyle=\itshape\color{Green},
  showstringspaces=false,
  aboveskip=8pt,
  belowskip=8pt,
  numbers=left,
  numberstyle=\tiny,
  stepnumber=1,
  numbersep=0.5em,
  % xleftmargin=0.5em,
  % xrightmargin=0.5em,
  morecomment=[l]{//},
  escapeinside={(*}{*)},
  captionpos=b,
}

\lstdefinestyle{XML-style} {
  language=XML,
  morestring=[b]",
  %morestring=[s]{>}{<},
  morecomment=[s]{<?}{?>},
  morecomment=[s]{<!--}{-->},
  stringstyle=\color{black},
  tagstyle=\color{Purple},
  keywordstyle=\color{MediumBlue},
  commentstyle=\color{gray}\slshape,
  morekeywords={xmlns:xsi,xmlns:xsd,xmlns:soap,xmlns,version,type,name,minOccurs,maxOccurs,value,base,
    elementFormDefault,targetNamespace,sex}% list your attributes here
  }
  
  \lstdefinestyle{JSON-style} {
    stringstyle=\color{black},
  }
  
  \lstdefinestyle{XMLX-style} {
    backgroundcolor=\color{white},
    frame=leftline,
    language=XML,
    morestring=[b]",
    %morestring=[s]{>}{<},
    morecomment=[s]{<?}{?>},
    morecomment=[s]{<!--}{-->},
    stringstyle=\color{black},
    tagstyle=\color{Purple},
    keywordstyle=\color{MediumBlue},
    commentstyle=\color{gray}\slshape,
    morekeywords={xmlns:xsi,xmlns:xsd,xmlns:soap,xmlns,version,type,name,minOccurs,maxOccurs,value,base,
      elementFormDefault,targetNamespace,sex},% list your attributes here
      numbers=none
    }
    
    \lstdefinestyle{JSONX-style} {
      backgroundcolor=\color{white},
      frame=leftline,
      stringstyle=\color{black},
      numbers=none
    }


%----------------------------------------------------------
%---- Definitions -----------------------------------------
%----------------------------------------------------------

%---- General definitions ---------------------------------

% Margin notes
\providecommand{\marginline}[1]{\marginpar[\RaggedLeft{#1}]{\RaggedRight #1}}
\newcommand*{\TODO}[1]{\marginline{\scriptsize\textcolor{red}{TODO: #1}}}

% Style abbreviations
\newcommand*{\ebf}[1]{\emph{\textbf{#1}}}
\newcommand*{\bs}[1]{\boldsymbol{#1}}
\renewcommand*{\_}{\ensuremath{\mathunderscore}\xspace}

% Enumeration
\newcommand*{\st}{\ensuremath{1^{\mbox{\scriptsize st}}}\xspace}
\newcommand*{\nd}{\ensuremath{2^{\mbox{\scriptsize nd}}}\xspace}
\newcommand*{\rd}{\ensuremath{3^{\mbox{\scriptsize rd}}}\xspace}
\newcommand*{\nth}[1]{\ensuremath{#1^{\mbox{\scriptsize th}}}\xspace}

% Arrows
\renewcommand*{\implies}{\ensuremath{\rightarrow}\xspace}
\renewcommand*{\iff}{\ensuremath{\leftrightarrow}\xspace}
\newcommand*{\IF}{\ensuremath{\Rightarrow}\xspace}
\newcommand*{\sra}{\ensuremath{\shortrightarrow}\xspace}
\newcommand*{\sla}{\ensuremath{\shortleftarrow}\xspace}


\newcolumntype{M}[1]{
  >{\RaggedRight\arraybackslash\hspace{0pt}}m{#1}%
}
\newcolumntype{R}[1]{
  >{\begin{turn}{90}\begin{minipage}{#1}%
  \RaggedRight\arraybackslash\hspace{0pt}}l
  <{\end{minipage}\end{turn}}%
}
\newcommand*{\armultirow}[3]{%
  \multicolumn{#1}{#2}{%
    \begin{picture}(0,0)%
    \put(0,0){\begin{tabular}[t]{@{}#2@{}}#3\end{tabular}}%
    \end{picture}%
  }%
}
\makeatletter
\newcolumntype{K}[1]{%
  >{\DC@{.}{.}{#1}}l<{\DC@end}%
}
\makeatother

%---- hyperref --------------------------------------------
\addto\extrasenglish{%
  \renewcommand*{\sectionautorefname}{Section}%
  \renewcommand*{\subsectionautorefname}{\sectionautorefname}%
  \renewcommand*{\subsubsectionautorefname}{\sectionautorefname}%
}

% \newcommand*{\definitionautorefname}{Definiton}
% \newcommand*{\subfigureautorefname}{\figureautorefname}
% \makeatletter
% \let\@autoref=\autoref
% \newcommand*{\autoref}[2][]{%
%       \ifthenelse{\equal{#1}{}}{%
%               \hyperref[{#2}]{\@autoref{#2}}%
%       }{%
%               \hyperref[{#2}]{\begin{NoHyper}\@autoref{#2}~\subref{#1}\end{NoHyper}}%
%       }\xspace%
% }
% \makeatother

%---- listings --------------------------------------------
\newcommand*{\lineref}[2][]{%
\ifthenelse{\equal{#1}{}}{%
\hyperref[{#2}]{line~\ref*{#2}}%
}{%
\hyperref[{#2}]{\begin{NoHyper}line~\ref{#1}--\ref{#2}\end{NoHyper}}%
}\xspace%
}


%----- Beamer settings --------------------------------------
\IfClass{beamer}{
  \setbeamercovered{transparent}                                  % Versteckter Text ist transparent
  \setbeamertemplate{caption}{\insertcaption}             % Only text as caption
  \setbeamerfont{caption}{size={\footnotesize}}           % Caption size
  
  \definecolor{background}{RGB}{0,86,139}
  \definecolor{background2}{RGB}{215,224,235}
  
  \usecolortheme{rose}
  \usecolortheme{sidebartab}
  \useoutertheme[hideallsubsections,width=55pt,height=30pt]{sidebar}
  
  \setbeamertemplate{navigation symbols}{\insertframenumber /\inserttotalframenumber}
  
  \setbeamersize{text margin left=1em,text margin right=1em}
  
  \renewcommand{\arraystretch}{1.3}
  
  \defbeamertemplate*{title page}{customized}[1][] % lean title page
  {
    \usebeamerfont{title} \inserttitle\par
    \usebeamerfont{subtitle}\insertsubtitle\par
    \bigskip \bigskip
    \usebeamerfont{author}\insertauthor\par
    \usebeamerfont{institute}\insertinstitute\par
    \medskip
    \usebeamerfont{date}\insertdate\par
    \vfill
    \hfill \usebeamercolor[fg]{titlegraphic}\inserttitlegraphic
  }
}


\renewcommand{\footnoterule}{%
  \kern -3pt
  \hrule width 1.5in
  \kern 2.6pt
}

\deffootnote{0em}{1.6em}{\textsuperscript{\thefootnotemark}}





\loadglsentries[\acronymtype]{preamble/Acronyms}

\makeglossaries                                 % Make the glossary
\makeindex                                      % Make the index

\begin{document}

\begin{titlepage}

\includegraphics[width=0.30\textwidth]{logo-econnect.jpg}
\hfill \includegraphics[width=0.30\textwidth]{logo-rwth.jpg}

\vspace{4em}

{\huge\bfseries \thistitle} \\

\vspace{4em}

{\Large\bfseries Autoren: \thisauthor}

\vspace{4em}

Datum: \thisdate ~ Version: \thisversion

\vspace{4em}

%\begin{abstract}
\thisabstract
%\end{abstract}

\vspace{4em}
\hfill \includegraphics[width=0.30\textwidth]{gefoerdert-durch-bmwi.jpg}


\end{titlepage}

\cleardoublepage


\tableofcontents

% \listoffigures 



\cleardoublepage

% \linenumbers

\chapter{Overview}
\label{cha:Zusammenfassung}

Aim of this interface specification is to link information systems for rental vehicles with travel information systems.
The reason for the linkage is to serve the trend driven requirement of creating intermodal travel chains in order to integrate rental systems to travel information systems.

The specification consists of:
\begin{itemize}
\item A Role Model of the involved partners,
\item A recommendation for a service hierarchy of information exchange with different qualities,
\item Interaction sequences to depict message ordering between the partners in order to serve the information linkage based on the specified service levels,
\item The data model,
\item Recommendations for the use of specific technologies for the data exchange and parsing,
\item Tables of allowed values for enumerations.
\end{itemize}
This version of IXSI contains the following - mobility broker specific - extensions:
\begin{itemize}
	\item Vehicle activation 
	\item Exchange of consumption data 
	\item Subscription of external bookings
	\item Creation and locking of VRS-users
	\item Synchronisation of vehicle settings 
\end{itemize}
This version of IXSI contains the following -  smartcar specific - extensions:
\begin{itemize}
	\item Synchronisation of vehicle settings
	\item Remote configuration navigation system and route monitoring
	\item Dialog public transport alternative (TODO)
\end{itemize}

\chapter{Rollenmodell}
\label{cha:Rollenmodell}

The role model describes the occuring roles in the information exchange.

\begin{figure*}[ht]
\centering
\includegraphics[width=0.8\textwidth]{rollenmodell_en.pdf}
\caption{Overview Roles.\label{fig:Rollenmodell}}
\end{figure*}

\section*{Travel Information System}
\label{sec:Rollenmodell:RIS}
\index{RIS}\index{Reiseinformationssystem}
The Travel Information System (TIS) is a information system, responsible for travel inquiries and covers the unification of travel options, the construction of travel chains, the calculation of a total price for a travel chain, the reservation of travel options for separated elements of the travel chain, the processing of data for representations in user interfaces and finally the presentation of information.

\subsection*{Use Cases}
\begin{itemize}
\item User inquiries the TIS concerning mobility options, using search filters and preferences. Search filters and preferneces might be starting- and traget- location, departure- and arrival- time, transportation modes, amount of (mode-) switchings, price range, etc. results are provided via user-interfaces of the TIS in the form of travel chains. 
\end{itemize}

\section*{Vehicle Sharing System}
\label{sec:Rollenmodell:FVS}
\index{FVS}\index{Fahrzeugverleihsystem}
The Vehicle Sharing System (VRS) is a information system, responsible for managing and booking of sharing vehicles. Vehicles may vary in type or might be attached or unattached to sharing stations.

\subsection*{Use Cases}
\begin{itemize}
\item A user books a vehicle via VRS to specific prices, times and stations and uses it.
\item A user inquiries the availability of a vehicle via VRS. 
\end{itemize}

\section*{User TIS}
\index{Benutzer!RIS}
User TIS – represents a legal person, which is authorized to book and utilize a travel chain under use of selected modes of transportation.

\subsection*{Use Cases}
\begin{itemize}
\item User sets up a travel inquiry towards the TIS.
\item User books a travel via TIS.
\end{itemize}

\section*{User VRS}
\index{Benutzer!FVS}
User VRS - represents a legal person, which is authorized to rent and use a vehicle.

\subsection*{Use Cases}
\begin{itemize}
\item User sets up a travel inquiry towards the VRS.
% \item Benutzer reserviert ein Fahrzeug über das FVS.
\item User books a travel via VRS.
\end{itemize}

\section*{Operator TIS}
\index{Betreiber!RIS}
Operator TIS – provides the TIS as a service for transportation service providers.

% \subsection{Anwendungsfälle}

\section*{Opreator VRS}
\index{Betreiber!FVS}
Operator VRS - provides the VRS as a service for vehicle rental companies.
% \subsection{Anwendungsfälle}

% \section{FVS-Standort}
% 
% Ein FVS-Standort (z.B. eine CarSharing-Station) stellt die Fahrzeuge eines Mobilitätsanbieters an einem Ort bereit.
% 
% % \subsection{Anwendungsfälle}
% 
% \section{Fahrzeug}
% 
% Fahrzeug – wird vom Mobilitätsanbieter über das FVS an einer Station als Mobilitätsangebot für den Kunden angeboten.

% \subsection{Anwendungsfälle}

\chapter{Hierarchical Model}
\label{cha:Hierachiemodell}
The hierarchical model describes different qualities of information coupling, based on service groups and serves as a recommendation concerning possible stages of implementation.
To realize a coupling between VRS and TIS, at least service 1 (static data) of both interacting parties has to be supported. In \cref{fig:depend}, dependencies between different services are depicted.

\begin{figure}[h]
\centering
\resizebox{1\columnwidth}{!} {
  \begin{tikzpicture}
  % Dienste
  \tikzstyle{ann} = [draw=none,fill=none,right]
  \node[rectangleTX, align=center](s1) {\textbf{Service 1} \\ Static Data};
  \node[rectangleTX, align=center, below left=10mm and -7mm of s1](s3) {\textbf{Service 3} \\ Availability \\ Subscription};
  \node[rectangleTX, align=center, left=5mm of s3](s2) {\textbf{Service 2} \\ Subscription \\ Handling};
  \node[rectangleTX, align=center, right=5mm of s3](s4) {\textbf{Service 4} \\ Booking \\ \vphantom{b}};
  \node[rectangleTX, align=center, right=5mm of s4](s6) {\textbf{Service 6} \\ Price \\ Information};
  \node[rectangleTX, align=center, below=10mm of s4](s5) {\textbf{Service 5} \\ Booking \\ Subscription};

  %   arrows
  \draw (s2) edge[->,thick=1cm] (s1);
  \draw (s3) edge[->,thick=1cm] (s1);
  \draw (s4) edge[->,thick=1cm] (s1);
  \draw (s5) edge[->,thick=1cm] (s4);
  \draw (s6) edge[->,thick=1cm] (s1);
  \end{tikzpicture}
}
\caption{IXSI Service Groups.}
\label{fig:depend}
\end{figure}


\section{Base Service A -- Session Handling}
\label{sec:Hierachiemodell:BasisdienstA}
Service A enables the authentication of end-customers towards the VRS.

\subsection*{Functions}
\begin{itemize}
\item Open / Close Session
\end{itemize}

\subsection*{Dependencies}
none

\section{Base Service B -- Subscription Handling)}
\label{sec:Hierachiemodell:BasisdienstB}
Service B contains a function to check the status of a subscription connection (heartbeat)

\subsection*{Functions}
\begin{itemize}
\item Heartbeat
\end{itemize}

\subsection*{Dependencies}
none

\section{Base Service C -- tokens}
\label{sec:Hierachiemodell:BasisdienstC}
Service C contains a function to create authentication tokens for users, which can be saved/ transferred instead of plaintext passwords.

\subsection*{Functions}
\begin{itemize}
\item Creation of tokens
\end{itemize}

\subsection*{Dependencies}
none


\section{Service 1 -- Static Data}
\label{sec:Hierachiemodell:Dienst1}
Service 1 serves the information exchange across vehicle rental companies and static data of booking targets. These include provider-, position-, and vehicle-data. 

For example, service 1 can be used, to display messages solely for existing locations of an VRS-provider in a TIS. 
\subsection*{Functions}
\begin{itemize}
\item Call of booking targets and provider information.
\end{itemize}

\subsection*{Dependencies}
none

\section{Service 2 -- Availability Query}
\label{sec:Hierachiemodell:Dienst2}
Service 2 serves for asynchronous calls of availability information.

The actual availability times of booking targets are called during the travel inquiry by the TIS at the VRS.
\subsection*{Functions}
\begin{itemize}
\item Calls of availabilities of booking targets
\item Calls of location capacities (Service 2a)
\end{itemize}

\subsection*{Dependencies}
\begin{itemize}
\item Service 1
\end{itemize}

\section{Service 3 -- Availability Subscription}
\label{sec:Hierachiemodell:Dienst3}
Services 2 serves the asynchronous exchange of availability information.

To accelerate the travel inquiry, the TIS can subscribe availability timescales of booking targets to avoid a query during the travel inquiry. After subscribing for an amount of booking targets, the VRS informs continuously about changes in availability timescales. 

\subsection*{Functions}
\begin{itemize}
\item Availability subscription
\item Availability information (push)
\item Subscription of location capacities (Service 3a)
\end{itemize}

\subsection*{Dependencies}
\begin{itemize}
\item Service 1
\item Base Service B
\end{itemize}


\section{Service 4 -- Booking}
\label{sec:Hierachiemodell:Dienst4}
Service 4 serves booking, booking changes and canceling of vehicles via the TIS on behalf of the customer of a VRS.

The booking of a vehicle requires that a customer authenticates himself towards the VRS. Therefore, authentication information is forwarded from the TIS to the VRS. The secret (Password, key, etc.) of the customer is not allowed to be saved due to security reasons. The TIS receives an authentication-token in case of a successful authentication of a user. It can either be saved on the TIS or the user's device. Using the authentication-token, the TIS can proceed queries concerning bookings and canceling. 
To allow booking changes, a booking can be replaced by another booking via a booking change request. In case of an impossible booking, is has to be secured that the initial booking retains it's validity. 

\subsection*{Functions}
\begin{itemize}
\item Authentication of users towards the VHR 
\item Booking Query
\item Query concerning booking changes/ canceling
\end{itemize}

\subsection*{Dependencies}
\begin{itemize}
\item Service 1
\end{itemize}

\section{Service 5 -- Booking subscription}
\label{sec:Hierachiemodell:Dienst5}
Service 5 serves the subscription of booking changes.

The TIS is able to subscribe conducted bookings at the VRS to inform users in case of changes, e.g., damaged vehicles.
\subsection*{Functions}
\begin{itemize}
\item Booking subscriptions 
\item Booking alert (Push)
\end{itemize}

\subsection*{Dependencies}
\begin{itemize}
\item Service 4
\item Base Service B
\end{itemize}

\section{Service 6 -- Price information}
\label{sec:Hierachiemodell:Dienst6}
Service 6 serves price information of rental services.

Through the transmission of starting and target location as well as departure and arrival time of the travel, the TIS is able to query price information at the VRS to inform the user. The VRS responds with an overall price and eventually individual price parts.

\subsection*{Functions}
\begin{itemize}
\item Query of Prices
\end{itemize}

\subsection*{Dependencies}
\begin{itemize}
\item Service 1
\end{itemize}



\chapter{Interaktionsprotokolle}
\label{cha:Interaktionsprotokolle}
Dieser Abschnitt gibt einen Überblick über die in IXSI verwendeten Interaktionsschemata. Zur Vereinfachung werden die Interaktionsschemata in den Sequenzdiagrammen informell ohne Verwendung der technischen Bezeichnungen der Funktionsaufrufe beschrieben. Grundsätzlich werden werden zwei Typen von Interaktionen verwendet: Das einfache und wohlbekannte Request/Response- bzw. Query-Interaktionsschema bei dem auf jede Anfrage des Clients (in diesem Fall das RIS) genau eine Antwort des Servers folgt (FVS). Weiterhin das Subscription-Schema bei dem einmalig ein Objekt durch den Client abonniert wird und dann Aktualisierungen an diesem Objekt laufend durch den Server geliefert werden. Hierbei bleibt der Kommunikationskanal die ganze Zeit über geöffnet.


\section{Überblick}
Das folgende Sequenzdiagramm gibt einen Überblick über eine beispielhafte Informationskopplung wie sie durch IXSI realisiert werden kann. Die im einzelnen verwendeten Dienste werden in den folgenden Abschnitten genauer beschrieben. In diesem Anwendungsfall holt ein Kunde beim RIS eine Reiseauskunft ein, bucht eine entsprechende Reise und lässt sich über Änderungen an der Buchung benachrichtigen.

Im ersten Block \textit{Austausch Buchungsziele} werden die vom \index{FVS}FVS zur Verfügung gestellten Buchungsziele mit dem \index{RIS}RIS ausgetauscht und relevante Buchungsziele abonniert (vgl. \cref{sec:Interaktionsprotokolle:Dienst1,sec:Interaktionsprotokolle:Dienst3}). Dies geschieht proaktiv ohne Involvierung eines Kunden.
Im Block \textit{Reiseauskunft} führt ein Kunde, z.\,B. mit einem Mobilgerät, eine Reiseauskunft beim RIS durch. Hierbei kommen mehrere Leihfahrzeuge in Frage, deren Verfügbarkeit dann synchron beim FVS abgefragt wird. Für die verfügbaren Fahrzeuge fragt das RIS zusätzlich eine Preisauskunft an. Als Ergebnis gibt das RIS eine Auswahl an möglichen Reiserouten / Verbindungen an den Kunden zurück. Da es sich um durch einen Kunden ausgelöste Kommunikation handelt, wird implizit eine Sitzung erstellt, in deren Kontext die Abfragen ausgeführt werden. Da sich der Kunde nicht auf seinem Gerät eingeloggt hat, wird eine anonyme Sitzung verwendet.
Im Block \textit{Reisebuchung} hat sich der Kunde für eine Reiseroute entschieden und möchte diese buchen. Hierzu loggt er sich auf seinem Mobilgerät ein, wodurch ein Token generiert wird (Block \emph{Anmeldung}). Mit diesem Token wird eine (nicht anonyme) Sitzung erstellt, in welcher der Buchungsvorgang durchgeführt wird. Hierzu übergibt das RIS die Buchungszielreferenz und einen Zeitvorschlag an das FVS. Das sendet eine Buchungsbestätigung, welche vom RIS an den Kunden weitergegeben wird. Zusätzlich abonniert das RIS die entsprechende Buchung beim FVS.
Im letzten Block \textit{Reiseüberwachung} wird der Kunde über Änderungen an der Buchung, die das RIS vom FVS erhält, benachrichtigt.
\begin{center}
\begin{sequencediagram}


\newthread{kunde}{:Kunde}
\newthread{ris}{:RIS}
\newinst[8]{fvs}{:FVS}


\begin{sdblock}{Austausch Buchungsziele}{Dienst 1, Dienst 3}

  \begin{call}{ris}{Anfrage}{fvs}{Liste Buchungsziele}
  \end{call}

  \begin{call}{ris}{Abonnement Buchungsziele}{fvs}{}
  \end{call}
  
  \begin{mess}{fvs}{Buchungszieländerung}{ris}
  \end{mess}
\end{sdblock}
\postlevel

\begin{sdblock}{Reiseauskunft}{}
  \begin{call}{kunde}{Reiseauskunft}{ris}{mögliche Reiserouten}

    \begin{sdblock}{Sitzung (anonym)}{Basisdienst A}
        
        \begin{sdblock}{Verfügbarkeitsauskunft}{Dienst 2}
          \begin{call}{ris}{Verfügbarkeitsauskunft Buchungsziele, Zeitraum}{fvs}{mögliche Buchungsziele}
          \end{call}
        \end{sdblock}

        \begin{sdblock}{Preisauskunft}{Dienst 6}
          \begin{call}{ris}{Preisauskunft Buchungsziele, Zeitraum, Distanz}{fvs}{Preis}
          \end{call}
        \end{sdblock}
      
    \end{sdblock}
  \end{call}
\end{sdblock}

\end{sequencediagram}

\smallskip


\begin{sequencediagram}
\newthread{kunde}{:Kunde}
\newthread{ris}{:RIS}
\newinst[8]{fvs}{:FVS}

\begin{sdblock}{Anmeldung}{}

  \begin{call}{kunde}{Anmeldung}{ris}{Token}
        
    \begin{sdblock}{Tokengenerierung}{Basisdienst C}


      \begin{call}{ris}{Benutzername, Passwort}{fvs}{Token}
      \end{call}

    \end{sdblock}

  \end{call}
\end{sdblock}
\postlevel

\begin{sdblock}{Reisebuchung}{}

  \begin{call}{kunde}{Auswahl Reiseroute, Token}{ris}{Buchungsbestätigung}
    \begin{sdblock}{Sitzung}{Basisdienst A}

      \begin{sdblock}{Fahrzeugbuchung}{Dienst 4}
        \begin{call}{ris}{Buchungszielreferenz, Zeitraumvorschlag}{fvs}{Buchungsreferenz, Zeitraum}
        \end{call}
      \end{sdblock}
      
    \end{sdblock}
    
  \end{call}
    
    \begin{sdblock}{Buchungsabonnement}{Dienst 5}

      \begin{call}{ris}{Buchungsreferenz}{fvs}{}
      \end{call}

    \end{sdblock}



\end{sdblock}

\postlevel

\begin{sdblock}{Reiseüberwachung}{Dienst 5}
\postlevel
  \begin{mess}{fvs}{Buchungsänderung}{ris}
  \end{mess}

  \begin{mess}{ris}{Buchungsänderung}{kunde}
  \end{mess}
  
\end{sdblock}

\end{sequencediagram}
\end{center}
\smallskip




\section{Dienst 1 -- Statische Daten}
\label{sec:Interaktionsprotokolle:Dienst1}

\subsection*{Abfrage Buchungsziele}

\begin{center}
\begin{sequencediagram}
\newthread{ris}{:RIS}
\newinst[8]{fvs}{:FVS}

\begin{sdblock}{Buchungszielaustausch}{}

\begin{call}{ris}{Buchungszielabfrage(Anbieter*)}{fvs}{Buchungszielliste}

\end{call}

\end{sdblock}

\end{sequencediagram}\\
\hfill\textit{* optional}
\end{center}
\smallskip

Als Basis für die Informationskopplung dient der Austausch von sogenannten Buchungszielen. Buchungsziele sind eine logische Repräsentation von einem oder mehreren Fahrzeugen mit gemeinsamen Eigenschaften, wie z.\,B. vom gleichen Anbieter bereitgestellt, gleicher Fahrzeugtyp und gleiche Verleihstation. Diese Eigenschaften sind statisch. Um nur Informationen zu Buchungszielen eines bestimmten Anbieters zu erhalten, kann nach Provider gefiltert werden. Die Übertragung wird vom RIS ausgelöst.

\subsection*{Abfrage Änderungen Buchungsziele}

\begin{center}
\begin{sequencediagram}
\newthread{ris}{:RIS}
\newinst[8]{fvs}{:FVS}

\begin{sdblock}{Änderungen Buchungsziele}{}

\begin{call}{ris}{Abfrage Änderungen Buchungsziele}{fvs}{Anbieterliste}
\end{call}

% \begin{messcall}{ris}{Asynchrone Nachricht}{fvs}
% \end{messcall}

\end{sdblock}
\end{sequencediagram}
\end{center}
\smallskip

Um nicht intervallweise alle Buchungszielinformationen übertragen zu müssen, kann mit dem Aufruf \texttt{ChangedProviders} angefragt werden, bei welchem Anbieter sich Änderungen seit einem bestimmten Zeitpunkt, vorgegeben durch den Parameter \texttt{timestamp}, ergeben haben. Zurückgegeben wird eine Providerreferenz, die wiederum bei beim Funktionsaufruf \texttt{Booking\-TargetsInfo} als Parameter übergeben werden kann.

\section{Dienst 2 -- Verfügbarkeitsauskunft}
\label{sec:Interaktionsprotokolle:Dienst2}

\subsection*{Abfrage Verfügbarkeit}

\begin{center}
\begin{sequencediagram}
\newthread{ris}{:RIS}
\newinst[8]{fvs}{:FVS}

\begin{sdblock}{Verfügbarkeitsauskunft}{}

\begin{call}{ris}{Buchungsziele/Gebiet, Zeitraum}{fvs}{Buchungsziele}
\end{call}

\end{sdblock}

\end{sequencediagram}
\end{center}
\smallskip

Um die konkreten Verfügbarkeiten abzufragen, sendet das RIS eine Anfrage die entweder eine Liste mit Buchungszielen oder ein geographisches Gebiet in Form einer Umgebungssuche oder als Rechteck und eine gewünschte Zeitperiode enthält. Ohne Angabe wird die Verfügbarkeit von allen Buchungszielen zurückgegeben. Als Antwort sendet das FVS eine Liste mit Buchungszielen und deren Verfügbarkeiten zurück.


\subsection*{Abfrage aktuelle Stationskapazität (Dienst 2a)}

\begin{center}
\begin{sequencediagram}
\newthread{ris}{:RIS}
\newinst[8]{fvs}{:FVS}

\begin{sdblock}{Abfrage Stationskapazität}{}

\begin{call}{ris}{Referenz Stationen / Gebiet}{fvs}{Liste Stationskapazitäten}
\end{call}

\end{sdblock}

\end{sequencediagram}
\end{center}
\smallskip
Das RIS kann die aktuellen Kapazitäten, bspw. zur Kartendarstellung, von Verleihstation anfragen. Hierzu wird eine Liste mit Standort IDs oder ein Gebiet übermittelt und eine Liste mit Standorten und deren aktueller Anzahl verfügbarer Fahrzeuge zurückgegeben.


\section{Dienst 3 -- Verfügbarkeitsabonnement}
\label{sec:Interaktionsprotokolle:Dienst3}

\subsection*{Verfügbarkeitsabonnement}
\label{subsec:Interaktionsprotokolle:Dienst3}

\begin{center}
\begin{sequencediagram}
\newthread{ris}{:RIS}
\newinst[8]{fvs}{:FVS}

\begin{sdblock}{Verfügbarkeit abonnieren}{}

\begin{call}{ris}{Buchungsziel, (Kündigung)*, Zeithorizont}{fvs}{}
\end{call}

\end{sdblock}
\postlevel
\begin{sdblock}{Verfügbarkeitsaktualisierungen}{}

\begin{mess}{fvs}{Verfügbarkeitsaktualisierung}{ris}
\end{mess}
\begin{mess}{fvs}{Verfügbarkeitsaktualisierung}{ris}
\end{mess}
\begin{mess}{fvs}{Verfügbarkeitsaktualisierung}{ris}
\end{mess}
\begin{mess}{fvs}{...}{ris}
\end{mess}
\end{sdblock}

\postlevel
\begin{sdblock}{Gesamte Verfügbarkeit}{}

\begin{call}{ris}{Maximale Anzahl Objekte}{fvs}{Buchungsziele}
\end{call}

\end{sdblock}



\end{sequencediagram}
\end{center}
\smallskip

Das RIS kann Informationen zu Buchungszielen abonnieren, um unmittelbar über Änderungen von Verfügbarkeiten informiert zu werden. Dies dient im Wesentlichen dazu, Reiseauskünfte ohne zusätzliche (synchrone) Anfrage an das FVS beantworten zu können.

Durch die initiale Anfrage \texttt{AvailabilitySubscriptionRequest} wird ein Abonnement (subscription) begonnen. Hierzu übergibt das RIS die entsprechende Buchungszielreferenz. Durch das Setzen des Flags Kündigung kann ein Abonnement storniert werden. Bei Änderungen an Verfügbarkeiten überträgt das FVS asynchron \texttt{AvailabilityPushMessage}s. Diese werden über den gleichen Kommunikationskanal geliefert, über den das Abonnement erstellt wurde. Beim Beenden des Kommunikationskanals werden alle Abonnements hinfällig.

Zur anfänglichen Synchronisierung aller Verfügbarkeiten kann das RIS die Funktion \texttt{Complete\-Availability\-Request} aufrufen.





\subsection*{Standortkapazitätabonnement (Dienst 3a)}
\label{subsec:Interaktionsprotokolle:Dienst3a}

\begin{center}
\begin{sequencediagram}
\newthread{ris}{:RIS}
\newinst[8]{fvs}{:FVS}

\begin{sdblock}{Standortkapazitäten abonnieren}{}

\begin{call}{ris}{Standortreferenz, (Kündigung)*}{fvs}{}
\end{call}

\end{sdblock}
\postlevel
\begin{sdblock}{Kapazitätsaktualisierungen}{}

\begin{mess}{fvs}{Standortkapazität}{ris}
\end{mess}

\begin{mess}{fvs}{Standortkapazität}{ris}
\end{mess}
\begin{mess}{fvs}{Standortkapazität}{ris}
\end{mess}
\begin{mess}{fvs}{...}{ris}
\end{mess}
\end{sdblock}
\postlevel

\begin{sdblock}{Vollständige Standortkapazitäten}{}

\begin{call}{ris}{}{fvs}{Standortkapazitäten}
\end{call}

\end{sdblock}



\end{sequencediagram}
\end{center}
\smallskip

Das RIS kann die Kapazitätsinformation von Standorten abonnieren. Der Interaktionsablauf ist analog zu \cref{subsec:Interaktionsprotokolle:Dienst3}.


\section{Dienst 4 -- Buchung / Buchungsänderung}
\label{sec:Interaktionsprotokolle:Dienst4}

\begin{center}
\begin{sequencediagram}
\newthread{ris}{:RIS}
\newinst[8]{fvs}{:FVS}

% \begin{sdblock}{OpenSession*}{}
%
% \begin{call}{ris}{}{fvs}{}
% \end{call}
%
% \end{sdblock}


\begin{sdblock}{Buchung}{}

\begin{call}{ris}{Buchungszielreferenz, Zeitraumvorschlag}{fvs}{Buchungsreferenz, Zeitraum}
\end{call}

\end{sdblock}
\postlevel

\begin{sdblock}{Buchungsänderung*}{}

\begin{call}{ris}{Vorschlag für neuen Zeitraum / Stornierung}{fvs}{Zeitraum}
\end{call}

\end{sdblock}

% \begin{sdblock}{CloseSession*}{}
%
% \begin{call}{ris}{}{fvs}{}
% \end{call}
%
% \end{sdblock}

\end{sequencediagram}
\end{center}
\smallskip

Um im Kundenauftrag ein Fahrzeug zu buchen, ist es erforderlich, dass das RIS den Kunden gegenüber dem FVS authentifizert. Hierzu gibt es drei Möglichkeiten, die in \cref{sec:Datenmodell:Auth} genauer dargestellt sind. In diesem Beispiel wird explizit eine Sitzung geöffnet und im Anschluss an die Transaktion wieder geschlossen. Danach kann eine Buchung durch den Aufruf von \texttt{Booking} mit Angabe der entsprechenden Buchungsziel ID und einem Vorschlag für einen Zeitraum durchgeführt werden. \blockquote{Vorschlag} deshalb, da das FVS z.\,B. den Zeitraum auf das verwendete Buchungsraster ändern kann. Als Antwort wird die verwendete Buchungsreferenz und der tatsächliche Buchungszeitraum zurückgegeben. Die Buchungsreferenz kann zur Überwachung der Buchung verwendet werden (vgl. \cref{sec:Interaktionsprotokolle:Dienst5}). Zur Änderung des Buchungszeitraums oder zur Stornierung kann \texttt{ChangeBooking} aufgerufen werden. Bei Änderung des Buchungsziels ist eine Stornierung und Neubuchung erforderlich.


\section{Dienst 5 --  Buchungsabonnement}
\label{sec:Interaktionsprotokolle:Dienst5}

\begin{center}
\begin{sequencediagram}
\newthread{ris}{:RIS}
\newinst[8]{fvs}{:FVS}

\begin{sdblock}{Buchungsabonnement}{}

\begin{call}{ris}{Buchungsreferenz, (Abostornierung)*}{fvs}{}
\end{call}

\end{sdblock}
\postlevel
\begin{sdblock}{Buchungsänderungen}{}

\begin{mess}{fvs}{Buchungsänderung}{ris}
\end{mess}

\begin{mess}{fvs}{Buchungsänderung}{ris}
\end{mess}
\begin{mess}{fvs}{Buchungsänderung}{ris}
\end{mess}
\begin{mess}{fvs}{...}{ris}
\end{mess}
\end{sdblock}
\postlevel

\begin{sdblock}{Vollständige Buchungsinformation}{}

\begin{call}{ris}{}{fvs}{Buchungsänderungen}
\end{call}

\end{sdblock}

\end{sequencediagram}
\end{center}
\smallskip

Das RIS kann Änderungen an Buchungen abonnieren, um diese Informationen dem Kunden weiterzugeben und ggfs. Alternativen anzubieten. Beispielsweise im Falle eines technischen Defekts an einem Fahrzeug kann das FVS das RIS darüber informieren, dass die Buchung nicht mehr möglich ist. Ebenfalls ist es möglich, eine Buchung als \blockquote{wieder möglich} festzulegen. Endgültig storniert werden kann eine Buchung nur vom Endkunden.

Der Interaktionsablauf ist analog zu \cref{subsec:Interaktionsprotokolle:Dienst3}.


\section{Dienst 6 -- Preisauskunft}
\label{sec:Interaktionsprotokolle:Dienst6}

% \subsection{Abfrage Preis}

\begin{center}
\begin{sequencediagram}
\newthread{ris}{:RIS}
\newinst[8]{fvs}{:FVS}

% \begin{sdblock}{OpenSession*}{}
%
% \begin{call}{ris}{}{fvs}{}
% \end{call}
%
% \end{sdblock}

\begin{sdblock}{Preisauskunft}{}

\begin{call}{ris}{Buchungszielreferenz, Vorschlag Zeitraum, Distanz}{fvs}{Preis}

\end{call}

\end{sdblock}

% \begin{sdblock}{EndSession*}{}
%
% \begin{call}{ris}{}{fvs}{}
% \end{call}
%
% \end{sdblock}

\end{sequencediagram}
\end{center}
\smallskip

Mit einer Anfrage \texttt{PriceInformationRequest} kann das RIS beim FVS eine Preisauskunft auf Basis von Buchungsziel ID, Zeitraum und zurückzulegende Distanz anfragen. Falls vorher eine Authentifizierung des Endkunden z.\,B. durch \texttt{OpenSession} stattgefunden hat, ist die Preisanfrage entsprechend des Kundenvertrags zu beantworten.

\chapter{Datenmodell}
\label{sec:Datenmodell}

\section{Dienst 1}
\label{subsec:Datenmodell:Dienst1}

\subsection*{Buchungsziel}

\input{xml/generated/BookingTargetType-docu}
% \TODO{XML Beispiel}

\subsection*{Verleihstation}

\input{xml/generated/PlaceType-docu}
% \TODO{XML Beispiel}

\subsection*{Provider}

\input{xml/generated/ProviderType-docu}
% \TODO{XML Beispiel}

\subsection*{Fahrzeugeigenschaft}

\input{xml/generated/AttributeType-docu}
% \TODO{XML Beispiel}


\section{Dienst 2}
\label{subsec:Datenmodell:Dienst1}

\subsection*{Buchungszieleigenschaften}

\input{xml/generated/BookingTargetPropertiesType-docu}
% \TODO{XML Beispiel}

\subsection*{Zeitrahmen}

\input{xml/generated/TimePeriodType-docu}
% \TODO{XML Beispiel}
\chapter{Technische Realisierung}
\label{sec:TechnischeRealisierung}
Die Schnittstelle verwendet die Standards XML und Websockets. Beide Standards werden bereits in verschiedenen Anwendungsbereichen erfolgreich eingestetzt.

\section{Nachrichtenkodierung}
Die Nachrichten zwischen den beiden System werden zur Übertragung als XML Dokumente dargestellt. Da es viele Werkzeuge zur Erzeugung und Verarbeitung von XML Dokumenten, vereinfacht dies die Realisierung der Schnittstelle. Des Weiteren erlaubt XML eine präzise Datentypdefinition, wodurch Uneindeutigkeiten vermieden werden können und eine grundlegende automatische Validierung von Nachrichten ermöglicht wird. Dadurch lassen sich insbesondere Fehler in der Implementierung einfacher feststellen. Falls sich der Overhead, der durch die Einführung von XML, entsteht als problematisch herausstellen sollte, besteht die Möglichkeit das Efficient XML Interchange (EXI) Protokoll einzusetzen. Der Einsatz von EXI würde die Größe der Nachrichten erheblich verringern, ohne die Vorteile der Verwendung von XML zu verlieren.

\section{Kommunikationskanal}
Da die Kommunikation nicht nach einem reinen Anfrage-Antwort-Schema abläuft, sondern beide System aktiv Nachrichten verschicken müssen (z.B. für Abo-Aktualisierungen) und eine geringe Antwortzeit wünschenswert ist, wird das WebSocket-Protokoll verwendet. Das WebSocket-Protokoll erlaubt es eine bestehende Verbindung der beiden Systeme herzustellen und über diese bidirektional Nachrichten auszutauschen.

\section{Verbindungssicherheit}
Die Schnittstelle kann auch über öffentliche Netze zur Verfügung gestellt werden. Um die Sicherheit der übermittelten Daten zu gewährleisten ist eine Verschlüsselung der Verbindung notwendig. Des Weiteren ist es notwendig sicherzustellen, dass die Schnittstelle nur von autorisierten Systemen genutzt werden kann. Beide Anforderungen werden vom SSL/TLS-Protokoll erfüllt. Dieses sollte verwendet werden, wenn die Verbindung nicht bereits durch andere entsprechende Maßnahmen (z.B. durch die Verwendung von Virtuellen Privaten Netzwerken (VPN)) gesichert ist.
\chapter{Nachrichten}
\label{sec:Nachrichten}
Die zwischen den Interaktionspartnern ausgetauschten Nachrichten basieren auf vier Nachrichtengrundtypen: Handshake, Request, Response und Update. Vom Typ Handshake sind nur die Nachrichten, die während der Handshake-Interaktion auf Dienstebene 0 ausgetauscht werden. Alle Nachrichten vom RIS an das FVS stellen Anfragen dar und sind daher vom Typ Request. Der Typ Response wird für direkte Antworten des FVS auf Anfragen des RIS verwendet. Um eine Zuordnung der Response-Nachrichten zu den zugehörigen Request-Nachrichten zu erlauben, wird jede Request-Nachrichte mit einer eindeutigen Transaction-ID markiert, die in der zugehörigen Response-Nachricht wieder mitgegeben werden muss. Eine Response-Nachricht kann anstelle ihres normalen Inhalts auch einen Fehler enthalten. Der letzte Nachrichtentyp „Update“ findet in den Fällen Verwendung, wenn das RIS fortlaufende Aktualisierungen vom FVS angefordert hat. Dazu wird dem RIS in der Antwort auf die Anfrage der fortlaufenden Aktualisierungen eine Abo-ID mitgeteilt, die in jeder zugehörigen Update-Nachricht mitgegeben wird.

\input{xml/generated/messages.tex}

\section{Fehler}
Die Nachrichten des Nachrichtentyps Response können alternativ zu ihrem sonstigen Inhalt einen Fehler enthalten. Ein Fehler besteht aus eine Fehlernummer, die den Typ des Fehlers definiert, einem kurzen und einem langen Fehlertext.

\input{xml/generated/error.tex}

\subsection{Fehlernummern}
Die Fehlernummern stellen die unterschiedlichen Typen von Fehlern dar und sind dreistellig. Die Fehlertypen spalten sich in zwei Gruppen. Kritische Fehler treten nur auf, wenn sich einer der beiden Interaktionspartner nicht Protokollkonform verhalten hat. Die Fehlernummer kritischer Fehler liegt zwischen 100 und 199. Die Nummern nicht kritischer Fehler liegen zwischen 200 und 299. Das FVS bricht, nachdem ein kritischer Fehler übertragen wurde, immer die entsprechende Verbindung zum RIS ab. Bei einem nicht kritischen Fehler ist nur die entsprechende Interaktion fehlgeschlagen und wird abgebrochen. Die Verbindung wird dadurch nicht beeinflusst und steht für weitere Interaktionen zur Verfügung.

110	Allgemeiner Autorisierungsfehler: Das RIS ist nicht autorisiert eine Verbindung zum FVS aufzubauen.
111	Verbindungslimit erreicht: Das RIS hält bereits zu viele geöffnete Verbindungen zum FVS.
112	IP nicht zugelassen: Die IP des RIS, unter der es versucht eine Verbindung zum FVS aufzubauen, ist nicht zugelassen.
120	Nicht Verstanden: Die Anfrage des RIS konnte nicht verarbeitet werden, da sie keine korrekte Form hatte.
130	Interner Serverfehler: Bei der Verarbeitung der Anfrage ist im FVS ein Fehler aufgetreten.
140, 240	Parameter Fehlerhaft: Die Parameter der Nachricht waren nicht im akzeptablen Bereich.
241	Versions-ID ungültig: Die Versions-ID der Anfrage konnte keiner existierenden Liste zugeordnet werden.
250	Benutzerauthentifizierung fehlgeschlagen: Der Benutzer konnte aus unbekannten Gründen nicht authentifiziert werden.
251	Benutzerauthentifizierung fehlgeschlagen (ungültige Daten): Der Benutzer konnte aufgrund ungültiger Authentifizierungsdaten nicht authentifiziert werden.
252	Benutzerauthentifizierung fehlgeschlagen (gesperrt): Der Benutzer konnte nicht authentifiziert werden, da sein Account gesperrt ist.
260	Benutzer-Token abgelaufen: Die Aktion konnte nicht durchgeführt werden, da das übermittelte Benutzer-Token nicht mehr gültig ist.
271	Stornierung fehlgeschlagen: Die Stornierung konnte nicht durchgeführt werden.
272	Buchung fehlgeschlagen: Die Buchung konnte nicht durchgeführt werden.

\subsection{XML Beispiel}
TODO

\section{Ebene 0 -- Handshake}
Der Handshake besteht aus fünf Nachrichtentypen. Diese sind alle abgeleitet vom abstrakten Nachrichtentyp Handshake.

\subsection{Liste der unterstützten Protokollversionen}
Nach dem Verbindungsaufbau durch das RIS sendet das FVS eine Nachricht vom Typ ProtocolVersionListHandshake mit der Bezeichnung „protocol-version-list-handshake“ mit einer Liste der von Protokollversionen. Diese Liste sollte alle Versionen des Protokolls enthalten mit denen mit dem FVS kommuniziert werden kann. Anstelle der Liste kann ggf. ein Autorisierungsfehler (110, 111, 112) oder ein interner Serverfehler (130) in der Nachricht enthalten sein.

\input{xml/generated/protocol-version-list-handshake.tex}

\subsection{Auswahl einer Protokollversion}
Das RIS wählt aus der erhaltenen Protokollversionsliste eine Protokollversion aus und sendet diese in einer Nachricht vom Typ ProtocolVersionSelectHandshake mit der Bezeichnung „protocol-version-select-handshake“ an das FVS.

\input{xml/generated/protocol-version-select-handshake.tex}

\subsection{Maximale unterstützte Dienstebene}
Nach erfolgter Auswahl einer Protokollversion, sendet das FVS eine Nachricht vom Typ ProtocolLevelMaxHandshake mit der Bezeichnung „protocol-level-max-handshake“ mit der maximalen vom FVS angebotene Dienstebene an das RIS. Alternativ kann die Nachricht ggf. auch einen Parameterfehler (140), Nicht-Verstanden-Fehler (120) oder internen Serverfehler (130) enthalten.

\input{xml/generated/protocol-level-max-handshake.tex}

\subsection{Auswahl einer Dienstebene}
Das RIS wählt eine Dienstebene gleich oder niedriger als die vom FVS angebotene maximale Dienstebene. Diese wird in einer Nachricht vom Typ ProtocolLevelSelectHandshake mit der Bezeichnung „protocol-level-select-handshake“ an das FVS gesendet.

\input{xml/generated/protocol-level-select-handshake.tex}

\subsection{Bestätigung}
Um den Handshake abzuschließen, sendet das FVS nach erhalt der Nachricht zur Auswahl der Dienstebene eine Nachricht vom Typ FinishHandshake mit der Bezeichnung „finish-handshake“ zur Bestätigung der Auswahl. Diese Nachricht kann ggf. anstelle der Bestätigung auch einen Parameterfehler (140), Nicht-Verstanden-Fehler (120) oder internen Serverfehler (130) enthalten. 

\input{xml/generated/finish-handshake.tex}

\subsection{XML Beispiel}
TODO

\section{Ebene 0 -- Abfrage der Betreiber}
Das RIS kann eine Liste der Betreiber vom FVS abfragen. Dazu sendet zunächst das RIS eine Nachricht vom Typ OperatorListRequest mit der Bezeichnung „operator-list-request“. Das FVS antwortet auf diese Nachricht mit einer Nachricht vom Typ OperatorListResponse mit der Bezeichnung „operator-list-response“.

\input{xml/generated/operator-list-messages.tex}

\subsection{Anfrage Betreiberliste}
Der Nachrichtentyp OperatorListRequest wird zur Anfrage von Betreiberlisten verwendet. Nachrichten dieses Typs können die Versions-ID einer aktuell bereits vorhandenen Betreiberliste enthalten.

\input{xml/generated/operator-list-request.tex}

\subsection{Antwort Betreiberliste}
Der Nachrichtentyp OperatorListResponse dient zur Übermittlung einer Betreiberliste vom FVS zu RIS. Er enthält eine Betreiberliste und eine zugehörige Versions-ID. Alternativ kann eine Nachricht dieses Typs auch ein „no-change“ Element enthalten, falls sich die aktuelle Betreiberliste nicht von der Liste mit der in der Anfrage gegebenen Versions-ID  unterscheidet. Falls ein Fehler aufgetreten ist, kann die Nachricht auch einen Nicht-Verstanden-Fehler (120), internen Serverfehler (130) oder Parameterfehler (240, 241) enthalten.

\input{xml/generated/operator-list-response.tex}

\subsection{XML Beispiel}
TODO

\section{Ebene 0 -- Abfrage der Stationen}
Das RIS kann eine Liste der Stationen vom FVS abfragen. Dazu sendet zunächst das RIS eine Nachricht vom Typ StationListRequest mit der Bezeichnung „station-list-request“. Das FVS antwortet auf diese Nachricht mit einer Nachricht vom Typ StationListResponse mit der Bezeichnung „station-list-response“.

\input{xml/generated/station-list-messages.tex}

\subsection{Anfrage Stationsliste}
Der Nachrichtentyp StationListRequest wird zur Anfrage von Stationslisten verwendet. Nachrichten dieses Typs müssen eine Betreiber-ID enthalten. Diese legt den Betreiber fest, dessen Stationen abgefragt werden sollen. Die Nachrichten können die Versions-ID einer aktuell bereits vorhandenen Stationsliste enthalten. Des weiteren können die Nachrichten einen Postleitzahlenbereich enthalten, der für eine Einschränkung der Stationsliste genutzt werden soll.

\input{xml/generated/station-list-request.tex}

\subsection{Antwort Stationsliste}
Der Nachrichtentyp StationListResponse dient zur Übermittlung einer Stationsliste vom FVS zu RIS. Er enthält eine Stationsliste und eine zugehörige Versions-ID. Alternativ kann eine Nachricht dieses Typs auch ein „no-change“ Element enthalten, falls sich die aktuelle Stationsliste nicht von der Liste mit der in der Anfrage gegebenen Versions-ID  unterscheidet. Die Liste muss genau die Stationen des in der Anfrage gegebenen Betreibers und ggf. die im gegebenen PLZ-Bereich liegen enthalten. Falls ein Fehler aufgetreten ist, kann die Nachricht auch einen Nicht-Verstanden-Fehler (120), internen Serverfehler (130) oder Parameterfehler (240, 241) enthalten.

\input{xml/generated/station-list-response.tex}

\subsection{XML Beispiel}
TODO

\section{Ebene 1 -- Abfrage der Buchungsziele}
Das RIS kann eine Liste der Buchungsziele vom FVS abfragen. Diese wird nach der Übertragung des aktuellen Standes durch Update-Nachrichten aktualisiert. Dazu sendet zunächst das RIS eine Nachricht vom Typ ReservationTargetRequest mit der Bezeichnung „reservation-target-request“. Das FVS antwortet auf diese Nachricht mit einer Nachricht vom Typ ReservationTargetResponse mit der Bezeichnung „reservation-target-response“. Aktualisierungen sendet das FVS als Nachrichten vom Typ ReservationTargetUpdate mit der Bezeichnung „reservation-target-update“. Ein bestehendes Aktualisierungsabonnement kann durch das RIS mit einer Nachricht vom Typ ReservationTargetUpdateCancelRequest mit der Bezeichnung „reservation-target-update-cancel-request“. Diese wird durch eine Nachricht vom Typ ReservationTargetUpdateCancelResponse mit der Bezeichnung „reservation-target-update-cancel-response“ vom FVS bestätigt.

\input{xml/generated/reservation-target-messages.tex}

\subsection{Anfrage Buchungsziele}
Buchungsziele werden durch eine Nachricht vom Typ ReservationTargetRequest angefordert. Diese muss eine Betreiber-ID enthalten. Diese schränkt die angefragten Buchungsziele auf Buchungsziele dieses Betreibers ein. Optional kann die Nachricht entweder einen Postleitzahlenbereich, ein GPS-Polygon oder eine Stations-ID enthalten. Diese dienen zur Filterung der zu übertragenden Buchungsziele.

\input{xml/generated/reservation-target-request.tex}

\subsection{Antwort Buchungszielliste}
Nachrichten vom Typ ReservationTargetResponse enthalten eine Buchungsliste und eine Abo-ID, die in folgenden zugehörigen Update-Nachrichten zur Referenzierung verwendet wird. Falls ein Fehler aufgetreten ist, kann die Nachricht auch einen Nicht-Verstanden-Fehler (120), internen Serverfehler (130) oder Parameterfehler (240, 241) enthalten.

\input{xml/generated/reservation-target-response.tex}

\subsection{Aktualisierung eines Buchungsziels}
Eine Aktualisierung der Buchungszielliste erfolgt durch Nachrichten vom Typ ReservationTargetUpdate. Wenn sich ein Buchungsziel geändert hat, enthält die Nachricht die veränderte Version des Buchungsziels. Die Buchungsziel-ID darf sich dabei nicht ändern. Wenn ein neues Buchungsziel hinzugekommen ist, enthält die Nachricht dieses Buchungsziel. Die Buchungsziel-ID des neuen Buchungsziels muss eindeutig über allen beim RIS noch vorliegenden Buchungszielen sein. Wenn ein Buchungsziel entfernt wurde, enthält die Nachricht nur eine Referenz auf das Buchungsziel in Form der Buchungsziel-ID. Der Inhalt des entfernten Buchungsziels wird nicht erneut übertragen.

\input{xml/generated/reservation-target-update.tex}

\subsection{Kündigung eines Buchungs-Abos}
Zur Kündigung eines Buchungszielabonnements wird vom RIS eine Nachricht vom Typ ReservationTargetUpdateCancelRequest verwendet. Diese enthält eine Referenz auf das Abo in Form einer Abo-ID.

\input{xml/generated/reservation-target-update-cancel-request.tex}

\subsection{Bestätigung der Kündigung}
Eine Nachricht zur Kündigung eines Buchungszielabonnements wird vom FVS mit einer Nachricht vom Typ ReservationTargetUpdateCancelResponse bestätigt. Die Nachricht kann anstelle einer Bestätigung auch einen Nicht-Verstanden-Fehler (120), internen Serverfehler (130) oder Parameterfehler (240) enthalten.

\input{xml/generated/reservation-target-update-cancel-response.tex}

\subsection{XML Beispiel}
TODO

\section{Ebene 2 -- Abfrage der Tarifliste}
Das RIS kann eine Liste der Tarife vom FVS abfragen. Dazu sendet zunächst das RIS eine Nachricht vom Typ TariffListRequest mit der Bezeichnung „tariff-list-request“. Das FVS antwortet auf diese Nachricht mit einer Nachricht vom Typ TariffListResponse mit der Bezeichnung „tariff-list-response“.

\input{xml/generated/tariff-list-messages.tex}

\subsection{Anfrage Tarifliste}
Der Nachrichtentyp TariffListRequest wird zur Anfrage von Tariflisten verwendet. Nachrichten dieses Typs müssen eine Betreiber-ID enthalten. Diese legt den Betreiber fest, dessen Tarife abgefragt werden sollen. Die Nachrichten können die Versions-ID einer aktuell bereits vorhandenen Tarifliste enthalten.

\input{xml/generated/tariff-list-request.tex}

\subsection{Antwort Tarifliste}
Der Nachrichtentyp TariffListResponse dient zur Übermittlung einer Tarifliste vom FVS zu RIS. Er enthält eine Tarifliste und eine zugehörige Versions-ID. Alternativ kann eine Nachricht dieses Typs auch ein „no-change“ Element enthalten, falls sich die aktuelle Tarifliste nicht von der Liste mit der in der Anfrage gegebenen Versions-ID  unterscheidet. Die Liste muss genau die Tarife des in der Anfrage gegebenen Betreibers enthalten. Falls ein Fehler aufgetreten ist, kann die Nachricht auch einen Nicht-Verstanden-Fehler (120), internen Serverfehler (130) oder Parameterfehler (240, 241) enthalten.

\input{xml/generated/tariff-list-response.tex}

\subsection{XML Beispiel}
TODO

\section{Ebene 3 -- Authentifizierung eines Benutzers}
Das RIS kann die Authentifizierung eines Benutzers beim FVS anfordern. Dazu sendet zunächst das RIS eine Nachricht vom Typ AuthenticationRequest mit der Bezeichnung „authentication-request“. Das FVS antwortet auf diese Nachricht mit einer Nachricht vom Typ AuthenticationResponse mit der Bezeichnung „authentication-response“.

\input{xml/generated/authentication-messages.tex}

\subsection{Anfrage Benutzerauthentifizierung}
Eine Nachricht vom Typ AuthenticationRequest wird zur Anforderung der Authentifizierung eines Benutzers verwendet. Die Nachricht enthält eine Betreiber-ID, ein Identifikationsmerkmal des Benutzers (z.B. Name oder Email-Adresse) und ein Authorisierungsmerkmal des Benutzers (z.B. Passwort).

\input{xml/generated/authentication-request.tex}

\subsection{Antwort Benutzerinformationen}
Eine Nachricht vom Typ AuthenticationResponse dient zur Übermittlung von Benutzerinformationen und Authentifizierungs-Token vom FVS zum RIS. Sie enthält einen Benutzerdatensatz. Falls ein Fehler aufgetreten ist, kann die Nachricht auch einen Nicht-Verstanden-Fehler (120), internen Serverfehler (130), Parameterfehler (240) oder Benutzerauthentifizierungsfehler (250, 251, 252) enthalten.

\input{xml/generated/authentication-response.tex}

\subsection{XML Beispiel}
TODO

\section{Ebene 3 -- Abfrage einer Preisauskunft}
Das RIS kann eine Preisauskunft vom FVS abfragen. Dazu sendet zunächst das RIS eine Nachricht vom Typ PriceinfoRequest mit der Bezeichnung „price-info-request“. Das FVS antwortet auf diese Nachricht mit einer Nachricht vom Typ PriceInfoResponse mit der Bezeichnung „price-info-response“.

\input{xml/generated/price-info-messages.tex}

\subsection{Anfrage Preis}
Eine Nachricht vom Typ PriceInfoRequest wird zur Anforderung einer Preisinformation verwendet. Die Nachricht enthält:
\begin{itemize}
\item eine Betreiber-ID,
\item eine User-ID als Referenz auf den Benutzer, für den die Preisauskunft gelten soll,
\item eine Buchungsziel-ID als Referenz auf das Buchungsziel, auf das sich die Preisauskunft beziehen soll,
\item eine Tarif-ID als Referenz auf den Tarif, auf den sich die Preisauskunft beziehen soll,
\item einen Zeitbereich, für den das Fahrzeug ausgeliehen werden soll,
\item eine Distanz, die mit dem Fahrzeug zurückgelegt werden soll
\item und ein optionales Ziel, an dem das Fahrzeug zurückgegeben werden soll.
\end{itemize}

\input{xml/generated/price-info-request.tex}

\subsection{Antwort Preisauskunft}
Eine Nachricht vom Typ PriceInfoResponse dient zur Übermittlung von Preisinformationen vom FVS zum RIS. Sie enthält einen Preisinformationsdatensatz bestehend aus Kostenbeträgen für die Anmietung des Fahrzeugs, die zurückgelegte Distanz, die Mietzeit und Sonstiges. Falls ein Fehler aufgetreten ist, kann die Nachricht auch einen Nicht-Verstanden-Fehler (120), internen Serverfehler (130) oder Parameterfehler (240) enthalten.

\input{xml/generated/price-info-response.tex}

\subsection{XML Beispiel}
TODO

\section{Ebene 3 -- Anfrage zur genauen Verfügbarkeit}
Das RIS kann eine Verfügbarkeitsauskunft vom FVS abfragen. Dazu sendet zunächst das RIS eine Nachricht vom Typ AvailabilityRequest mit der Bezeichnung „availability-request“. Das FVS antwortet auf diese Nachricht mit einer Nachricht vom Typ AvailabilityResponse mit der Bezeichnung „availability-response“.

\input{xml/generated/availability-messages.tex}

\subsection{Anfrage Verfügbarkeit}
Eine Nachricht vom Typ AvailabilityRequest wird zur Anforderung einer Verfügbarkeitsauskunft verwendet. Die Nachricht enthält:
\begin{itemize}
\item eine Betreiber-ID,
\item eine User-ID als Referenz auf den Benutzer, für den die Verfügbarkeitsauskunft gelten soll,
\item eine Buchungsziel-ID als Referenz auf das Buchungsziel, auf das sich die Preisauskunft beziehen soll,
\item einen Zeitbereich, für den das Fahrzeug ausgeliehen werden soll
\item und ein optionales Ziel, an dem das Fahrzeug zurückgegeben werden soll.
\end{itemize}

\input{xml/generated/availability-request.tex}

\subsection{Antwort Verfügbarkeit}
Eine Nachricht vom Typ AvailabilityResponse dient zur Übermittlung von Verfügbarkeitsinformationen vom FVS zum RIS. Sie enthält einen Verfügbarkeitsinformationsdatensatz bestehend aus einem Zeitbereich, für den ein Fahrzeug mit den bei der Anfrage angegebenen Parametern verfügbar ist, und einer Liste von Tarif-IDs, welche die möglichen Tarife für Buchungen darstellen. Der Zeitbereich muss mindestens den Zeitbereich, der bei der Anfrage angegeben wurde umfassen. Falls ein Fehler aufgetreten ist, kann die Nachricht auch einen Nicht-Verstanden-Fehler (120), internen Serverfehler (130) oder Parameterfehler (240) enthalten.

\input{xml/generated/availability-response.tex}

\subsection{XML Beispiel}
TODO

\section{Ebene 3 -- Auftrag zur Buchung}
Das RIS kann eine Buchung beim FVS durchführen. Dazu sendet zunächst das RIS eine Nachricht vom Typ ReservationRequest mit der Bezeichnung „reservation-request“. Das FVS antwortet auf diese Nachricht mit einer Nachricht vom Typ ReservationResponse mit der Bezeichnung „reservation-response“.

\input{xml/generated/reservation-messages.tex}

\subsection{Anfrage Buchung}
Eine Nachricht vom Typ ReservationRequest wird zur Anfrage für die Durchführung einer Buchung verwendet. Die Nachricht enthält:
\begin{itemize}
\item eine Betreiber-ID,
\item eine User-ID als Referenz auf den Benutzer, für den die Buchung durchgeführt werden soll,
\item ein Authentifizierungs-Token, zur Authentifizierung des Benutzers,
\item eine Tarif-ID als Referenz auf den Tarif, der für die Buchung verwendet warden soll,
\item eine Buchungsziel-ID als Referenz auf das Buchungsziel, auf das sich die Buchung beziehen soll,
\item einen Zeitbereich, für den das Fahrzeug ausgeliehen werden soll
\item und ein optionales Ziel, an dem das Fahrzeug zurückgegeben werden soll.
\end{itemize}

\input{xml/generated/reservation-request.tex}

\subsection{Antwort Buchung}
Eine Nachricht vom Typ ReservationResponse dient zur Bestätigung der Durchführung einer Buchung und zur einmaligen Übermittlung der Buchungsinformationen vom FVS zum RIS. Sie enthält einen Buchungsinformationsdatensatz. Die Informationen des Datensatzes müssen den Informationen die in der zugehörigen Anfrage angegeben wurden entsprechen. Falls ein Fehler aufgetreten ist, kann die Nachricht auch einen Nicht-Verstanden-Fehler (120), internen Serverfehler (130), Parameterfehler (240), Benutzer-Token-Fehler (260) oder Buchungsfehler (271) enthalten.

\input{xml/generated/reservation-response.tex}

\subsection{XML Beispiel}
TODO

%\appendix
%\input{content/X-Appendix.tex}

% \addchap{Index}                                      
\setglossarysection{section}
\printglossary[type=\acronymtype,style=long]

\renewcommand{\lstlistlistingname}{Schemaverzeichnis}
\lstlistoflistings

\renewcommand{\indexname}{Stichwortverzeichnis}
\printindex

% \bibliographystyle{plain}
%{\small\bibliography{literature/Literature}}


\vfill
\end{document}

